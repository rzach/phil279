\documentclass[bigger]{beamer}

\usetheme{Frankfurt}
\useoutertheme[subsection=false]{miniframes}
\setbeamercolor{structure}{parent=palette quaternary,fg=uofcred}
\setbeamercolor{section in toc}{fg=black,bg=white}
\setbeamercolor{alerted text}{fg=uofcred}
\setbeamercolor*{palette quaternary}{fg=white,bg=black}
\setbeamercolor*{palette secondary}{fg=black,bg=uofcyellow}
\setbeamercolor*{palette tertiary}{bg=uofcred,fg=white}
\setbeamercolor*{palette primary}{fg=white,bg=black}
\setbeamercolor*{sidebar}{fg=uofcred,bg=black!15!white}
\setbeamercolor*{palette sidebar primary}{fg=uofcred!10!black}
\setbeamercolor*{palette sidebar secondary}{fg=white}
\setbeamercolor*{palette sidebar tertiary}{fg=uofcred!50!black}
\setbeamercolor*{palette sidebar quaternary}{fg=black!10!white}

\setbeamercolor{titlelike}{parent=palette primary}
\setbeamercolor{frametitle}{fg=white,bg=black}
\setbeamercolor{frametitle right}{bg=gray!60!white}
\let\emph\alert
\beamertemplatenavigationsymbolsvertical
\def\bits{\begin{itemize}[<+->]}
\def\bens{\begin{enumerate}[<+->]}
\def\items#1{\item<#1>}
\def\uncovers#1{\uncover<#1>}
\def\pauses{\pause}

 % change to header-handout for handouts

% ====================
\title[Lecture 21]{Logic I F13 Lecture 21}
\date{November 28, 2013}
% ====================

% lecture header include

\usepackage{lplfitch,amsmath}
\usepackage{qtree,hyperref}
\usepackage{pgf,amssymb}

\author{Richard Zach}
\institute{Department of Philosophy\\
University of Calgary\\
  \href{http://ucalgary.ca/rzach/279}{ucalgary.ca/rzach/279}}

\definecolor{uofcred}{RGB}{227,39,38}
\definecolor{uofcyellow}{RGB}{255,210,0}

\DeclareSymbolFont{symbolsC}{U}{txsyc}{m}{n}
\DeclareMathSymbol{\strictif}{\mathrel}{symbolsC}{74}
\DeclareMathSymbol{\boxright}{\mathrel}{symbolsC}{128}

\let\IFF\Leftrightarrow
\let\iff\leftrightarrow
\let\impl\to

\def\T{{\color{green}\begin{colormixin}{25!black}\text{T}\end{colormixin}}}
\def\F{{\color{red}\begin{colormixin}{25!black}\text{F}\end{colormixin}}}

\long\def\subsec#1#2{\subsection{#1}\frame{\frametitle{#1} #2}}
\def\bit{\begin{itemize}[<1->]}
\def\eit{\end{itemize}}
\def\ben{\begin{enumerate}[<1->]}
\def\een{\end{enumerate}}

\makeatletter\let\@makefnmark\noindent\makeatother

%\setbeamercolor{footnote}{fg=black!70}

\def\foot#1{\footnotetext{\color{black!70}#1}}
\def\deemph#1{{\color{black!70}#1}}
\let\phi\varphi

\setbeamertemplate{theorems}[numbered]
%\useinnertheme{circles}
\setbeamertemplate{itemize subitems}[triangle]

\renewcommand{\beamertemplatetransparentcovereddynamic}{
  \beamersetuncovermixins
  {\opaqueness<1>{50}\opaqueness<2>{30}\opaqueness<3>{15}\opaqueness<4->{5}}%
  {\opaqueness<1>{50}\opaqueness<2>{30}\opaqueness<3>{15}\opaqueness<4->{5}}}
                                                                                
%\beamertemplatetransparentcovereddynamic

\defbeamertemplate*{footline}{my theme}
{%
  \leavevmode%
  \hbox{\begin{beamercolorbox}[wd=.5\paperwidth,ht=2.5ex,dp=1.125ex,leftskip=.3cm,rightskip=.3cm]{author in head/foot}%
    \insertframenumber/\inserttotalframenumber \hfil \usebeamerfont{author in head/foot}\insertshortauthor
  \end{beamercolorbox}%
  \begin{beamercolorbox}[wd=.5\paperwidth,ht=2.5ex,dp=1.125ex,leftskip=.3cm,rightskip=.3cm plus1fil]{title in head/foot}%
    \usebeamerfont{title in head/foot}Logic I F13---\insertshorttitle---\insertdate
  \end{beamercolorbox}}%
  \vskip0pt%
}

\begin{document}
\setlength{\fitchargwidth}{7em}
\setlength{\fitchprfwidth}{7em}

\frame{\frametitle{\insertshorttitle\ (\insertdate)}
\tableofcontents[hidesubsections]
}










\setlength{\fitchprfwidth}{14em}

\section{What is Logical Metatheory?}

\subsec{What is Logical Metatheory?}{

\bit
\item Most of Logic I is logical \emph{theory}: working \emph{in} a logical language, \emph{in} a logical proof system (Fitch), \emph{using} logical concepts like ``tautology'' and ``FO interpretation''
\item Logical \emph{meta}theory investigates language, proof system, logical concepts, and proves theorems about them
\eit

}

\subsec{Provability and Consequence}{

\bit
\item We write 
\bit
\item $A_1$, \dots, $A_n \models B$ if $B$ is a conseqence of $A_1$, \dots, $A_n$ and
\item $A_1$, \dots, $A_n \vdash B$ if $B$ has a formal proof from premises $A_1$, \dots, $A_n$. 
\eit
\item What is the relation between $\models$ and $\vdash$?
\eit

}

\subsec{Soundness}{

\bit
\item A proof system is \emph{sound} relative to a semantics if
\[
A_1, \dots, A_n \vdash B \Rightarrow A_1, \dots, A_n \models B
\]
\item This guarantees the proof system is \emph{safe}: you \emph{can't} prove anything that doesn't follow from the premises
\item In particular you can't prove anything that's not a tautology/FO validity from no premises 
\eit

\textbf{Soundness Theorem.} Fitch is a sound proof system

}

\subsec{Completeness}{

\bit
\item A proof system is \emph{complete} relative to a semantics if
\[
A_1, \dots, A_n \models B \Rightarrow A_1, \dots, A_n \vdash B
\]
\item This guarantees the proof system is \emph{as powerful as possible}: you \emph{can} prove anything that follows from the premises
\item In particular, you can prove all tautologies/FO validities from no premises
\eit

\textbf{Completeness Theorem (G\"odel 1930).}\\ Fitch is a complete proof system

}

\subsec{Satisfiability}{

\bit
\item A set of sentences $A_1$, \dots, $A_n$ is \emph{satisfiable} if there is a FO structure that makes all of the sentences true. (Cf. TT-possible)
\item A FO structue that makes a set of sentences true is called a \emph{model} of it
\item If a set is satisfiable, what kinds of models does it have?
\bit
\item Does it have (only) finite models?
\item Does it have (only) infinite models?
\item Does it have models with a specific domain (e.g., the natural numbers)?
\eit
\eit

\textbf{L\"owenheim-Skolem Theorem.} If a set has an infinite model, it has a model with domain the natural numbers.

}

\subsec{Decidability and Complexity}{

\bit
\item A \emph{decision problem} is a yes/no question, asked about a specific (syntactic) object, e.g.,
\bit
\item Given a sequence of symbols:\\
Is it awell-formed formula?
\item Given a sentence containing only $\land$, $\lor$, $\lnot$:\\
Is it TT-possible?
\item Given a sentence of the blocks language:\\
Is it a FO validity?
\item Given a sentence of an arbitrary FO language:\\
 Does it only have infinite models?
\eit 
\item A problem is \emph{decidable} if there is a computer program that will always find the correct answer to it, \emph{undecidable} otherwise.
\item Complexity of a problem: \emph{how fast} can a computer find the answer?
\eit

}

\section{Soundness}

\subsec{Soundness}{

\bit
\item Remember: Fitch is \emph{sound} if whenever there's a proof of $B$ from premises $A_1$, \dots, $A_n$, then it's guaranteed that $B$ is a conseqence of $A_1$, \dots, $A_n$
\item Fitch is, in fact, sound
\item How do you prove it?
\bit
\item We have to show this for all infinitely many possible proofs!
\item But: each proof uses only finitely many rules
\item Idea: use \emph{induction} on the number of rules used, i.e.,
\item Show that any proof that uses no rules at all is correct, and
\item Show that any proof correct to line $n$ is also correct to line $n+1$
\eit
\eit

}

\subsec{Correct Proof}{

\bit
\item A proof is \emph{correct down to line $n$} if every line which isn't a premise or an assumption of a subproof down to and including $n$ is a consequence of
\ben
\item the premises of the proof
\item all assumptions which are open at line~$n$, i.e., all assumptions of subproofs that line~$n$ is contained in
\een
\item E.g., the proof \qquad
\fitchprf{\pline[1.]{A}\\ \pline[2.]{B}}{
\pline[3.]{C}\\
\subproof{\pline[4.]{D}}{\pline[5.]{E}}}\\
is correct to line 5 if $\sf A, B \models C$ and $\sf A, B, D \models E$.
\eit

}

\subsec{Induction on Numbers of Steps in Proofs}{

\bit
\item The \emph{induction basis} is: any proof correct down to line~1\\
(trivially true, since line1 is a premise!)
\item The \emph{induction step} is:\\
If every proof is correct down to line $n$, then every proof is correct down to line $n+1$
\item In other words, if proofs are correct down to line $n$, the next step, if it is a correct application of a rule, does not break things
\eit
}

\subsec{One More Line: Reit}{

\begin{columns}
\begin{column}{3.5cm}
\hskip-2em\setlength{\fitchprfwidth}{8em}
\fitchprf{\pline[1.]{P}\\ \pline[2.]{Q}}{
\subproof{\pline[3.]{R}}{
\ellipsesline\\
\pline[$k$.]{A}\\
\ellipsesline\\
\subproof{\pline[$l$.]{S}}{
\ellipsesline\\
\pline[$(n+1)$.]{A}[\reit{$k$}]}}}
\end{column}
\begin{column}{7cm}
\vskip-2cm
\bits
\item Can only click on line $k$ if the innermost subproof that line $k$ is in has not ended before line $n+1$!
\item So: all assumptions open at line $k$ still open at line $n+1$
\item All proofs correct down to line $n$, so:
\bit
\item $\sf P, Q, R \models A$
\eit
\item But then also:
\bit
\item $\sf P, Q, R, S \models A$
\eit
\eit
\end{column}\end{columns}

}

\subsec{One More Line: $\land$Intro}{

\bit
\item Suppose we have a proof correct down to line $n$, and line $n+1$ is justified by $\land$Intro
\item If that step checks out, we know:
\bit
\item Line $n+1$ contains a conjunction, $A \land B$
\item There are exactly two support steps cited, $k$ and $l$
\item Line $k$ contains A, line $l$ contains B (or vice versa)
\item All assumptions open at line $k$ and $l$ are still open at line $n+1$ (there may be more) 
\eit
\item Since the proof is correct down to line $n$, in particular
\bit
\item A is a consequence of the premises and assumptions open at line $k$
\item B is a consequence of the premises and assumptions open at line $l$
\eit
\eit

}


\subsec{One More Line: $\land$Intro}{

\begin{columns}
\begin{column}{3.5cm}
\hskip-2em\setlength{\fitchprfwidth}{8em}
\fitchprf{\pline[1.]{P}\\ \pline[2.]{Q}}{
\subproof{\pline[3.]{R}}{
\ellipsesline\\[-.5ex]
\pline[$k$.]{A}\\
\subproof{\pline[$k+1$.]{S}}{
\ellipsesline\\[-.5ex]
\pline[$l$.]{B}\\
\subproof{\pline[$l+1$.]{T}}{
\ellipsesline\\[-.5ex]
\pline[$(n+1)$.]{}\\
\pline[\quad]{A\land B}[\landi{$k, l$}]}}}}
\end{column}
\begin{column}{7cm}
\vskip-2cm
\bits
\item Correct to line $k$ and line $l$:
\bit
\item $\sf P, Q, R \models A$
\item $\sf P, Q, R, S \models B$
\eit
\item Then also:
\bit
\item $\sf P, Q, R, S, T \models A$
\item $\sf P, Q, R, S, T \models B$
\eit
\item Show: $\sf P, Q, R, S, T \models A \land B$
\item Suppose not: then there is an interpretation that makes
\bit
\item $\sf P, Q, R, S, T$ true and 
\item $A \land B$ false
\eit
\item This would make either
\bit
\item $\sf P, Q, R, S, T$ true, A false, or
\item $\sf P, Q, R, S, T$ true, B false!
\eit
\eit
\end{column}\end{columns}

}

\subsec{One More Line: $\lnot$Intro}{

\begin{columns}
\begin{column}{3.5cm}
\hskip-2em\setlength{\fitchprfwidth}{8em}
\fitchprf{\pline[1.]{P}\\ \pline[2.]{Q}}{
\subproof{\pline[3.]{R}}{
\ellipsesline\\
\subproof{\pline[$k$.]{A}}{
\ellipsesline\\
\pline[$k'$.]{\bot}}
\ellipsesline\\
\pline[$(n+1).$]{\lnot A}[\lnoti{$k$--$k'$}]}}
\end{column}
\begin{column}{7cm}
\vskip-2cm
\bits
\item Correct to line $k'$:
\bit
\item $\sf P, Q, R, A \models \bot$
\eit
\item Show: $\sf P, Q, R \models \lnot A$
\item Suppose not: then there is an interpretation that makes
\bit
\item $\sf P, Q, R$ true and 
\item $\sf\lnot A$ false
\eit
\item This would make
\bit
\item $\sf P, Q, R, A$ true, $\bot$ false
\eit
\eit
\end{column}\end{columns}

}

\subsec{One More Line: $\exists$Elim}{

\begin{columns}
\begin{column}{3.5cm}
\hskip-2em\setlength{\fitchprfwidth}{8em}
\fitchprf{\pline[1.]{P}\\ \pline[2.]{Q}}{
\subproof{\pline[3.]{R}}{
\ellipsesline\\
\pline[$m$.]{\exists x\, A(x)}\\
\ellipsesline\\
\boxedsubproof[$k$.]{c}{A(c)}{
\ellipsesline\\
\pline[$k'$.]{B}}
\ellipsesline\\
\pline[$(n+1).$]{B}[\lexie{$m$}{$k$--$k'$}]}}
\end{column}
\begin{column}{7cm}
\vskip-1.5cm

\bits
\item Correct to $m$: $\sf P, Q, R \models \exists x\, A(x)$
\item<.-> Correct to $k'$: $\sf P, Q, R, A(c) \models B$
\item Suppose not $\sf P, Q, R \models B$: 
then there is a FO interpretation $I$ that makes
$\sf P, Q, R$ true and $B$ false
\item $\sf P, Q, R \models \exists x\,A(x)$, so at least one object $\alpha$ 
satisfies A(x)
\item $I^*$: just like $I$, except $c : \alpha$
\bit
\item makes A(c) true
\item makes P, Q, R true, \emph{B false} (don't contain c!)
\eit
\item Since $\sf P, Q, R, A(c) \models B$, $I^*$ also makes \emph{B true}: contradiction.
\eit
\end{column}\end{columns}

}

\section{Completeness}

\subsec{Completeness}{

\bit
\item Remember: (Completeness 1)\\
Fitch is \emph{complete} if:

\medskip
 whenever $B$ is a consequence of $A_1$, \dots, $A_n$, there's guaranteed to be a proof of $B$ from premises $A_1$, \dots, $A_n$
\medskip

\item Much harder to prove than soundness
\item Equivalent to: (Completeness 2)

\medskip
If a set of sentences is consistent, it is satisfiable

\medskip
(\emph{Consistent} means: there is no proof of $\bot$ from those sentences as premises)
\eit

}

\subsec{Completeness 2 Implies Completeness 1}{

\bit
\item Suppose any consistent set of sentences is satisfiable (Completeness 2)
\item Show: if $\sf A_1, \dots, A_n \models B$ then $\sf A_1, \dots, A_n \vdash B$ (Completeness 1)
\item Or show contrapositive:\\
if \emph{not} $\sf A_1, \dots, A_n \vdash B$ then \emph{not} $\sf A_1, \dots, A_n \models B$
\item Observe:
\bens
\item $\sf \sf A_1, \dots, A_n, \lnot B$ is satisfiable $\Rightarrow$ not: $\sf \sf A_1, \dots, A_n \models B$
\item $\sf A_1, \dots, A_n, \lnot B \vdash \bot$ $\Rightarrow$ $\sf A_1, \dots, A_n \vdash B$, so
\item not $\sf A_1, \dots, A_n \vdash B$ $\Rightarrow$ not $\sf A_1, \dots, A_n, \lnot B \vdash \bot$
\een
\eit

}

\subsec{Proof of (2)}{

\begin{columns}[t]
\begin{column}{2.5cm}
\fitchprf{\pline[1.]{A_1}\\
\ellipsesline\\
\pline[$n$.]{A_n}\\
\pline[$(n+1)$.]{\lnot B}}{
\ellipsesline\\
\pline[$m$.]{\bot}}
\end{column}
\begin{column}{7cm}\setlength{\fitchprfwidth}{8em}
$\Rightarrow$\fitchprf{\pline[1.]{A_1}\\
\ellipsesline\\
\pline[$n$.]{A_n}}{
\subproof{\pline[$(n+1)$.]{\lnot B}}{
\ellipsesline\\
\pline[$m$.]{\bot}}
\pline[$m+1$.]{\lnot\lnot B}[\lnoti{$(n+1)$--$m$}]\\
\pline[$m+2$.]{B}[\lnote{$(m+1)$}]}
\end{column}\end{columns}
}

\subsec{Completing the Proof}{

\bit
\item We're proving:\\
not $\sf A_1, \dots, A_n \vdash B$ $\Rightarrow$ not $\sf A_1, \dots, A_n \models B$
\item Assume: not $\sf A_1, \dots, A_n \vdash B$
\item We had (3), contrapositive of (2):\\
 not $\sf A_1, \dots, A_n \vdash B$ $\Rightarrow$ not $\sf A_1, \dots, A_n, \lnot B \vdash \bot$
\item So:  not $\sf A_1, \dots, A_n, \lnot B \vdash \bot$ (by MP), i.e.,\\
 $\sf A_1, \dots, A_n, \lnot B$ is consistent
\item So: $\sf A_1, \dots, A_n, \lnot B$ is satisfiable (Completeness 2)
\item We also had (1):\\
$\sf \sf A_1, \dots, A_n, \lnot B$ is satisfiable $\Rightarrow$ not $\sf \sf A_1, \dots, A_n \models B$
\item So not $\sf \sf A_1, \dots, A_n \models B$ (by MP). QED
\eit

}

\section{Decision Problems}


\subsec{Decision Problems}{

\bit
\item Given a sentence containing only $\land$, $\lor$, $\lnot$:\\
Is it true in the first line of its truth table?
\item Given a sentence containing only $\land$, $\lor$, $\lnot$:\\
Is it TT-possible?
\item Given a sentence of the blocks language:\\
Is it a FO validity?
\item Given a sentence of an arbitrary FO language:\\
 Does it only have infinite models?
\eit 
}

\subsec{An Easy Decision Problem}{

\begin{block}{Instance: Sentence A containing only $\land$, $\lor$, $\lnot$\\
Problem: Is A true when all atomic sentences are true?}

\bit
\item Decidable
\item Complexity is \emph{linear}, i.e.,
\item There are programs that compute the answer after $c\cdot n$
steps, where 
\bit 
\item $n$ is the number of symbols in A
\item $c$ is a constant number
\eit
\eit
\end{block}

}

\subsec{SAT: An Hard Decision Problem}{

\begin{block}{Instance: Sentence A containing only $\land$, $\lor$, $\lnot$\\
Problem: Is A TT-possible?}

\bit
\item Decidable
\item Complexity is \emph{unknown}
\item There are programs that compute the answer after $c\cdot 2^n$
steps (SAT is in EXPTIME)
\item For instance: search through all lines of truth table
\item Noone has proved yet that there aren't faster programs
\item SAT is \emph{NP-complete}
\item Finding complexity of SAT would settle \emph{P vs NP} problem
\eit
\end{block}

}

\subsec{2SAT: Another Easy Decision Problem}{

\begin{block}{Instance: Sentence A: conjunction of disjunctions of at most 2 atmic or negated atomic sentences\\
Problem: Is A TT-possible?}

\bit
\item Decidable
\item Complexity is \emph{linear}
\item Non-trivial algorithm!
\item Applications in VLSI circuit design, graph theory, tomography, planning turnaments (optimal home/away games) 
\eit
\end{block}

}

\subsec{Church-Turing Theorem}{

\begin{block}{Instance: Sentence A of any FO language\\
Problem: Is A a FO validity?}

\bit
\item Undecidable
\item Proved independently by Alonzo Church and Alan Turing in 1935
\item Reason we have Turing machines
\eit
\end{block}
}


\end{document} 





