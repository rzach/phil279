\documentclass[bigger]{beamer}

\usetheme{Frankfurt}
\useoutertheme[subsection=false]{miniframes}
\setbeamercolor{structure}{parent=palette quaternary,fg=uofcred}
\setbeamercolor{section in toc}{fg=black,bg=white}
\setbeamercolor{alerted text}{fg=uofcred}
\setbeamercolor*{palette quaternary}{fg=white,bg=black}
\setbeamercolor*{palette secondary}{fg=black,bg=uofcyellow}
\setbeamercolor*{palette tertiary}{bg=uofcred,fg=white}
\setbeamercolor*{palette primary}{fg=white,bg=black}
\setbeamercolor*{sidebar}{fg=uofcred,bg=black!15!white}
\setbeamercolor*{palette sidebar primary}{fg=uofcred!10!black}
\setbeamercolor*{palette sidebar secondary}{fg=white}
\setbeamercolor*{palette sidebar tertiary}{fg=uofcred!50!black}
\setbeamercolor*{palette sidebar quaternary}{fg=black!10!white}

\setbeamercolor{titlelike}{parent=palette primary}
\setbeamercolor{frametitle}{fg=white,bg=black}
\setbeamercolor{frametitle right}{bg=gray!60!white}
\let\emph\alert
\beamertemplatenavigationsymbolsvertical
\def\bits{\begin{itemize}[<+->]}
\def\bens{\begin{enumerate}[<+->]}
\def\items#1{\item<#1>}
\def\uncovers#1{\uncover<#1>}
\def\pauses{\pause}

 % change to header-handout for handouts

% ====================
\title[Lecture 22]{Logic I F13 Lecture 22}
\date{December 3, 2013}
% ====================

% lecture header include

\usepackage{lplfitch,amsmath}
\usepackage{qtree,hyperref}
\usepackage{pgf,amssymb}

\author{Richard Zach}
\institute{Department of Philosophy\\
University of Calgary\\
  \href{http://ucalgary.ca/rzach/279}{ucalgary.ca/rzach/279}}

\definecolor{uofcred}{RGB}{227,39,38}
\definecolor{uofcyellow}{RGB}{255,210,0}

\DeclareSymbolFont{symbolsC}{U}{txsyc}{m}{n}
\DeclareMathSymbol{\strictif}{\mathrel}{symbolsC}{74}
\DeclareMathSymbol{\boxright}{\mathrel}{symbolsC}{128}

\let\IFF\Leftrightarrow
\let\iff\leftrightarrow
\let\impl\to

\def\T{{\color{green}\begin{colormixin}{25!black}\text{T}\end{colormixin}}}
\def\F{{\color{red}\begin{colormixin}{25!black}\text{F}\end{colormixin}}}

\long\def\subsec#1#2{\subsection{#1}\frame{\frametitle{#1} #2}}
\def\bit{\begin{itemize}[<1->]}
\def\eit{\end{itemize}}
\def\ben{\begin{enumerate}[<1->]}
\def\een{\end{enumerate}}

\makeatletter\let\@makefnmark\noindent\makeatother

%\setbeamercolor{footnote}{fg=black!70}

\def\foot#1{\footnotetext{\color{black!70}#1}}
\def\deemph#1{{\color{black!70}#1}}
\let\phi\varphi

\setbeamertemplate{theorems}[numbered]
%\useinnertheme{circles}
\setbeamertemplate{itemize subitems}[triangle]

\renewcommand{\beamertemplatetransparentcovereddynamic}{
  \beamersetuncovermixins
  {\opaqueness<1>{50}\opaqueness<2>{30}\opaqueness<3>{15}\opaqueness<4->{5}}%
  {\opaqueness<1>{50}\opaqueness<2>{30}\opaqueness<3>{15}\opaqueness<4->{5}}}
                                                                                
%\beamertemplatetransparentcovereddynamic

\defbeamertemplate*{footline}{my theme}
{%
  \leavevmode%
  \hbox{\begin{beamercolorbox}[wd=.5\paperwidth,ht=2.5ex,dp=1.125ex,leftskip=.3cm,rightskip=.3cm]{author in head/foot}%
    \insertframenumber/\inserttotalframenumber \hfil \usebeamerfont{author in head/foot}\insertshortauthor
  \end{beamercolorbox}%
  \begin{beamercolorbox}[wd=.5\paperwidth,ht=2.5ex,dp=1.125ex,leftskip=.3cm,rightskip=.3cm plus1fil]{title in head/foot}%
    \usebeamerfont{title in head/foot}Logic I F13---\insertshorttitle---\insertdate
  \end{beamercolorbox}}%
  \vskip0pt%
}

\begin{document}
\setlength{\fitchargwidth}{7em}
\setlength{\fitchprfwidth}{7em}

\frame{\frametitle{\insertshorttitle\ (\insertdate)}
\tableofcontents[hidesubsections]
}










\setlength{\fitchprfwidth}{14em}

\section{Logical Metatheory}

\subsec{Truth-functional Completeness}{

\bit
\item Every truth function can be expressed by a sentence containing only
\ben
\item $\{\land, \lnot\}$,
\item $\{\lor, \lnot\}$, 
\item $\{\to, \lnot\}$,
\item $\{\to, \bot\}$, 
\item $\{\mid\}$: not-both, truth table: \F\T\T\T 
\item $\{\downarrow\}$: neither-nor, truth table: \F\F\F\T
\een

\item Others, e.g., $\{\land, \lor\}$, $\{\liff, \lnot\}$ are not
  truth-functionally complete

\eit

}

\subsec{Semantics}{

\bit
\item A \emph{truth-value assignment} is an assignment of \T{} or \F{} to the atomic sentences (schematic letters in the truth-functional form)
\item A \emph{FO structure} is a non-empty domain together with
\bit
\item extensions for each predicate symbol
\item objects in the domain for each constant symbol
\item functions for each function symbol
\eit
\item A tautology is a sentence (the truth-functional form of) which is true in all truth-value assignments
\item A FO validity is a sentence that's true in all FO interpretations 
\eit

}


\subsec{Soundness and Completeness}{

\bit
\item Soundness

Arguments have formal proofs only if they are valid\\
If there is a proof of B from premises $\sf A_1, \dots A_n$, then B is a consequence of $\sf A_1, \dots A_n$

\item Completeness

Arguments have formal proofs if they are valid\\
If B is a FO consequence of $\sf A_1, \dots A_n$, then there is a proof of B from premises $\sf A_1, \dots A_n$\\[2ex]
Proved by Kurt G\"odel (1929)
\eit
}

\subsec{Other Proof Systems: Resolution}{

\bit
\item Fitch is a proof system for \emph{validity/tautologies}
\item Also possible to design proof systems for dual notion: \emph{unsatisfiability}\\
(A is a tautology $\Leftrightarrow$ $\sf\lnot A$ is unsatisfiable) 
\item A \emph{clause} is a disjunction $\sf A_1 \lor \ldots \lor A_n$
where each $\sf A_i$ is atomic or negated atomic.
\item The resolution rule:
\[
\fitchctx{\pline{A_1 \lor \ldots \lor A_n \lor C}\\
\pline{B_1 \lor \ldots \lor B_m \lor \lnot C}\\
\fpline{A_1 \lor \ldots \lor A_n \lor B_1 \lor \ldots \lor B_m}}
\]
\item Preserves \emph{joint satisfiability}
\item If you can prove the empty clause $\bot$ from a set of clauses, they can't be
  jointly satisfiable.
\eit

}

\section{Theories and Decidability}

\subsec{Church-Turing Theorem}{

\begin{block}{Instance: Sentence A of any FO language\\
Problem: Is A a FO validity?}

\bit
\item Undecidable
\item Proved independently by Alonzo Church and Alan Turing in 1935
\eit
\end{block}
}

\subsec{Decidable Classes}{

\bit
\item The decision problem \emph{in general} is undecidable
\item But special cases \emph{can} be decided, e.g.:
\eit
\begin{block}{Instance: Sentence A with only 1-place predicate symbols\\
Problem: Is A a FO validity?}

\bit
\item Decidable
\item Proved by Leopold L\"owenheim (1915)
\eit
\end{block}
}

\subsec{Theories}{

\bit
\item A set of sentences of FOL also called a \emph{theory}, and the sentences in it \emph{axioms}
\item Some (types of) FO structures can be characterized 
as the models of a theory
\item Examples:
\bit
\item Mathematical theories (theory of orders, group theory, arithmetic)
\item KR classification systems, e.g., SNOMED-CT
\item Mereology, theories of truth, scientific theories
\eit
\eit

}

\subsec{The Axiomatic Method}{

\bit
\item Theories + logic: what follows from axioms?
\item Axiomatic method: do science by investigating what follows from
  the axioms of a theory 
\item Logic can also determine:
\bit
\item Are axioms (in)consistent?
\item Are axioms independent, or is one superfluous?
\eit
\item Paradigm of axiomatic method: geometry (Euclid)
\eit
}

\subsec{Examples of Theories: Linear Orders}{

A relation $\preceq$ on set $O$ is a \emph{linear order} if
it is a model of the theory LO with axioms:
\begin{align*}
& \forall x\forall y((x \preceq y \land y \preceq x) \to x = y) & \text{Antisymmetry}\\
& \forall x\forall y\forall z((x \preceq y \land y \preceq z) \to x \preceq z) &\text{Transitivity}\\
& \forall x\forall y(x \preceq y \lor y \preceq x) & \text{Totality}
\end{align*}
Every total relation is reflexive: 
\[
{\sf LO} \models \forall x\, x \preceq x
\]
}

\subsec{Examples of Theories: Robinson's Q}{

Theories of arithmetic, such as Robinson's theory Q:
\begin{align*}
 \lnot\exists x\, (x + 1) & = 0\\
 \forall x(x \neq 0  \to \exists y\, (y+1) & = x)\\
 \forall x\forall y((x + 1) = (y + 1)  \to x & = y)\\
 \forall x\,(x + 0) & = x\\
 \forall x\forall y\, (x + (y+1)) & = ((x + y) +1) \\
 \forall x\,(x \times 0) & = 0\\
 \forall x\forall y\, (x \times (y + 1)) & = ((x \times y) + x)
\end{align*}
}



\subsec{Examples of Theories: SNOMED-CT}{

\begin{align*}
\texttt{bacterial } & \texttt{pneumonia} = \\
& \texttt{is-a|bacterial infectious disease}\\
& \texttt{is-a|infective pneumonia}\\
& \texttt{causative agent|bacteria}\\
& \texttt{finding site|lung structure}\\[2ex]
%\begin{align*}
\forall x(& BacterialPneumonia(x) \liff  \\
& BacterialInfectiousDisease(x) \land\\
& InfectivePneumonia(x) \land \\
& \exists y(HasCausativeAgent(x, y) \land Bacteria(y)) \land\\
&\exists y(HasFindingSite(x, y) \land LungStructure(y)))
\end{align*}
}

\subsec{Examples of Theories: SNOMED-CT}{

\bit
\item Over 300,000 concepts (predicate symbols), e.g.,\\
\bit 
\item 1-place predicates:\\
parts of body, findings, organisms, physical objects, procedures, substances, diseases, \dots
\item 2-place predicates:\\
has finding site, has causative agent, with method, has active ingredient, laterality is, using device, \dots
\eit
\item About 1,000,000 descriptions (axioms)
\item SNOMED-CT is decidable
\eit

}

\subsec{Examples of Theories: Mereology}{

\bit
\item Mereology: the theory of the part-whole relation (\emph{metaphysics})
\item Primitive relation: Pt(x, y), ``x is a part of y''
\item Some axioms:
\begin{align*}
& \forall x\, Pt(x, x) & \text{Reflexivity}\\
& \forall x\forall y\forall z((Pt(x,y) \land Pt(y, z)) \to Pt(x,z)) & \text{Transitivity}\\
& \forall x\forall y((Pt(x, y) \land Pt(y, x)) \to x = y) & \text{Antisymmetry}
\end{align*}
\item Defined properties and relations
\[
PP(x, y) \dots Pt(x, y) \land x \neq y \qquad At(x) \dots \lnot\exists y\,PP(y, x)
\]
\item Different theories settle questions differently, e.g.,
\bit
\item Are there atoms?
\item Does everything comprise at least one atom?
\item Is everything made of atomless ``gunk''?
\eit
\eit

}

\subsec{Property Theories and Grelling's Paradox}{

\bit
\item Primitive relation: App(x ,y), ``x applies to y''
\item Proposed axiom (``comprehension''): For any wff P(y),
\[\sf
\exists x\forall y(App(x, y) \liff P(y))
\]
\item Axiom is \emph{inconsistent} (contradictory)
\item A property is \emph{heterological} if it does not apply to itself, i.e., $
\sf \lnot Ap(x, x)$
\item Is the property of being heterological itself heterological?
\eit
\fitchprf{\pline{\forall y(Ap(h, y) \liff \lnot Ap(y, y))}}{
\pline{Ap(h, h) \land \lnot Ap(h, h)}}
}

\subsec{Completeness of Theories}{

\bit 
\item A theory T is \emph{complete} if for every sentence A in its language,
either $\sf T \models A$ or $\sf T \models \lnot A$
\item Every complete theory is decidable!
\item Some incomplete theories are still decidable (e.g., LO)
\item Some incomplete theories are incomplete\emph{able}: no consistent extension is complete
\item Philosophical upshot of this: truth in the intended model(s) of the theory outstrips provability from the theory
\eit
\textbf{G\"odel's Incompleteness Theorem (1930)}\\
Arithmetic, set theory, mereology are incompleteable

}



\section{Philosophy and Nonstandard Logics}

\subsec{Logical Consequence and FO Consequence}{

\bit
\item Philosophers interested in \emph{valid arguments}, i.e.,
those where the conclusion is a \emph{logical consequence} of the premises
\item Definition: There is no logically possible circumstance where the 
premises are true and the conclusion is false
\bit
\item \emph{Important:} It does not say ``it \emph{isn't the case} that the premises
are true and the conclusion is false''
\item That would make every argument with 
\bit
\item true premises, true conclusion
\item false premises, true conclusion
\item false premises, false conclusion
\eit 
logically valid.  But that's not the case.
\item It says ``it is \emph{impossible} that the premises could be true and the conclusion false!''
\eit
\item Difficulty: What logically possible circumstances are there?
\eit

}

\subsec{What Logic Does For Logical Validity}{

\bit
\item Truth-tables, FO interpretations, Fitch give \emph{sufficient conditions}
for validity, i.e.,
\bit
\item Every tautologically valid argument is logically valid
\item Every FO valid argument is logically valid
\item Every argument with a formal proof in Fitch is logically valid (soundness!)
\eit
\item Also provide tools to do more:
\bit
\item FO interpretations show why consequence fails
\item Show which relations between predicate symbols would suffice for logical validity
\item Provide ways to symbolize these relations
\item Formal theories about informal concepts can capture these relations, or elucidate consequences of assumptions about them (axiomatic method)
\eit
\eit

}

\subsec{Nonstandard Logics}{

\bit
\item Formal models of logical consequence make a number of simplifying assumptions:
\bit
\item Only \emph{determinate} properties allowed, e.g, no vague properties
\item Every (atomic) sentence either \T{} or \F; not both and nothing in between 
\item Every constant must refer, i.e., no empty names
\item Only truth-functional connectives, e.g., no subjunctive contionals, ``because'', or tenses
\eit
\item Non-standard logics: expand FO logic to deal with these
\eit

}

\subsec{Many-valued Logic}{
\def\I{{\color{blue}\mathsf{U}}}
\bit
\item Add to the truth-values \T{} and \F, e.g.,
\bit
\item ``Undetermined'': neither true nor false
\[
\begin{array}{c|c}
P & \lnot P\\
\hline
\T & \F \\
\I & \I\\
\F & \T
\end{array}
\qquad
\begin{array}{cc|c}
P & Q & (P \land Q)\\
\hline
\T & \T & \T\\
\T & \I & \I\\
\T & \F & \F\\
\I & \T & \I\\
\I & \I & \I\\
\I & \F & \F\\
\F & \T & \F\\
\F & \I & \F\\
\F & \F & \F
\end{array}
\qquad
\begin{array}{cc|c}
P & Q & (P \lor Q)\\
\hline
\T & \T & \T\\
\T & \I & \T\\
\T & \F & \T\\
\I & \T & \T\\
\I & \I & \I\\
\I & \F & \I\\
\F & \T & \T\\
\F & \I & \I\\
\F & \F & \F
\end{array}
\]
\item ``Inconsistent'': both true and false
\item Fuzzy truth values: any number between 0 and 1
\eit
\eit

}

\subsec{Modal Logic}{

\bit
\item Alethic logic\\
``It is possible that'' ($\Diamond$), ``it is necessary that'' ($\Box$)
\[
\Box A \to A \qquad \Diamond\Box A \to \Box A
\]
\item Epistemic logic: 
``It is known that'' (K)
\[
KK A \to K A
\]
\item Conditional logic\\
Subjunctive conditionals, ``if it were true that \dots, then\\ \quad it would be true that --- ---'' ($\strictif$)
\[
(A \strictif B) \to (A \to B)
\]
\item Temporal logic\\
``It was true that'' (P), ``It will be true that'' (F)
\[
F\,P\, A \to (P\, A \lor A \lor F\, A)
\]
\eit

}

\subsec{Christmas Trick}{

If the first sentence on this slide is true, \\\qquad then Santa Claus exists.

Proof that Santa Claus exists:
\bens
\item The first sentence on this slide is true\\\qquad (Assumption for
  conditional proof).
\item If the first sentence on this slide is true, then Santa Claus exists (from 1, since if S is true, then S).  
\item Santa Claus exists (from (1) and (2), by modus ponens)
\item If the first sentence on this slide is true, Santa Claus exists\\\qquad(from (1)--(3), by conditional proof).
\item We've just proved (4) [= the first sentence on this slide], so the first sentence on this slide is true.
\item Santa Claus exists (from (4) and (5), by modus ponens)
\een
}

\end{document} 





