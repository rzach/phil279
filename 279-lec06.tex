\documentclass[bigger]{beamer}

\usetheme{Frankfurt}
\useoutertheme[subsection=false]{miniframes}
\setbeamercolor{structure}{parent=palette quaternary,fg=uofcred}
\setbeamercolor{section in toc}{fg=black,bg=white}
\setbeamercolor{alerted text}{fg=uofcred}
\setbeamercolor*{palette quaternary}{fg=white,bg=black}
\setbeamercolor*{palette secondary}{fg=black,bg=uofcyellow}
\setbeamercolor*{palette tertiary}{bg=uofcred,fg=white}
\setbeamercolor*{palette primary}{fg=white,bg=black}
\setbeamercolor*{sidebar}{fg=uofcred,bg=black!15!white}
\setbeamercolor*{palette sidebar primary}{fg=uofcred!10!black}
\setbeamercolor*{palette sidebar secondary}{fg=white}
\setbeamercolor*{palette sidebar tertiary}{fg=uofcred!50!black}
\setbeamercolor*{palette sidebar quaternary}{fg=black!10!white}

\setbeamercolor{titlelike}{parent=palette primary}
\setbeamercolor{frametitle}{fg=white,bg=black}
\setbeamercolor{frametitle right}{bg=gray!60!white}
\let\emph\alert
\beamertemplatenavigationsymbolsvertical
\def\bits{\begin{itemize}[<+->]}
\def\bens{\begin{enumerate}[<+->]}
\def\items#1{\item<#1>}
\def\uncovers#1{\uncover<#1>}
\def\pauses{\pause}

 % change to header-handout for handouts

% ====================
\title[Lecture 6]{Logic I F13 Lecture 6}
\date{September 26, 2013}
% ====================


%NOTE remember to do truth table example (and put one in these slides)
%CUt last section/look at lec 7

% lecture header include

\usepackage{lplfitch,amsmath}
\usepackage{qtree,hyperref}
\usepackage{pgf,amssymb}

\author{Richard Zach}
\institute{Department of Philosophy\\
University of Calgary\\
  \href{http://ucalgary.ca/rzach/279}{ucalgary.ca/rzach/279}}

\definecolor{uofcred}{RGB}{227,39,38}
\definecolor{uofcyellow}{RGB}{255,210,0}

\DeclareSymbolFont{symbolsC}{U}{txsyc}{m}{n}
\DeclareMathSymbol{\strictif}{\mathrel}{symbolsC}{74}
\DeclareMathSymbol{\boxright}{\mathrel}{symbolsC}{128}

\let\IFF\Leftrightarrow
\let\iff\leftrightarrow
\let\impl\to

\def\T{{\color{green}\begin{colormixin}{25!black}\text{T}\end{colormixin}}}
\def\F{{\color{red}\begin{colormixin}{25!black}\text{F}\end{colormixin}}}

\long\def\subsec#1#2{\subsection{#1}\frame{\frametitle{#1} #2}}
\def\bit{\begin{itemize}[<1->]}
\def\eit{\end{itemize}}
\def\ben{\begin{enumerate}[<1->]}
\def\een{\end{enumerate}}

\makeatletter\let\@makefnmark\noindent\makeatother

%\setbeamercolor{footnote}{fg=black!70}

\def\foot#1{\footnotetext{\color{black!70}#1}}
\def\deemph#1{{\color{black!70}#1}}
\let\phi\varphi

\setbeamertemplate{theorems}[numbered]
%\useinnertheme{circles}
\setbeamertemplate{itemize subitems}[triangle]

\renewcommand{\beamertemplatetransparentcovereddynamic}{
  \beamersetuncovermixins
  {\opaqueness<1>{50}\opaqueness<2>{30}\opaqueness<3>{15}\opaqueness<4->{5}}%
  {\opaqueness<1>{50}\opaqueness<2>{30}\opaqueness<3>{15}\opaqueness<4->{5}}}
                                                                                
%\beamertemplatetransparentcovereddynamic

\defbeamertemplate*{footline}{my theme}
{%
  \leavevmode%
  \hbox{\begin{beamercolorbox}[wd=.5\paperwidth,ht=2.5ex,dp=1.125ex,leftskip=.3cm,rightskip=.3cm]{author in head/foot}%
    \insertframenumber/\inserttotalframenumber \hfil \usebeamerfont{author in head/foot}\insertshortauthor
  \end{beamercolorbox}%
  \begin{beamercolorbox}[wd=.5\paperwidth,ht=2.5ex,dp=1.125ex,leftskip=.3cm,rightskip=.3cm plus1fil]{title in head/foot}%
    \usebeamerfont{title in head/foot}Logic I F13---\insertshorttitle---\insertdate
  \end{beamercolorbox}}%
  \vskip0pt%
}

\begin{document}
\setlength{\fitchargwidth}{7em}
\setlength{\fitchprfwidth}{7em}

\frame{\frametitle{\insertshorttitle\ (\insertdate)}
\tableofcontents[hidesubsections]
}










\section[Tautological Validity]{Logical Validity and Tautological Validity}


\subsec{Validity of Arguments}{

\begin{block}{Valid Arguments}

\fitcharg{
$\left.\begin{array}{@{}c@{}}
\formula{P}_1 \\
\vdots\\
\formula{P}_n
\end{array}
\right\} \quad \text{Premises}$}{
\formula{C\qquad  \text{Conclusion}}}
\end{block}

\bit
\item Argument is \emph{valid} if 
\bit 
\item every possible world in which $P_1$, \dots, $P_n$ are all true is also one which makes $C$ true, or:
\item there is \emph{no} possible world in which $P_1$, \dots, $P_n$ are all true and $C$ is false.
\eit
\eit

}

\subsec{Types of Validity}{

\setlength{\fitchprfwidth}{1.2in}

\begin{tabular}{l@{\qquad}l}
\fitchprf{\pline{Cube(a) \lor Tet(a)}\\
\pline{\lnot Tet(a)}}{
\pline{Cube(a)}}
&
\fitchprf{\pline{\lnot Larger(a, b)}\\
\pline{\lnot Larger(b, a)}}{
\pline{SameSize(a, b)}}
\end{tabular}
}

\subsec{Logical Validity}{

\bit
\item Second argument is valid since there is no world in which premises
are true and conclusion is false
\item But this is (in part) a function of the meanings of the predicate symbols involved
\item Could make a world in which the premises are true and conclusion is false if we changed meanings of predicate symbols
\item (Or: A different argument, where we only change the predicate symbols
in the old argument, is invalid)
\eit
}

\subsec{Logic Validity}{

\setlength{\fitchprfwidth}{1.2in}
\begin{tabular}{ll}
\fitchprf{\pline{\lnot Larger(a, b)}\\
\pline{\lnot Larger(b, a)}}{
\pline{SameSize(a, b)}}
&
\fitchprf{\pline{\lnot LeftOf(a, b)}\\
\pline{\lnot LeftOf(b, a)}}{
\pline{SameSize(a, b)}}\\[2ex]
valid & invalid
\end{tabular}
}

\subsec{Tautological Validity}{

\bit
\item First argument also valid since there is no world in which premises
are true and conclusion is false
\item This is \emph{only} a function of the meanings of the logical connectives
\item If we change the predicate symbols
in the old argument, it is still valid
\eit
}

\subsec{Tautological Validity}{

\setlength{\fitchprfwidth}{1.3in}

\begin{tabular}{lll}
\fitchprf{\pline{Cube(a) \lor Tet(a)}\\
\pline{\lnot Tet(a)}}{
\pline{Cube(a)}}
&
\fitchprf{\pline{Small(a) \lor Dodec(a)}\\
\pline{\lnot Dodec(a)}}{
\pline{Small(a)}}
&
\fitchprf{\pline{P \lor Q}\\
\pline{\lnot Q}}{
\pline{P}}
\end{tabular}

}

\subsec{Definitions}{

\bit
\item Argument is \emph{logically valid} if 
there is no \emph{possible world} in which the premises are all true and conclusion is false.
\item Argument is \emph{tautologically valid} if there is no
  \emph{truth-value assignment} in which the premises are all true and
  conclusion is false.\\[2ex] (truth-value assignment = assignment of \T,
  \F{} to atomic sentences in the argument)
 \eit
}

\subsec{TV Assignments and Possible Worlds}{

\bit
\item To every possible world corresponds a TV assignment: 
\bit
\item Atomic sentence assigned $\displaystyle\left\{\begin{array}{c}\T \\ \F\end{array}\right\}$ if it is $\displaystyle\left\{\begin{array}{c}\text{true} \\ \text{false}\end{array}\right\}$ in the possible world
\item Not every TV assignment so corresponds to a possible world
\item E.g., a TV assignment may assign \F{} to $\sf a = a$, but no possible world makes $\sf a = a$ false!
\eit
\eit
}

\subsec{Types of Validity}{

\bit
\item If there is a possible world in which premises are true and conclusion false, then there is also a truth-value assignment in which the premises are true and the conclusion false
\item Consequently, if an argument is logically \emph{in}valid, it is also tautologically \emph{in}valid
\item This means that if an argument is tautologically valid, it is also logically valid.
\item But not every logically valid argument is tautologically valid (see previous examples).
\eit

}




\section{Tautologies and Equivalence}

\subsec{Sentences Which are Always True}{

\[
\sf Cube(a) \lor \lnot Cube(a)\]\[
\sf Smaller(a, b) \lor Larger(a, b) \lor SameSize(a, b)
\]

}

\subsec{Tautologies, Logic Truths, TW Necessities}{

\bit
\item A sentence is a \emph{tautology} if it is true in every \emph{truth-value assignment}
\item A sentence is a \emph{logical truth} if it is true in every \emph{logically possible world}
\eit
}

\subsec{Tautological Equivalence}{

\bit
\item Two sentences are \emph{tautologically equivalent} if there is no \emph{truth-value assignment} where their truth-values differ.
\item Two sentences are \emph{logically equivalent} if there is no \emph{possible world} where their truth-values differ.
\bit
\item $\sf Larger(a, b)$ and $\sf Smaller(b, a)$ are logically equivalent, but not tautologically equivalent
\item $\sf\lnot(Cube(a) \lor Cube(b))$ and $\sf\lnot Cube(a) \land \lnot Cube(b)$ are tautologically equivalent
\eit
\eit

}

\subsec{Important Equivalences}{
\[\begin{array}{rcl}
\sf P \land Q & \IFF &\sf Q \land \sf P \\
\sf P \lor Q & \IFF &\sf Q \lor P\\
&& \text{(Commutativity)} \\
\sf P \land (Q \land R) & \IFF &\sf (\sf P \land Q) \land R \\
\sf P \lor (Q \lor R) & \IFF &\sf (\sf P \lor Q) \lor R \\
&&\text{(Associativity)} \\
\sf P \land (Q \lor R) & \IFF &\sf (\sf P \land Q) \lor (\sf P \land R) \\
\sf P \lor (Q \land R) & \IFF &\sf (\sf P \lor Q) \land (\sf P \lor R) \\
&& \text{(Distributivity)} \\
\neg (\sf P \land Q) & \IFF &\sf (\neg \sf P \lor \neg Q) \\
\neg (\sf P \lor Q) & \IFF &\sf (\neg \sf P \land \neg Q) \\
&& \text{(De Morgan's Laws)}\\
\neg\neg \sf P & \IFF & \sf P\\
&&\text{(Double Negation)}
\end{array}\]
}

\subsec{Truth Tables}{

\bit
\item List all possible TV assignments to the atomic sentences in an argument, sentence, or pair of sentences
\item For each one, compute truth values of entire sentences
\item Use it to test for tautological validity, being a tautology, or
  tautological equivalence 
\bit 
\item If one TV assignment makes all premises true but conclusion false: not tautologically valid
\item If one TV assignment makes sentence false: not tautology
\item If one TV assignment results in different truth values: not
  tautologically equivalent \eit\eit

}

\section{The Limits of Tarski's World}

\subsec{Limitations of Tarski's World}{

\bit
\item Only $8 \times 8$ squares
\item Only three shapes, three sizes
\item Nothing can adjoin a large block
\item No more than 12 blocks on the grid
\eit

}

\subsec{Sentences Which are Always True}{

\[
\sf Cube(a) \lor \lnot Cube(a)\]\[
\sf Smaller(a, b) \lor Larger(a, b) \lor SameSize(a, b)
\]
\[
\sf Cube(a) \lor Tet(a) \lor Dodec(a)
\]

}

\subsec{Tautologies, Logic Truths, TW Necessities}{

\bit
\item A sentence is a \emph{tautology} if it is true in every \emph{truth-value assignment}
\item A sentence is a \emph{logical truth} if it is true in every \emph{logically possible world}
\item A sentence is a \emph{TW necessity} if it is true in every \emph{TW world}
\eit

}

\subsec{TW Validity, Possible Worlds, TV Assignments}{

\bit
\item Every TW-world is also a possible world
\item Every possible world generates a truth-value assignment which makes the same sentences true/false.
\item Not TW-necessity $\Rightarrow$ Not logical necessity $\Rightarrow$ not tautology
\item tautology $\Rightarrow$ logical necessity $\Rightarrow$ TW-necessity
\eit
}


\subsec{Types of Validity}{

\setlength{\fitchprfwidth}{1.2in}

\begin{tabular}{lll}
\fitchprf{\pline{Cube(a) \lor Tet(a)}\\
\pline{\lnot Tet(a)}}{
\pline{Cube(a)}}
&
\fitchprf{\pline{\lnot Larger(a, b)}\\
\pline{\lnot Larger(b, a)}}{
\pline{SameSize(a, b)}}
&
\fitchprf{\pline{\lnot Cube(a)}\\
\pline{\lnot Tet(a)}}{
\pline{Dodec(a)}}
\end{tabular}

}

\subsec{TW Validity}{

\bit
\item No TW-world in which premises of last argument
are true and conclusion is false
\item But this is (in part) a function of limitation of Tarski's World
\item Could make a world in which the premises are true and conclusion is false if we changed Tarski's World (e.g., add shapes, sizes)
\eit

}

\subsec{TW Validity}{


\fitchprf{\pline{\lnot Cube(a)}\\
\pline{\lnot Tet(a)}}{
\pline{Dodec(a)}}

\vspace{1ex}
Premises true and conclusion false if a were a sphere.

}

\subsec{TW Validity, Possible Worlds, TV Assignments}{
\bit
\item A TW-counterexample is also a possible world counterexample
\item A possible world counterexample is also TV assignment counterexample
\item TW \emph{in}valid $\Rightarrow$ logically \emph{in}valid $\Rightarrow$ tautologically \emph{in}valid
\item tautologically valid $\Rightarrow$ logically valid $\Rightarrow$ tautologically valid
\eit
}

\subsec{Overview}{

\small

\begin{tabular}{l|lll}
Concept & applies to & definition & based on \\\hline\hline
\begin{tabular}{@{}l@{}}
tautology $\Downarrow$ \\
logical truth $\Downarrow$ \\
TW-necessity
\end{tabular} & sentences & never false &
\begin{tabular}{@{}l@{}}
TV-assignment \\
possible world\\
TW-world
\end{tabular}\\
\hline
\begin{tabular}{@{}l@{}}
tautologically valid $\Downarrow$ \\
logically valid $\Downarrow$ \\
TW-valid
\end{tabular} & arguments & 
\begin{tabular}{@{}l@{}}
never true \\ premises and \\ false conclusion
\end{tabular} & 
\begin{tabular}{@{}l@{}}
TV-assignment \\
possible world\\
TW-world
\end{tabular}\\
\hline
\begin{tabular}{@{}l@{}}
tautologically \\ \quad equivalent $\Downarrow$ \\
logically \\ \quad equivalent
\end{tabular} & 2 sentences & 
\begin{tabular}{@{}l@{}}
never different \\ truth values
\end{tabular} & 
\begin{tabular}{@{}l@{}}
TV-assignment \\[2ex]
possible world
\end{tabular}
\end{tabular}

}


\end{document}
