\documentclass[bigger]{beamer}

\usetheme{Frankfurt}
\useoutertheme[subsection=false]{miniframes}
\setbeamercolor{structure}{parent=palette quaternary,fg=uofcred}
\setbeamercolor{section in toc}{fg=black,bg=white}
\setbeamercolor{alerted text}{fg=uofcred}
\setbeamercolor*{palette quaternary}{fg=white,bg=black}
\setbeamercolor*{palette secondary}{fg=black,bg=uofcyellow}
\setbeamercolor*{palette tertiary}{bg=uofcred,fg=white}
\setbeamercolor*{palette primary}{fg=white,bg=black}
\setbeamercolor*{sidebar}{fg=uofcred,bg=black!15!white}
\setbeamercolor*{palette sidebar primary}{fg=uofcred!10!black}
\setbeamercolor*{palette sidebar secondary}{fg=white}
\setbeamercolor*{palette sidebar tertiary}{fg=uofcred!50!black}
\setbeamercolor*{palette sidebar quaternary}{fg=black!10!white}

\setbeamercolor{titlelike}{parent=palette primary}
\setbeamercolor{frametitle}{fg=white,bg=black}
\setbeamercolor{frametitle right}{bg=gray!60!white}
\let\emph\alert
\beamertemplatenavigationsymbolsvertical
\def\bits{\begin{itemize}[<+->]}
\def\bens{\begin{enumerate}[<+->]}
\def\items#1{\item<#1>}
\def\uncovers#1{\uncover<#1>}
\def\pauses{\pause}

 % change to header-handout for handouts

% ====================
\title[Lecture 10]{Logic I F13 Lecture 10}
\date{October 10, 2013}
% ====================

% lecture header include

\usepackage{lplfitch,amsmath}
\usepackage{qtree,hyperref}
\usepackage{pgf,amssymb}

\author{Richard Zach}
\institute{Department of Philosophy\\
University of Calgary\\
  \href{http://ucalgary.ca/rzach/279}{ucalgary.ca/rzach/279}}

\definecolor{uofcred}{RGB}{227,39,38}
\definecolor{uofcyellow}{RGB}{255,210,0}

\DeclareSymbolFont{symbolsC}{U}{txsyc}{m}{n}
\DeclareMathSymbol{\strictif}{\mathrel}{symbolsC}{74}
\DeclareMathSymbol{\boxright}{\mathrel}{symbolsC}{128}

\let\IFF\Leftrightarrow
\let\iff\leftrightarrow
\let\impl\to

\def\T{{\color{green}\begin{colormixin}{25!black}\text{T}\end{colormixin}}}
\def\F{{\color{red}\begin{colormixin}{25!black}\text{F}\end{colormixin}}}

\long\def\subsec#1#2{\subsection{#1}\frame{\frametitle{#1} #2}}
\def\bit{\begin{itemize}[<1->]}
\def\eit{\end{itemize}}
\def\ben{\begin{enumerate}[<1->]}
\def\een{\end{enumerate}}

\makeatletter\let\@makefnmark\noindent\makeatother

%\setbeamercolor{footnote}{fg=black!70}

\def\foot#1{\footnotetext{\color{black!70}#1}}
\def\deemph#1{{\color{black!70}#1}}
\let\phi\varphi

\setbeamertemplate{theorems}[numbered]
%\useinnertheme{circles}
\setbeamertemplate{itemize subitems}[triangle]

\renewcommand{\beamertemplatetransparentcovereddynamic}{
  \beamersetuncovermixins
  {\opaqueness<1>{50}\opaqueness<2>{30}\opaqueness<3>{15}\opaqueness<4->{5}}%
  {\opaqueness<1>{50}\opaqueness<2>{30}\opaqueness<3>{15}\opaqueness<4->{5}}}
                                                                                
%\beamertemplatetransparentcovereddynamic

\defbeamertemplate*{footline}{my theme}
{%
  \leavevmode%
  \hbox{\begin{beamercolorbox}[wd=.5\paperwidth,ht=2.5ex,dp=1.125ex,leftskip=.3cm,rightskip=.3cm]{author in head/foot}%
    \insertframenumber/\inserttotalframenumber \hfil \usebeamerfont{author in head/foot}\insertshortauthor
  \end{beamercolorbox}%
  \begin{beamercolorbox}[wd=.5\paperwidth,ht=2.5ex,dp=1.125ex,leftskip=.3cm,rightskip=.3cm plus1fil]{title in head/foot}%
    \usebeamerfont{title in head/foot}Logic I F13---\insertshorttitle---\insertdate
  \end{beamercolorbox}}%
  \vskip0pt%
}

\begin{document}
\setlength{\fitchargwidth}{7em}
\setlength{\fitchprfwidth}{7em}

\frame{\frametitle{\insertshorttitle\ (\insertdate)}
\tableofcontents[hidesubsections]
}










\setlength{\fitchprfwidth}{5em}


\section{Conditionals}

\subsec{Indicative Conditionals}{

\bit
\item Conditionals = ``if---then\dots'' sentences
\item \emph{Indicative} conditionals: in indicative mood
\item ``If P, then Q'' or ``Q, if P'' with P, Q indicative sentences
\item P: \emph{antecedent}; Q: \emph{consequent}
\eit
}

\subsec{Truth Conditions of Conditionals}{

\[\text{If\ } \underbrace{\text{X is drinking alcohol}}_{P}, 
\text{then\ }\underbrace{\text{X is over 18}}_{Q}\]

\bit
\item ``If P, then Q'' can only be \emph{false} if:
\bit
\item P is \emph{true}: we check age if X is drinking beer (P~true), not if
 drinking pop; \emph{and}
\item Q is \emph{false}: we check drink if X under age (Q false),\\ not 
if over 18
\eit
\item ``If P, then Q'' is true if:
\bit
\item P is \emph{false:} we don't check people drinking pop; \emph{or}
\item Q is \emph{true:} we don't card if X is over 18;
\item (or both)
\eit
\eit }



\subsec{The Material Conditional $\to$}{
\[
\begin{array}{cc|c}
P & Q & (P \to Q)\\
\hline
\T & \T & \T\\
\T & \F & \F\\
\F & \T & \T\\
\F & \F & \T
\end{array}
\]
}

\subsec{Other Indicative Conditionals}{

\bit \item The following all get translated by $P \to Q$:
\bit
\item If P, then Q
\item Q if P
\item Q provided P
\item P only if Q
\eit
\eit
}

\subsec{If and Only If}{

\bit
\item Sue drinks beer only if she is over 18
\[
\sf Drinks(sue, beer) \to Over18(sue)
\]
\item Sue drinks beer if she is over 18.
\[
\sf Over18(sue) \to Drinks(sue, beer)
\]
\eit
}

\subsec{Subjunctive Conditionals}{

\bit
\item Subjunctive conditionals = if---then statements in \emph{subjunctive} mood
\item ``If P were true, then Q would be true.''
\item Indicative conditional is \emph{truth-functional}: truth value of ``If P, then Q'' depends
  \emph{only on truth values} of P and Q.
\item Not truth functional, e.g., consider:
\ben
\item If the world were just, no evil deed would go unpunished.\\
$P_1$ = the world is just\\
$Q_1$ = no evil deed goes unpunished
\item If the world were flat, no evil deed would go unpunished.\\
$P_2$ = the world is flat\\
$Q_2$ = no evil deed goes unpunished
\een
\item $P_1$, $Q_1$ both false; $P_2$, $Q_2$ both false, but
\item (1) is true, but (2) is false
\eit
}

\subsec{The Biconditional}{

\bit
\item ``P if and only if Q'' short for:\\
``P if Q and P only if Q'' and\\
``P just in case Q''
\item Translated as: $P \iff Q$
\item Truth table:
\[
\begin{array}{cc|c}
P & Q & (P \iff Q)\\
\hline
\T & \T & \T\\
\T & \F & \F\\
\F & \T & \F\\
\F & \F & \T
\end{array}
\]
\eit

}

\subsec{Unless}{

Which of these people do you have to check (age or drink) to ensure that:
\begin{quote}
People are drinking pop unless they are over 18.
\end{quote}

\begin{tabular}{cccc}
\begin{beamerboxesrounded}[width=4em]{}
\vskip 2ex
\Large 22 years\\
\end{beamerboxesrounded} &
\begin{beamerboxesrounded}[width=4em]{}
\vskip 2ex
\Large 16 years\\
\end{beamerboxesrounded} &
\begin{beamerboxesrounded}[width=4em]{}
\vskip 2ex
\Large drinks pop\\
\end{beamerboxesrounded} &
\begin{beamerboxesrounded}[width=4em]{}
\vskip 2ex
\Large drinks beer\\
\end{beamerboxesrounded} \\
(1) & (2) & (3) & (4)
\end{tabular}

}


\subsec{Unless}{

\[\underbrace{\text{X is drinking pop}}_{P}, 
\text{unless\ }\underbrace{\text{X is over 18}}_{Q}\]

\bit
\item ``P unless Q'' can only be \emph{false} if:
\bit
\item P is \emph{false}\\
(we check age if person is drinking beer), \emph{and}
\item Q is \emph{false}\\
(we check drink if person not at least 18)
\eit
\item ``P unless Q'' is true (test ok) if:
\bit
\item P is false, and Q is true;
\item Q is false, and P is true;
\item P is true and Q is true
\eit
\item ``P unless Q'' is equivalent to
\bit
\item ``P if not Q'' ($\sf \lnot Q \to P$)
\item ``P or Q'' ($\sf P \lor Q$)
\eit
\eit

}

\section{Proof Methods for Conditionals}


\subsec{Modus Ponens}{
\setlength{\fitchprfwidth}{1.2in}

\fitchprf{\tline{If P, then Q}\\ \tline{P}}{\tline{Q}}
}

\subsec{Conditional Proof}{

If on the assumption that P is true, you can prove Q, then you have
given a proof of the conditional ``If P, then Q''


}

\subsec{Example}{

\setlength{\fitchprfwidth}{1.2in}

\begin{columns}
\begin{column}{3.5cm}
\fitchprf{
\nline{Cube(a) \lor Small(a)}}{\nline{\lnot Cube(a) \to Small(a)}}
\end{column}
\begin{column}{7cm}
\bens
\item Suppose a is not a cube.  
\item We are told (premise) that a is a cube or a is small.
\item By disjunctive syllogism, a is small.
\item So by conditional proof, if a is not a cube, it is small. QED
\een
\end{column}
\end{columns}
}

\subsec{Example}{

\setlength{\fitchprfwidth}{1.2in}

\begin{columns}
\begin{column}{3.5cm}
\fitchprf{
\nline{\lnot Cube(a) \to Small(a)}}{\nline{Cube(a) \lor Small(a)}}
\end{column}
\begin{column}{7cm}
\bens
\item We proceed indirectly: Assume $Cube(a) \lor Small(a)$ were false.
\item Then a is not a cube, and a is also not small.
\item In particular, a is not a cube.
\item We are told (premise) that if a is not a cube, it is small.
\item By modus ponens, a is small.
\item Contradiction, since from (2) we have that a isn't small.
\item So by indirect proof, a is either a cube or it is small. QED
\een
\end{column}
\end{columns}
}

\section{Formal Proofs with $\to$}

\subsec{Formal Rules for $\to$}{

\noindent\setlength{\fitchprfwidth}{.9in}
\begin{tabular}{@{}l@{}l@{}}
\fitchctx{
\subproof{\nline[$k.$]{P}}{\dots\\ \nline[$l.$]{Q}}
\fpline[\quad]{P \to Q}[\lifi{$k$--$l$}]}
&
\fitchctx{
\nline[$n.$]{P \to Q}\\
\nline[$m.$]{P}\\
\fpline[\quad]{Q}[\life{n}{m}]}
\end{tabular}
}

\subsec{Formal Rules for $\iff$}{

\noindent\setlength{\fitchprfwidth}{.9in}

\fitchctx{
\subproof{\nline[$k.$]{P}}{\dots\\ \nline[$l.$]{Q}}\\
\subproof{\nline[$m.$]{Q}}{\dots\\ \nline[$n.$]{P}}\\
\fpline[\quad]{P \iff Q}[\liffi{$k$--$l$}{$m$--$n$}]}}

\subsec{Formal Rules for $\iff$}{

\noindent\setlength{\fitchprfwidth}{.9in}

\noindent\fitchctx{
\nline[$n.$]{P \iff Q}\\
\nline[$m.$]{P}\\
\fpline[\quad]{Q}[\liffe{n}{m}]}\\[2ex]
\fitchctx{
\nline[$n.$]{P \iff Q}\\
\nline[$m.$]{Q}\\
\fpline[\quad]{P}[\liffe{n}{m}]}

}

\subsec{Strategies}{

\bit
\item If your goal sentence is of the form $P \to Q$:
\bit
\item Start subproof
\item Put P in assumption line
\item Put Q as last line
\item Justify goal sentence with $\to$Intro
\item Try to prove Q inside subproof
\item This strategy usually works, so use it sooner rather than later
\eit
\item If you have $P \to Q$ available:
\bit
\item You can use it to get Q via $\to$Elim
\item For that, you need to prove P first
\item Wait with applying this strategy until you need Q
\eit
\eit

}

\subsec{Strategies}{

\bit
\item If your goal sentence is of the form $P \iff Q$:
\bit
\item Start 2 subproofs
\item In one, put P in assumption line and Q as last line
\item In one, put Q in assumption line and P as last line
\item Justify goal sentence with $\iff$Intro
\item Try to prove Q/P inside subproofs
\item This strategy usually works, so use it sooner rather than later
\eit
\item If you have $P \iff Q$ available:
\bit
\item You can use it to get P or Q via $\to$Elim
\item For that, you need to prove Q or P, resp., first
\item Wait with applying this strategy until you need Q or P
\eit
\eit

}

\subsec{Example}{

\fitchprf{\tline{\ }}{\nline{((A \lor B) \to C) \iff ((A \to C) \land (B \to C))}}

}
\subsec{Example}{

\fitchprf{\tline{\ }}{\nline{(\lnot A \iff B) \to (A \iff \lnot B)}}

}

\subsec{Example}{

\fitchprf{\tline{\ }}{\nline{(A\to B) \lor (B \to A)}}

}


\end{document}




