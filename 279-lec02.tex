\documentclass[bigger]{beamer}
\usepackage{tikz}
\usetikzlibrary{arrows}

\usetheme{Frankfurt}
\useoutertheme[subsection=false]{miniframes}
\setbeamercolor{structure}{parent=palette quaternary,fg=uofcred}
\setbeamercolor{section in toc}{fg=black,bg=white}
\setbeamercolor{alerted text}{fg=uofcred}
\setbeamercolor*{palette quaternary}{fg=white,bg=black}
\setbeamercolor*{palette secondary}{fg=black,bg=uofcyellow}
\setbeamercolor*{palette tertiary}{bg=uofcred,fg=white}
\setbeamercolor*{palette primary}{fg=white,bg=black}
\setbeamercolor*{sidebar}{fg=uofcred,bg=black!15!white}
\setbeamercolor*{palette sidebar primary}{fg=uofcred!10!black}
\setbeamercolor*{palette sidebar secondary}{fg=white}
\setbeamercolor*{palette sidebar tertiary}{fg=uofcred!50!black}
\setbeamercolor*{palette sidebar quaternary}{fg=black!10!white}

\setbeamercolor{titlelike}{parent=palette primary}
\setbeamercolor{frametitle}{fg=white,bg=black}
\setbeamercolor{frametitle right}{bg=gray!60!white}
\let\emph\alert
\beamertemplatenavigationsymbolsvertical
\def\bits{\begin{itemize}[<+->]}
\def\bens{\begin{enumerate}[<+->]}
\def\items#1{\item<#1>}
\def\uncovers#1{\uncover<#1>}
\def\pauses{\pause}

 % change to header-handout for handouts

% ====================
\title[Lecture 2]{Logic I F13 Lecture 2}
\date{September 12, 2013}

% ====================

% lecture header include

\usepackage{lplfitch,amsmath}
\usepackage{qtree,hyperref}
\usepackage{pgf,amssymb}

\author{Richard Zach}
\institute{Department of Philosophy\\
University of Calgary\\
  \href{http://ucalgary.ca/rzach/279}{ucalgary.ca/rzach/279}}

\definecolor{uofcred}{RGB}{227,39,38}
\definecolor{uofcyellow}{RGB}{255,210,0}

\DeclareSymbolFont{symbolsC}{U}{txsyc}{m}{n}
\DeclareMathSymbol{\strictif}{\mathrel}{symbolsC}{74}
\DeclareMathSymbol{\boxright}{\mathrel}{symbolsC}{128}

\let\IFF\Leftrightarrow
\let\iff\leftrightarrow
\let\impl\to

\def\T{{\color{green}\begin{colormixin}{25!black}\text{T}\end{colormixin}}}
\def\F{{\color{red}\begin{colormixin}{25!black}\text{F}\end{colormixin}}}

\long\def\subsec#1#2{\subsection{#1}\frame{\frametitle{#1} #2}}
\def\bit{\begin{itemize}[<1->]}
\def\eit{\end{itemize}}
\def\ben{\begin{enumerate}[<1->]}
\def\een{\end{enumerate}}

\makeatletter\let\@makefnmark\noindent\makeatother

%\setbeamercolor{footnote}{fg=black!70}

\def\foot#1{\footnotetext{\color{black!70}#1}}
\def\deemph#1{{\color{black!70}#1}}
\let\phi\varphi

\setbeamertemplate{theorems}[numbered]
%\useinnertheme{circles}
\setbeamertemplate{itemize subitems}[triangle]

\renewcommand{\beamertemplatetransparentcovereddynamic}{
  \beamersetuncovermixins
  {\opaqueness<1>{50}\opaqueness<2>{30}\opaqueness<3>{15}\opaqueness<4->{5}}%
  {\opaqueness<1>{50}\opaqueness<2>{30}\opaqueness<3>{15}\opaqueness<4->{5}}}
                                                                                
%\beamertemplatetransparentcovereddynamic

\defbeamertemplate*{footline}{my theme}
{%
  \leavevmode%
  \hbox{\begin{beamercolorbox}[wd=.5\paperwidth,ht=2.5ex,dp=1.125ex,leftskip=.3cm,rightskip=.3cm]{author in head/foot}%
    \insertframenumber/\inserttotalframenumber \hfil \usebeamerfont{author in head/foot}\insertshortauthor
  \end{beamercolorbox}%
  \begin{beamercolorbox}[wd=.5\paperwidth,ht=2.5ex,dp=1.125ex,leftskip=.3cm,rightskip=.3cm plus1fil]{title in head/foot}%
    \usebeamerfont{title in head/foot}Logic I F13---\insertshorttitle---\insertdate
  \end{beamercolorbox}}%
  \vskip0pt%
}

\begin{document}
\setlength{\fitchargwidth}{7em}
\setlength{\fitchprfwidth}{7em}

\frame{\frametitle{\insertshorttitle\ (\insertdate)}
\tableofcontents[hidesubsections]
}










\section{First Order Languages}

\subsec{Syntax of FOL}{

\bit
\item Individual constants

\bit 
\item Start lowercase
\item $a$, $b$, $c$, $\sf max$, $\sf claire$, \dots
\eit
\item Predicate symbols
\bit
\item Start uppercase
\item Have \emph{arity} (= number of argument places)
\item $P$, $Q$, $R$, \dots, $\sf Cube$, $\sf Larger$, $\sf Between$,\dots
\eit
\eit
}

\subsec{Atomic Sentences}{

\bit
\item Predicate symbol $P$ or arity $n$ \dots
\item \dots followed by $n$ constant symbols
\item enclosed in parentheses, separated by commas
\item e.g., $\sf Cube(a)$, $\sf Larger(c, d)$, $\sf SameSize(a, a)$,
  $\sf Between(a, b, c)$.  
\item $\sf Large(Cube(a))$ is \emph{not well formed} (ungrammatical)
\eit
}

\subsec{Meaning of Individual Constants}{

\bit 
\item Individual constant in FOL = names in English
\item Individual constants pick out exactly one object
\bit
\item No ``empty'' names, e.g., Santa Claus 
\item No individual constant names more than one object
\item But: an object can be named by more than one constant
\eit
\eit
}

\subsec{Meanings of Predicate Symbols}{

\bit
\item Predicate symols in FOL = verbs, adjectives, common nouns in English
\item Predicate symbols mean \emph{determinate} properties (relations
  if arity $\ge 2$)
\item Determinate property: determinate matter of fact for every
  object whether it has the property or not
\item Order of argument places is important
\eit

}

\subsec{Meanings of Atomic Sentences}{

\bit
\item Atomic sentences = simple subject--predicate sentences of English
\item Atomic sentences describe a simple fact about the way the world could be
\item Atomic sentences have truth values: true, false\\
(provided all constant symbols name something)
\item $P(a)$ is true iff object named by $a$ has the determinate
  property corresponding to~$P$; false if it doesn't
 \eit
}



\section{Possible Worlds and First-order Structures}

\subsec{Possible Worlds}{

\bit
\item A \emph{possible world} is a way (part of) the world described by a FOL 
could be
\bit
\item There is a collection (non-empty) of objects of a certain type:
  \emph{domain}
\item Individual constants each name one of these objects
\item Each of the objects have or lack the properties we interpret the
  1-place predicate symbols by
\item Each pair of objects either stands or fails to stand in the
  relations we interpret the 2-place objects by, \dots \eit
\item Every atomic sentence is true or false in a possible world
\eit
}

\subsec{Possible Worlds for the Blocks Language}{

\bit
\item The domain consists of solid blocks arranged spatially
\item Some but not necessarily all of the blocks are named by
  individual constants
\item Every object either is or isn't a cube, a tetrahedron, a
  dodecahedron
\item Of any two objects, 
\bit
\item the first is larger than the second or it isn't, 
\item it's left of the second or isn't, \dots
\eit
\eit

}

\subsec{First-order Structures}{

\bit
\item Specific characteristics of objects in domain not relevant
for truth or falsity of atomic sentences
\item More abstract: \emph{first-order structures}
\bit
\item Domain can be \emph{any} non-empty collection of objects
\item Some of them named by individual constants
\item Predicate symbols have \emph{extensions} assigned to them
\bit
\item Extensions of 1-place predicate symbols: collection of objects
  of the domain (may be empty),
\item Extension of 2-place predicate symbols: collection of
  \emph{pairs} of objects, \dots 
\eit \eit \eit }

\subsec{Possible Worlds and FO Structures}{

\bit
\item Truth in a FO structure:
\bit
\item $P(a)$ is true iff the object named by $a$ is in the extension of $P$
\item $Q(a, b)$ is true iff the pair of objects named by $a$, $b$ is
  in the extension of $Q$ \eit
\item Every possible world generates a first-order structure
\bit
\item Domain and referents of individual constants are the same
\item Extension of $P$: the collection of all (pairs of) objects which
  have the property (stand in the relation) assigned to $P$ 
\eit 
\eit

}

\subsec{TW World to FO Structure}{

\hfill\pgfimage[height=2cm]{world.png}

\vspace{-1cm}
\bit
\item Domain: $1$, $2$, $3$
\item Constants: $a \mapsto 2$, $b \mapsto 3$
\item Extensions:\\[.5ex]

\begin{tabular}{@{}llll}
Small: $1$ & Medium: $2$ & Large: $3$ \\
Cube: $1$  & Tet: $3$ & Dodec: $2$\\
\multicolumn{2}{@{}l}{Smaller: $\langle 1, 2\rangle$, $\langle 1, 3\rangle$, 
$\langle 2, 3\rangle$} & 
\multicolumn{2}{l}{Larger: $\langle 2, 1\rangle$, $\langle 3, 1\rangle$, 
$\langle 3, 2\rangle$}\\
\multicolumn{2}{@{}l}{LeftOf: $\langle 1, 2\rangle$, $\langle 1, 3\rangle$, $\langle 2, 3\rangle$} &
\multicolumn{2}{l}{Adjoins: $\langle 1, 2\rangle$, $\langle 2, 1\rangle$}\\
\multicolumn{3}{@{}l}{SameRow: $\langle 1, 1\rangle$, $\langle 1, 2\rangle$, $\langle 2, 1\rangle$, $\langle 2, 2\rangle$, $\langle 3, 3\rangle$} & Between: $\emptyset$
\end{tabular}
\eit
}

\subsec{Diagrams for First-order Structures}{

\bit
\item Represent domain as numbered dots
\item Individual constants label on a dot
\item Extensions of predicate symbols as areas containing dots
\item Extension of 2-place predicate symbols: arrows
\eit

}

\subsec{Diagrams for First-order Structures}{

Representation of extensions of Cube, Large, SameRow, and Smaller

\begin{center}
\tikz {
	%Set nodes and draw dots
	\fill (-1.5,0) circle (2pt) node[label=below:1] (1) {};
	\fill (1.5,0) circle (2pt) node[label=above:a, label=below:2] (2) {};
	\fill (0, 1.5) circle (2pt) node[label=above:b, label=below:3] (3) {};
	%Draw circles for Cube, Dodec, and Tet
	\node [circle, draw, thick, label=180:Cube] at (-1.5,0) [minimum size=2cm] {};
	\node [circle, draw, thick, label=0:Large] at (0,1.7) [minimum size=2cm] {};
	%SameShape arrows
	\draw[-latex, thick] (1) to[bend left] (2) ;
	\draw[-latex, thick] (2) to[bend left] (1) ;
	\path[-latex, thick, every loop/.style={min distance=10mm, looseness=10}] (1) edge [in=-180, out=-120, loop]  ();
	\path[-latex, thick, every loop/.style={min distance=10mm, looseness=10}] (2) edge [in=-30, out=30, loop]  ();
	\path[-latex, thick, every loop/.style={min distance=10mm, looseness=10}] (3) edge [in=180, out=120, loop]  ();
	%Larger arrows
	\draw[-latex, thick, dashed] (1) -> (2) ;
	\draw[-latex, thick, dashed] (2) -> (3) ;
	\draw[-latex, thick, dashed] (1) -> (3) ;
        \draw[-latex, thick] (-1.5, -1.5) to (0, -1.5) node[label=0:SameRow] {};
        \draw[-latex, thick, dashed] (-1.5, -2) to (0, -2) node[label=0:Smaller] {};
}
\end{center}

}

\section{Identity and Function Symbols}

\subsec{Identity}{

\bit
\item Special predicate symbol included in every FO language: \[=\] 
\item Arity 2, written between arguments: $a = b$
\item $a = b$ is true iff $a$ and $b$ name the very same object
\item $a = a$ is always true
\eit
}

\subsec{Function Symbols}{

\bit
\item Important way of naming objects indirectly: functions (operations)
\item e.g., ``the father of'',  $\sqrt{\phantom{X}}$, $\sf head(l)$
\item Some FO languages include \emph{function symbols}
\item Lowercase like constants, but have arity like predicates
\item e.g., in blocks language, fm, bm, lm, rm
\eit

}

\subsec{Terms and Atomic Sentences}{

\bit
\item Function symbols and constants combine to form \emph{terms}
\item But, terms can themselves be arguments to function symbols,
  i.e., function symbols can be nested in terms
\item E.g.,
\bit
\item $\sf lm(a)$
\item $\sf fm(lm(a))$
\item $\sf Cube(fm(a))$ is an atomic sentence
\item $\sf fm(Cube(a))$ is \emph{not well formed} (ungrammatical)
\eit
\item Precise definition:
\bit
\item Every constant symbol is a term
\item If $f$ is a function symbol of arity $n$, and $t_1$, \dots, $t_n$ are terms, then $f(t_1, \dots, t_n)$ is a term
\item If $P$ is a predicate symbol of arity $n$, and $t_1$, \dots, $t_n$ are terms, then $P(t_1, \dots, t_n)$ is an atomic sentence
\eit\eit
}

\subsec{The Language of Arithmetic}{

\bit 
\item Predicate symbols: $<$, $=$ 
\item Constant symbols: $0$, $1$
\item Function symbols: $+$, $\times$ (arity 2, infix)
\item Terms: $(((1 + 1) +1) + 1)$, $(1 + ((1 + 1) \times (1 + (1 + 1))))$\\
Note: neither $1 + 1 + 1$ nor $(1 + 1 +1)$ are terms!
\item Atomic sentences:
\bit
\item $0 = 1$, $0 < 1$
\item $(1 + (1 + 1)) = ((1 + 1) + 1)$
\eit
\eit

}


\end{document}
