\documentclass[bigger]{beamer}

\usetheme{Frankfurt}
\useoutertheme[subsection=false]{miniframes}
\setbeamercolor{structure}{parent=palette quaternary,fg=uofcred}
\setbeamercolor{section in toc}{fg=black,bg=white}
\setbeamercolor{alerted text}{fg=uofcred}
\setbeamercolor*{palette quaternary}{fg=white,bg=black}
\setbeamercolor*{palette secondary}{fg=black,bg=uofcyellow}
\setbeamercolor*{palette tertiary}{bg=uofcred,fg=white}
\setbeamercolor*{palette primary}{fg=white,bg=black}
\setbeamercolor*{sidebar}{fg=uofcred,bg=black!15!white}
\setbeamercolor*{palette sidebar primary}{fg=uofcred!10!black}
\setbeamercolor*{palette sidebar secondary}{fg=white}
\setbeamercolor*{palette sidebar tertiary}{fg=uofcred!50!black}
\setbeamercolor*{palette sidebar quaternary}{fg=black!10!white}

\setbeamercolor{titlelike}{parent=palette primary}
\setbeamercolor{frametitle}{fg=white,bg=black}
\setbeamercolor{frametitle right}{bg=gray!60!white}
\let\emph\alert
\beamertemplatenavigationsymbolsvertical
\def\bits{\begin{itemize}[<+->]}
\def\bens{\begin{enumerate}[<+->]}
\def\items#1{\item<#1>}
\def\uncovers#1{\uncover<#1>}
\def\pauses{\pause}

 % change to header-handout for handouts

% ====================
\title[Lecture 11]{Logic I F13 Lecture 11}
\date{October 15, 2013}
% ====================

% lecture header include

\usepackage{lplfitch,amsmath}
\usepackage{qtree,hyperref}
\usepackage{pgf,amssymb}

\author{Richard Zach}
\institute{Department of Philosophy\\
University of Calgary\\
  \href{http://ucalgary.ca/rzach/279}{ucalgary.ca/rzach/279}}

\definecolor{uofcred}{RGB}{227,39,38}
\definecolor{uofcyellow}{RGB}{255,210,0}

\DeclareSymbolFont{symbolsC}{U}{txsyc}{m}{n}
\DeclareMathSymbol{\strictif}{\mathrel}{symbolsC}{74}
\DeclareMathSymbol{\boxright}{\mathrel}{symbolsC}{128}

\let\IFF\Leftrightarrow
\let\iff\leftrightarrow
\let\impl\to

\def\T{{\color{green}\begin{colormixin}{25!black}\text{T}\end{colormixin}}}
\def\F{{\color{red}\begin{colormixin}{25!black}\text{F}\end{colormixin}}}

\long\def\subsec#1#2{\subsection{#1}\frame{\frametitle{#1} #2}}
\def\bit{\begin{itemize}[<1->]}
\def\eit{\end{itemize}}
\def\ben{\begin{enumerate}[<1->]}
\def\een{\end{enumerate}}

\makeatletter\let\@makefnmark\noindent\makeatother

%\setbeamercolor{footnote}{fg=black!70}

\def\foot#1{\footnotetext{\color{black!70}#1}}
\def\deemph#1{{\color{black!70}#1}}
\let\phi\varphi

\setbeamertemplate{theorems}[numbered]
%\useinnertheme{circles}
\setbeamertemplate{itemize subitems}[triangle]

\renewcommand{\beamertemplatetransparentcovereddynamic}{
  \beamersetuncovermixins
  {\opaqueness<1>{50}\opaqueness<2>{30}\opaqueness<3>{15}\opaqueness<4->{5}}%
  {\opaqueness<1>{50}\opaqueness<2>{30}\opaqueness<3>{15}\opaqueness<4->{5}}}
                                                                                
%\beamertemplatetransparentcovereddynamic

\defbeamertemplate*{footline}{my theme}
{%
  \leavevmode%
  \hbox{\begin{beamercolorbox}[wd=.5\paperwidth,ht=2.5ex,dp=1.125ex,leftskip=.3cm,rightskip=.3cm]{author in head/foot}%
    \insertframenumber/\inserttotalframenumber \hfil \usebeamerfont{author in head/foot}\insertshortauthor
  \end{beamercolorbox}%
  \begin{beamercolorbox}[wd=.5\paperwidth,ht=2.5ex,dp=1.125ex,leftskip=.3cm,rightskip=.3cm plus1fil]{title in head/foot}%
    \usebeamerfont{title in head/foot}Logic I F13---\insertshorttitle---\insertdate
  \end{beamercolorbox}}%
  \vskip0pt%
}

\begin{document}
\setlength{\fitchargwidth}{7em}
\setlength{\fitchprfwidth}{7em}

\frame{\frametitle{\insertshorttitle\ (\insertdate)}
\tableofcontents[hidesubsections]
}










\setlength{\fitchprfwidth}{5em}

\section{Quantifiers}

\subsec{Determiners in English}{

\bit
\item Articles: a, the
\item Cardinal numbers: zero, one, two, \dots
\item Quantifiers: all, few, many, several, some, every, each, any, no
\item either, neither, both
\item \dots
\eit

}


\subsec{Quantifiers in FOL}{

\bit
\item $\forall$: \emph{universal} quantifier, ``for all''
\[
\sf \forall x\, Cube(x)
\]
(Everything is a cube)
\item $\exists$: \emph{existential} quantifier, ``exists''
\[
\sf \exists y\, Large(y)
\]
(Something is large)
\eit

}

\subsec{Variables and Terms}{

\bit
\item Variables: $x$, $y$, $z$, $x_1$, $x_2$, \dots
\item Terms:
\bit
\item Constant symbols and variables are terms
\item If $f$ is an $n$-place function symbol, and $t_1$, \dots, $t_n$
are terms, then $f(t_1, \dots, t_n)$ is a term
\eit
\item Eg, in the language of arithmetic:
\[
1 \qquad x \qquad (1 + (x + y))
\]
\eit
}

\subsec{Well-Formed Formulas (wff's)}{

\bit
\item If $P$ is an $n$-place predicate symbol, and $t_1$, \dots, $t_n$
are terms, then $P(t_1, \dots, t_n)$ is an atomic wff
\item Every atomic wffs is a wff
\item If $P$ is a wff, so is $\lnot P$
\item If $P_1$, \dots, $P_n$ are wffs, so are
\[
(P_1 \land {} \dots \land P_n) \text{ and }
(P_1 \lor {} \dots \lor P_n) 
\]
\item If $P$ and $Q$ are wffs, then so are
\[
(P \to Q) \text{ and } (P \iff Q)
\]
\item If $P$ is a wff and $x$ is a variable, then
\[
\forall x\, P \text{ and } \exists x\, P
\]
are wffs
\eit

}

\subsec{Examples}{

\bit
\item Atomic wffs: $\sf Cube(x)$, $\sf Larger(x, x)$, $\sf Larger(x, y)$, $\sf Larger(z, a)$, $\sf a = y$
\item $\sf Large(x) \land Cube(x)$
\item $\sf Large(x) \to (Cube(x) \land Larger(x, b))$ 
\item $\sf \forall x\, Cube(x)$
\item $\sf \exists x\, (Large(x) \land Cube(x))$ 
\item $\sf \exists y\, Cube(x)$
\item $\sf \forall x\, Cube(x) \land \forall x\, Large(x)$
\item $\sf \forall x\, (Cube(x) \land \forall x\, Large(x))$
\item $\sf \forall x\, Cube(x) \land Large(x)$
\eit 

}

\subsec{Scope and Binding}{

\[
\underline{\forall {\color{red}x}(Cube({\color{red}x}) \to (\underline{\exists {\color{blue}x}\, Smaller({\color{blue}x}, y)} \land \underline{\exists {\color{green}y}(Tet({\color{green}y}) \land Larger({\color{green}y}, {\color{red}x}))}))}
\]
\bits
\item The \emph{scope} of a quantifier extends till end of wff it precedes
\item Variables are \emph{bound} by the nearest matching quantifier to the left in whose scope they are
\item Variables not bound are \emph{free}
\item A \emph{sentence} is a wff with no free variables
\eit
}

\section{Semantics of Quantifiers and Variables}

\subsec{Possible Worlds and First-order Structure}{

\bit
\item Collection of objects (\emph{domain}, not empty)
\item Referents for each individual constant (which object it names)
\item Properties of each object (shape, size, position on board)
  \bit\item Extension of each 1-place predicate symbol: collection of
  objects it applies to\eit
\item Relations of each pair of objects (larger, same row,
  etc)\bit\item Extension of each $n$-place predicate symbol: tuple of
  objects standing in the relation
 \eit\eit

}

% Redo this -- motivate by ``\forall x Cube(x)'' should say ``everything is a cube'' -- but what if it's more complicated than just''Cube(x)''?  Less abstract at the beginning!

\subsec{Truth of Sentences}{

\bit
\item Suppose $P(x)$ contains only $x$ free 
\item Suppose $n$ is a new constant symbol not occurring in $P(x)$
\item Let $P(n)$ be $P(x)$ with all free occurrences of $x$ replaced by $n$
\item $\forall x\, P(x)$ is true in a world $W$ iff for every $\alpha$ among the objects in $W$, $P(n)$ is true in the world $W_\alpha$ which is just like $W$ except $n$ names $\alpha$
\item $\exists x\, P(x)$ is true in a world $W$ iff for at least one $\alpha$ among the objects in $W$, $P(n)$ is true in the world $W_\alpha$ which is just like $W$ except $n$ names $\alpha$
\eit

}


\subsec{Satisfaction}{

\bit 
\item Define relation of \emph{satisfaction} between an object $\alpha$,
a wff with one free variable $P(x)$, and a world~$W$\\
``$\alpha$ satisfies $P(x)$''
\item  \emph{$\alpha$ satisfies $P(x)$ in~$W$} iff $P(n)$ is true in the world~$W'$ which is just like $W$ except $n$ names $\alpha$
\item $P(x)$ defines a property in the world~$W$: the collection of all the objects $\alpha$ that satisfy $P(x)$
\item{} [More general: \emph{$\langle\alpha_1, \dots, \alpha_k\rangle$
  satisfies $P(x_1, \dots, x_k)$ in $W$} iff $P(n_1, \dots, n_k)$ is
  true in $W'$ which is like $W$ except $n_i$ names $\alpha_i$ ($i =
  1, \dots, k$)]
\item $P(x_1, \dots, x_k)$ defines a $k$-place relation
\eit

}



\subsec{Satisfaction and Expressing Properties and Relations}{

\bit
\item $\sf Large(x) \land Cube(x)$ \dots ``$x$ is a large cube''
\item $\sf LeftOf(x, a) \lor RightOf(x, a)$ \dots ``$x$ is left of or right of a''; ``$x$ is to the side of a''
\item More than one way to express a property:
\bit
\item $\sf LeftOf(a, x) \lor RightOf(a, x)$
\item $\sf LeftOf(x, a) \lor LeftOf(a, x)$
\item $\sf \lnot SameCol(x, a)$
\eit
\eit
}

\section{Combining Quantifiers with WFFs}

\subsec{Quantifiers and Properties: $\forall$}{

\bit
\item The sentence
\[\sf
\forall x\, P(x)
\]
says (is true iff) 
\bit
\item $P(n)$ is true for \emph{whatever} object is named by~$n$
\item every object in the domain (= world) satisfies $P(x)$
\item every object in the domain has the property expressed by $P(x)$,
\eit
\eit
}

\subsec{Quantifiers and Properties: $\exists$}{

\bit
\item The sentence
\[\sf
\exists x\, P(x)
\]
says (is true iff) 
\bit
\item $P(n)$ is true for \emph{at least one} object named by~$n$
\item at least one object in the domain satisfies $P(x)$
\item at least one object in the domain has the property expressed by $P(x)$,
\eit
\eit

}

\subsec{Expressing ``Everything''}{

\bits
\item ``Everything is a cube''\pause
\bits
\item $\sf \forall x\, Cube(x)$
\item True iff every object in the domain is a cube.
\eit
\item ``Everything is a large cube''
\bits
\item $\sf \forall x(Large(x) \land Cube(x))$
\item True iff every object in the domain is large and a cube
\eit
\item ``Everything is either left of or right of b''
\bits
\item $\sf \forall x(LeftOf(x, b) \lor RightOf(x, b))$
\item True iff every object in the domain is: either left of b or right of b
\eit
\eit
}

\subsec{Expressing ``Something''}{

\bits
\item ``Something is a cube''\pause
\bits
\item $\sf \exists x\, Cube(x)$
\item True iff at least one object in the domain is a cube.
\item ``There is a cube'', ``A cube exists''
\item ``There are cubes'', ``Cubes exist''
\eit
\item ``Something is a large cube''
\bits
\item $\sf \exists x(Large(x) \land Cube(x))$
\item True iff at least one object in the domain is large and a cube
\eit
\item ``Something is either left of or right of b''
\bits
\item $\sf \exists x(LeftOf(x, b) \lor RightOf(x, b))$
\item True iff at least one object in the domain is: either left of b or right of b
\item ``There are things left of or right of b''
\eit
\eit
}

\subsec{Expressing ``Nothing''}{

\bit
\item ``Nothing is a cube''
\bits
\item $\sf \forall x\, \lnot Cube(x)$
\item True iff every object in the domain is \emph{something other than} a cube
\item $\sf \lnot \exists x\, Cube(x)$
\item True iff it \emph{isn't} the case that at least one object is a cube
\eit
\item ``Nothing is a large cube'' 
\bits
\item $\sf \forall x\,\lnot(Large(x) \land Cube(x))$
\item $\sf \lnot\exists x(Large(x) \land Cube(x))$
\eit
\item ``Nothing is either left of or right of b''
\bits
\item $\sf \forall x\lnot(LeftOf(x, b) \lor RightOf(x, b))$
\item $\sf \lnot\exists x (LeftOf(x, b) \lor RightOf(x, b))$
\eit
\eit
}

\subsec{Expressing ``Something Isn't''}{

\bit
\item ``Something isn't a cube''
\bit
\item $\sf \exists x\, \lnot Cube(x)$
\eit
\item ``Something isn't a large cube''
\bit
\item $\sf \exists x\, \lnot(Large(x) \land Cube(x))$
\eit
\item ``Something isn't either left of or right of b''
\bit
\item $\sf \exists x\,\lnot(LeftOf(x, b) \lor RightOf(x, b))$
\eit
\eit
}


\subsec{Determiner Phrases}{

\bit
\item Determiners combine with noun phrases to make determiner phrases (DP):
\bit
\item ``\emph{A} large terahedron''
\item ``\emph{Three} cubes which are to the left of b''
\item ``\emph{Every} even number''
\item ``\emph{Some} cube(s) between d and e''
\item ``\emph{Most} philosophy majors''
\item ``\emph{Both} small cubes''
\eit
\eit
}

\subsec{Determiner Phrases in Sentences}{

\bit
\item DPs make subjects of sentence, just like names/constants and coordination constructions of them do
\bit
\item ``\emph{Claire and Alex} study logic''---``\emph{Most philosophy majors} study logic''
\item ``\emph{2} is prime''---``\emph{Some even number} is prime''
\item ``\emph{a} is between b and c''---``\emph{Every large cube} is between b and c''
\eit
\item We know how to translate the former---how do we deal with the latter?
\eit
}


\end{document}




