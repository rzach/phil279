\documentclass[bigger]{beamer}

\usetheme{Frankfurt}
\useoutertheme[subsection=false]{miniframes}
\setbeamercolor{structure}{parent=palette quaternary,fg=uofcred}
\setbeamercolor{section in toc}{fg=black,bg=white}
\setbeamercolor{alerted text}{fg=uofcred}
\setbeamercolor*{palette quaternary}{fg=white,bg=black}
\setbeamercolor*{palette secondary}{fg=black,bg=uofcyellow}
\setbeamercolor*{palette tertiary}{bg=uofcred,fg=white}
\setbeamercolor*{palette primary}{fg=white,bg=black}
\setbeamercolor*{sidebar}{fg=uofcred,bg=black!15!white}
\setbeamercolor*{palette sidebar primary}{fg=uofcred!10!black}
\setbeamercolor*{palette sidebar secondary}{fg=white}
\setbeamercolor*{palette sidebar tertiary}{fg=uofcred!50!black}
\setbeamercolor*{palette sidebar quaternary}{fg=black!10!white}

\setbeamercolor{titlelike}{parent=palette primary}
\setbeamercolor{frametitle}{fg=white,bg=black}
\setbeamercolor{frametitle right}{bg=gray!60!white}
\let\emph\alert
\beamertemplatenavigationsymbolsvertical
\def\bits{\begin{itemize}[<+->]}
\def\bens{\begin{enumerate}[<+->]}
\def\items#1{\item<#1>}
\def\uncovers#1{\uncover<#1>}
\def\pauses{\pause}

 % change to header-handout for handouts

% ====================
\title[Lecture 18]{Logic I F13 Lecture 18}
\date{November 19, 2013}
% ====================

% lecture header include

\usepackage{lplfitch,amsmath}
\usepackage{qtree,hyperref}
\usepackage{pgf,amssymb}

\author{Richard Zach}
\institute{Department of Philosophy\\
University of Calgary\\
  \href{http://ucalgary.ca/rzach/279}{ucalgary.ca/rzach/279}}

\definecolor{uofcred}{RGB}{227,39,38}
\definecolor{uofcyellow}{RGB}{255,210,0}

\DeclareSymbolFont{symbolsC}{U}{txsyc}{m}{n}
\DeclareMathSymbol{\strictif}{\mathrel}{symbolsC}{74}
\DeclareMathSymbol{\boxright}{\mathrel}{symbolsC}{128}

\let\IFF\Leftrightarrow
\let\iff\leftrightarrow
\let\impl\to

\def\T{{\color{green}\begin{colormixin}{25!black}\text{T}\end{colormixin}}}
\def\F{{\color{red}\begin{colormixin}{25!black}\text{F}\end{colormixin}}}

\long\def\subsec#1#2{\subsection{#1}\frame{\frametitle{#1} #2}}
\def\bit{\begin{itemize}[<1->]}
\def\eit{\end{itemize}}
\def\ben{\begin{enumerate}[<1->]}
\def\een{\end{enumerate}}

\makeatletter\let\@makefnmark\noindent\makeatother

%\setbeamercolor{footnote}{fg=black!70}

\def\foot#1{\footnotetext{\color{black!70}#1}}
\def\deemph#1{{\color{black!70}#1}}
\let\phi\varphi

\setbeamertemplate{theorems}[numbered]
%\useinnertheme{circles}
\setbeamertemplate{itemize subitems}[triangle]

\renewcommand{\beamertemplatetransparentcovereddynamic}{
  \beamersetuncovermixins
  {\opaqueness<1>{50}\opaqueness<2>{30}\opaqueness<3>{15}\opaqueness<4->{5}}%
  {\opaqueness<1>{50}\opaqueness<2>{30}\opaqueness<3>{15}\opaqueness<4->{5}}}
                                                                                
%\beamertemplatetransparentcovereddynamic

\defbeamertemplate*{footline}{my theme}
{%
  \leavevmode%
  \hbox{\begin{beamercolorbox}[wd=.5\paperwidth,ht=2.5ex,dp=1.125ex,leftskip=.3cm,rightskip=.3cm]{author in head/foot}%
    \insertframenumber/\inserttotalframenumber \hfil \usebeamerfont{author in head/foot}\insertshortauthor
  \end{beamercolorbox}%
  \begin{beamercolorbox}[wd=.5\paperwidth,ht=2.5ex,dp=1.125ex,leftskip=.3cm,rightskip=.3cm plus1fil]{title in head/foot}%
    \usebeamerfont{title in head/foot}Logic I F13---\insertshorttitle---\insertdate
  \end{beamercolorbox}}%
  \vskip0pt%
}

\begin{document}
\setlength{\fitchargwidth}{7em}
\setlength{\fitchprfwidth}{7em}

\frame{\frametitle{\insertshorttitle\ (\insertdate)}
\tableofcontents[hidesubsections]
}










\setlength{\fitchprfwidth}{14em}

\section{Review}

\subsec{Rules for $\forall$}{

\fitchctx{
\nline[$k.$]{\forall x\, A(x)}\\
\fpline{\quad A(b)}[\lalle{$k$}]
}

\bigskip 

\fitchctx{
\boxedsubproof[$m$.]{c}{}{\ellipsesline \\ \nline[$n.$]{A(c)}}
\fpline{\quad \forall x\,A(x)}[\lalli{$m$--$n$}]}

\bit
\item c is special: c must not appear anywhere outside the subproof
\eit
}

\subsec{Rules for $\forall$}{
\setlength{\fitchctxwidth}{13em}

\fitchctx{
\boxedsubproof[$m$.]{c}{A(c)}{\ellipsesline \\ \nline[$n.$]{B(c)}}
\fpline{\quad \forall x(A(x) \to B(x))}[\lalli{$m$--$n$}]}


\bit
\item c is special: c must not appear anywhere outside the subproof
\eit
}

\subsec{Rules for $\exists$}{

\fitchctx{\pline[$m.$]{A(b)}\\
\fpline{\quad \exists x\, A(x)}[\lexii{$m$}]}

\bigskip

\fitchctx{
\pline[$k$.]{\exists x\, A(x)}\\
\boxedsubproof[$m$.]{c}{A(c)}{\ellipsesline \\ \nline[$n.$]{B}}
\fpline{\quad B}[\lexie{k}{$m$--$n$}]}
\bit
\item c is special: c must not appear anywhere outside the subproof
\eit
}

\subsec{Everybody Loves my Baby}{

\setlength{\fitchprfwidth}{13em}

\href{http://www.youtube.com/watch?v=cxS0FpTNir0}{Red Nichols (1935)}\\
\href{http://www.youtube.com/watch?v=TyD9rZ2bOq4}{Doris Day (1955)}\\
\href{http://www.youtube.com/watch?v=Y4wWpeGaNBw}{Brigitte Bardot (1968)}

\pause
\bigskip

\fitchprf{\tline{Everybody loves my baby}\\
\tline{My baby loves noone but me}}{
\tline{My baby is me}}

\bigskip

\fitchprf{\pline{\forall x\, Loves(x, b)}\\
\pline{\forall x(Loves(b, x) \lif x = i)}}{
\pline{b = i}}
}

\section{Strategies for Proofs with Quantifiers}

\setlength{\fitchprfwidth}{8em}

\subsec{Strategy for Proving Universal Sentences}{

\begin{tabular}{lp{10em}}
\fitchprf{\pline[1.]{}}{
\ellipsesline\\
\pline[5.]{\forall x\, A(x)}[?]
} &
Suppose we want to prove $\sf\forall x\, A(x)$.  For this, we have to use $\forall$Intro.
\end{tabular}
}

\subsec{Strategy for Proving Universal Sentences}{

\begin{tabular}{lp{10em}}
\fitchprf{\pline[1.]{}}{
\boxedsubproof[2.]{c}{}{
\ellipsesline\\
\pline[4.]{A(c)}
}
\pline[5.]{\forall x\, A(x)}[\lalli{2--4}]
} &
For $\forall$Intro, we need a subproof that ends with a substitution instance of
$\sf \forall x\, A(x)$ where we replace every free occurrence of x in
A(x) by a \emph{new} constant c. c has to be boxed in the assumption
line.  This always works, so do it earlier rather than later.
\end{tabular}
}


\subsec{Strategy for Proving Universal Sentences}{

\setlength{\fitchprfwidth}{10em}

\begin{tabular}{@{}l@{}p{10em}@{}}
\fitchprf{\pline[1.]{}}{
\boxedsubproof[2.]{c}{A(c)}{
\ellipsesline\\
\pline[4.]{B(c)}
}
\pline[5.]{\forall x(A(x) \to B(x))}[\lalli{2--4}]
} &
If the wff after the universal quantifier is a conditional, we can use the special form of $\forall$Intro instead of using a $\to$Intro inside the boxed subproof.
\end{tabular}
}


\setlength{\fitchprfwidth}{8em}

\subsec{Strategy for Proving Existential Sentences}{

\begin{tabular}{lp{10em}}
\fitchprf{\pline[1.]{}}{
\ellipsesline\\
\pline[3.]{\exists x\, A(x)}[?]
} &
We want to prove $\sf \exists x\, A(x)$.  For this, we can often use
$\exists$Intro.
\end{tabular}
}

\subsec{Strategy for Proving Existential Sentences}{

\begin{tabular}{lp{10em}}
\fitchprf{\pline[1.]{}}{
\ellipsesline\\
\pline[2.]{A(b)}\\
\pline[3.]{\exists x\, A(x)}[\lexii{2}]
} &
For $\exists$Intro, we need a substitution instance A(b) of A(x).
Any b will do. But save this strategy for last, just like $\lor$Intro. Often you can only  prove A(b) if
you're in a subproof, and b is a boxed constant in a surrounding subproof.
\end{tabular}
}

\subsec{Strategy for Proving Existential Sentences}{

\begin{tabular}{lp{10em}}
\fitchprf{\pline[1.]{}}{ \ellipsesline\\ \pline[2.]{A(b,
    b)}\\ \pline[3.]{\exists x\, A(x,
    b)}[\lexii{2}]\\ \pline[4.]{\exists x\, A(b,
    x)}[\lexii{2}]\\ \pline[5.]{\exists y\exists x\, A(y,
    x)}[\lexii{4}]\\ 
\pline[6.]{\exists x\, A(x, x)}[\lexii{2}]} & If b occurs more than once in your sentence,
you don't have to replace all occurrences of b by x.
\end{tabular}
}
\subsec{Strategy for Using Existential Premises}{

\begin{tabular}{lp{10em}}
\fitchprf{\pline[1.]{\exists x\, A(x)}}{
\ellipsesline\\
\pline[4.]{B}[?]
} &
Suppose you want to prove some sentence B, and you are ready to use
$\sf\exists x\, A(x)$. ($\sf\exists x\, A(x)$ might be something you've
proved, or is a premise, or an assumption of a subproof).
\end{tabular}
}

\subsec{Strategy for Using Existential Premises}{

\begin{tabular}{lp{10em}}
\fitchprf{\pline[1.]{\exists x\, A(x)}}{
\boxedsubproof[2.]{c}{A(c)}{
\ellipsesline\\
\pline[3.]{B}
}
\pline[4.]{B}[\lexie{1}{2--3}]
} &
To get B from $\sf\exists x\, A(x)$, set up a subproof where you assume
A(c). c has to be \emph{new} (in particular, it can't be in B), and boxed.
In the subproof, look for a proof of B.  This always works, so do it earlier rather than later.
\end{tabular}
}


\subsec{Strategy for Using Universal Premises}{

\begin{tabular}{lp{10em}}
\fitchprf{\pline[1.]{\forall x\, A(x)}}{
\ellipsesline} &
To use a universal sentence which you've proved, assumed, or is one of
your premises, use $\forall$Elim. 
\end{tabular}
}

\subsec{Strategy for Using Universal Premises}{

\begin{tabular}{lp{10em}}
\fitchprf{\pline[1.]{\forall x\, A(x)}}{
\ellipsesline \\
\pline[$7$.]{A(b)}[\lalle{1}]
} &
To do that, you can write down any substitution instance of A(x), i.e.,
A(b) where b is any constant. You can always do that, but it's best to wait until you know which A(b) you need.
\end{tabular}
}

\subsec{Strategy for Using Universal Premises}{

\begin{tabular}{lp{10em}}
\fitchprf{\pline[1.]{\forall x\forall y\forall z\, A(x, y, z)}}{
\ellipsesline \\
\pline[$7$.]{A(a, b, c)}[\lalle{1}]
}&
\raggedright If there is more than one $\forall$ (they have to be
together ``in a block''), you can replace all variables at once. Here
we replaced x by a, y by b, and z by c.
\end{tabular}
}

\subsec{Strategy for Using Universal Premises}{

\begin{tabular}{lp{10em}}
\fitchprf{\pline[1.]{\forall x\forall y\forall z\, A(x, y, z)}}{
\ellipsesline \\
\pline[$7$.]{A(c, a, c)}[\lalle{1}]
} &
You don't have to keep the alphabetical order, and the constants don't
all have to be distinct.  E.g., here line 7 comes from line 1 by replacing
x by c, y by a, and z by c.
\end{tabular}
}

\subsec{Tips}{

\bit
\item Use the strategies for $\forall$Intro and $\exists$Elim as early as possible, and those for $\forall$Elim and $\exists$Intro as late as possible
\item \emph{When you have a choice between a rule with a subproof and one without, pick the one with the subproof.} (Except indirect proof for a sentence without $\lnot$!)
\item If you're thinking of looking for a proof of A(b) to get $\sf\exists x\, A(x)$ using $\exists$Intro, you can test if A(b) is a consequence of the premises and assumptions using \textbf{FO~Con}
\item If you type ``$\sf: x > b$'', click on $\sf\forall x\,A(x)$, select $\forall$Elim, and check the step, Fitch will fill in A(b) in the line
\item If you type ``$\sf: b > x$'', click on A(b), select $\exists$Intro, and check the step, Fitch will fill in $\sf\exists x\,A(x)$ in the line
\eit
}

\subsec{Examples}{

We'll now apply these strategies (and some strategies we remember from 
propositional proofs) to give proofs of
\begin{align*}\sf
\sf\lnot \forall x\, A(x) & \sf\leftrightarrow \exists x\, \lnot A(x)\\
\sf\exists x (A(x) \to B) & \sf\leftrightarrow (\forall x\, A(x) \to B)\\
\end{align*}
\fitchprf{\pline{\forall x\exists y\, R(x, y)}\\
\pline{\forall x\forall y(R(x, y) \to R(y, x)}}{
\pline{\forall x\, R(x, x)}}

}
\end{document} 





