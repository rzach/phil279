\documentclass[bigger]{beamer}

\usepackage{tikz}
\usetikzlibrary{graphs, arrows, shapes}

\usetheme{Frankfurt}
\useoutertheme[subsection=false]{miniframes}
\setbeamercolor{structure}{parent=palette quaternary,fg=uofcred}
\setbeamercolor{section in toc}{fg=black,bg=white}
\setbeamercolor{alerted text}{fg=uofcred}
\setbeamercolor*{palette quaternary}{fg=white,bg=black}
\setbeamercolor*{palette secondary}{fg=black,bg=uofcyellow}
\setbeamercolor*{palette tertiary}{bg=uofcred,fg=white}
\setbeamercolor*{palette primary}{fg=white,bg=black}
\setbeamercolor*{sidebar}{fg=uofcred,bg=black!15!white}
\setbeamercolor*{palette sidebar primary}{fg=uofcred!10!black}
\setbeamercolor*{palette sidebar secondary}{fg=white}
\setbeamercolor*{palette sidebar tertiary}{fg=uofcred!50!black}
\setbeamercolor*{palette sidebar quaternary}{fg=black!10!white}

\setbeamercolor{titlelike}{parent=palette primary}
\setbeamercolor{frametitle}{fg=white,bg=black}
\setbeamercolor{frametitle right}{bg=gray!60!white}
\let\emph\alert
\beamertemplatenavigationsymbolsvertical
\def\bits{\begin{itemize}[<+->]}
\def\bens{\begin{enumerate}[<+->]}
\def\items#1{\item<#1>}
\def\uncovers#1{\uncover<#1>}
\def\pauses{\pause}

 % change to header-handout for handouts

% ====================
\title[Lecture 3]{Logic I F13 Lecture 3}
\date{September 17, 2013}
% ====================


% lecture header include

\usepackage{lplfitch,amsmath}
\usepackage{qtree,hyperref}
\usepackage{pgf,amssymb}

\author{Richard Zach}
\institute{Department of Philosophy\\
University of Calgary\\
  \href{http://ucalgary.ca/rzach/279}{ucalgary.ca/rzach/279}}

\definecolor{uofcred}{RGB}{227,39,38}
\definecolor{uofcyellow}{RGB}{255,210,0}

\DeclareSymbolFont{symbolsC}{U}{txsyc}{m}{n}
\DeclareMathSymbol{\strictif}{\mathrel}{symbolsC}{74}
\DeclareMathSymbol{\boxright}{\mathrel}{symbolsC}{128}

\let\IFF\Leftrightarrow
\let\iff\leftrightarrow
\let\impl\to

\def\T{{\color{green}\begin{colormixin}{25!black}\text{T}\end{colormixin}}}
\def\F{{\color{red}\begin{colormixin}{25!black}\text{F}\end{colormixin}}}

\long\def\subsec#1#2{\subsection{#1}\frame{\frametitle{#1} #2}}
\def\bit{\begin{itemize}[<1->]}
\def\eit{\end{itemize}}
\def\ben{\begin{enumerate}[<1->]}
\def\een{\end{enumerate}}

\makeatletter\let\@makefnmark\noindent\makeatother

%\setbeamercolor{footnote}{fg=black!70}

\def\foot#1{\footnotetext{\color{black!70}#1}}
\def\deemph#1{{\color{black!70}#1}}
\let\phi\varphi

\setbeamertemplate{theorems}[numbered]
%\useinnertheme{circles}
\setbeamertemplate{itemize subitems}[triangle]

\renewcommand{\beamertemplatetransparentcovereddynamic}{
  \beamersetuncovermixins
  {\opaqueness<1>{50}\opaqueness<2>{30}\opaqueness<3>{15}\opaqueness<4->{5}}%
  {\opaqueness<1>{50}\opaqueness<2>{30}\opaqueness<3>{15}\opaqueness<4->{5}}}
                                                                                
%\beamertemplatetransparentcovereddynamic

\defbeamertemplate*{footline}{my theme}
{%
  \leavevmode%
  \hbox{\begin{beamercolorbox}[wd=.5\paperwidth,ht=2.5ex,dp=1.125ex,leftskip=.3cm,rightskip=.3cm]{author in head/foot}%
    \insertframenumber/\inserttotalframenumber \hfil \usebeamerfont{author in head/foot}\insertshortauthor
  \end{beamercolorbox}%
  \begin{beamercolorbox}[wd=.5\paperwidth,ht=2.5ex,dp=1.125ex,leftskip=.3cm,rightskip=.3cm plus1fil]{title in head/foot}%
    \usebeamerfont{title in head/foot}Logic I F13---\insertshorttitle---\insertdate
  \end{beamercolorbox}}%
  \vskip0pt%
}

\begin{document}
\setlength{\fitchargwidth}{7em}
\setlength{\fitchprfwidth}{7em}

\frame{\frametitle{\insertshorttitle\ (\insertdate)}
\tableofcontents[hidesubsections]
}










\setlength{\fitchargwidth}{10cm}

\section{Arguments and Validity}

\subsec{What are Arguments?}{

\begin{block}{Arguments in Logic}

\fitcharg{
$\left.\begin{array}{@{}c@{}}
\formula{P}_1 \\
\vdots\\
\formula{P}_n
\end{array}
\right\} \quad \text{Premises}$}{
\formula{C\qquad  \text{Conclusion}}}
\end{block}
\bit
\item Argument is \emph{valid} if conclusion follows from (is a consequence of) the premises
\item Otherwise it is \emph{invalid}
\eit

}


\subsec{Logical Consequence}{

\bit
\item $C$ is a \emph{logical consequence} of $P_1$, \dots, $P_n$ iff $C$ is true in every possible world in which $P_1$, \dots, $P_n$ are all true together 
\item NB: This does \emph{not} say that $P_1$, \dots, $P_n$ and $C$
  have to be true in every possible world (or even one)! 
\item $C$ is a logical consequence of $P_1$, \dots, $P_n$ iff there is no possible world in which $P_1$, \dots, $P_n$ are all true together while $C$ is false
\item $C$ is \emph{not} a \emph{consequence} of $P_1$, \dots, $P_n$ iff there is a world in which $P_1$, \dots, $P_n$ are all true but $C$ is false.
\eit
}

\subsec{Validity}{

\bit
\item An argument is \emph{logically valid} iff its conclusion is true in every possible world in which the premises are all true 
\item An argument is \emph{logically invalid} iff there is a world in which its premises all true but its conclusion is false.\\[2ex]
\item Determine invalidity by finding a possible world in which premises are true and conclusion is false.
\item No such world can be found: argument is (probably) valid
\item To establish validity, give a proof 
\eit
}

\subsec{Soundness}{

\bit
\item An argument is \emph{sound} (with respect to a given world) iff it is valid \emph{and} its premises are true in the world
\item An argument is \emph{unsound} (with respect to a given world) iff it is invalid \emph{or} at least one premise is false in the world (or both)
\eit

}

\subsec{Examples}{

\fitcharg{
  \formula{LeftOf(a, b)} \\
  \formula{RightOf(c, b)}}{\formula{Between(b, a, c)}}

}

\subsec{Examples}{

\fitcharg{
\formula{LeftOf(a, b)} \\
\formula{RightOf(d, c)}\\
\formula{b = d}}{
\formula{LeftOf(a, c)}}

}

\subsec{Examples}{

\fitcharg{
\formula{LeftOf(a, b)} \\
\formula{RightOf(c, d)} \\
\formula{b = d}}{
\formula{LeftOf(a, c)}}

}

\subsec{First-order Consequence}{

\bit
\item $C$ is a logical consequence of $P_1$, \dots, $P_n$ iff there is no possible world in which $P_1$, \dots, $P_n$ are all true together while $C$ is false
\item $C$ is a \emph{first-order consequence} of $P_1$, \dots, $P_n$ iff there is no \emph{first-order structure} in which $P_1$, \dots, $P_n$ are all true together while $C$ is false
\eit
\begin{columns}
\begin{column}{5cm}\fitcharg{\formula{LeftOf(a, b)} \\
\formula{RightOf(c, b)}}{\formula{LeftOf(a, c)}}
\end{column}
\begin{column}{5cm}
\tikz [x=2cm] \graph {
        "$\bullet_a$" ->[label=LeftOf] "$\bullet_b$" ;
        "$\bullet_b$" <-[dashed] "$\bullet_c$";
};
\end{column}
\end{columns}

}



\section{Properties of Predicates}

\subsec{Reflexive Predicates}{

\bits
\item $P(\_,\_)$ is \emph{reflexive} iff $P(a, a)$ is always true
\item $=$, SameSize, SameShape, SameCol, SameRow
\item Extensions of reflexive predicates:
\[
\tikz [x=2cm] \graph {
        "$\bullet_1$" ->[in=60, out=120, loop] "$\bullet_1$" -!-
        "$\bullet_2$" ->[in=60, out=120, loop] "$\bullet_2$";
};
\]
\eit
}

\subsec{Symmetric Predicates}{

\bits
\item $P(\_,\_)$ is \emph{symmetric} iff: whenever $P(a, b)$ is true, $P(b, a)$ is also true, and vice versa
\item $=$, SameSize, SameShape, SameCol, SameRow
\item Adjoins
\item Extensions of symmetric predicates:
\[\tikz [x=2cm] \graph {
        "$\bullet_1$" ->[bend left] "$\bullet_2$";
        "$\bullet_1$" <-[bend right] "$\bullet_2$";
};
\]
\eit
}

\subsec{Transitive Predicates}{

\bits
\item $P(\_,\_)$ is \emph{transitive} iff: whenever $P(a, b)$ and $P(b, c)$ are true, $P(a, c)$ is also true
\item $=$, SameSize, SameShape, SameCol, SameRow
\item Larger, Smaller, LeftOf, RightOf, FrontOf, BackOf
\item \emph{not} Adjoins
\item Extensions of transitive predicates:
\[\tikz [x=2cm] \graph {
        "$\bullet_1$" ->[bend left] "$\bullet_2$" ->[bend left] "$\bullet_3$";
        "$\bullet_1$" ->[bend right] "$\bullet_3$";
};
\]
\eit
}

\subsec{Inverse Predicates}{

\bits
\item $P(\_,\_)$ and $Q(\_, \_)$ are \emph{inverses} iff: whenever $P(a, b)$ is true, $Q(b, a)$ is also true, and vice versa
\item $=$, SameSize, SameShape, SameCol, SameRow, Adjoins are inverses of themselves
\item Larger---Smaller, LeftOf---RightOf, FrontOf---BackOf
\item Extensions of inverse predicates:
\[
\tikz [x=2cm] \graph {
        "$\bullet_1$" ->[bend left] "$\bullet_2$";
        "$\bullet_1$" <-[bend right,
         dashed] "$\bullet_2$";
};
\]
\eit
}

\subsec{Indiscernibility of Identicals}{

If $P(a)$ is true, and $a = b$ is also true, then $P(b)$ is true

}

\section{Informal Proofs}

\subsec{Proofs}{

\begin{block}{Definition}

A \emph{proof} of conclusion $C$ from premises $P_1$, \dots, $P_n$ is
a demonstration that $C$ is a consequence of $P_1$, \dots, $P_n$ which
\bit
\item proceeds \emph{step-by-step}
\item each step is \emph{justified}
\item is \emph{rigorous} (no gaps)
\eit 
\end{block}
}

\subsec{Informal vs Formal Proof}{

\bit
\item Informal proofs in English
\item Formal proofs 
\bit
\item in first-order logic, 
\item using system Fitch, 
\item steps justified by \emph{inference rules}
\eit
\eit

}

\subsec{Example}{

\begin{columns}
\begin{column}{3cm}
\fitcharg{\formula{LeftOf(a, b)} \\
\formula{RightOf(c, d)}\\
\formula{b = d}}{\formula{LeftOf(a, c)}}
\end{column}
\begin{column}{7cm}
\bens
\item We are told that $c$ is right of $d$ (premise 2)
\item We are told that $b$ is identical to $d$ (premise 3)
\item Thus, $c$ is right of $b$ (from 1, 2, by the
  indiscernibility of identicals)
\item So $b$ is left of $c$ (from 3, as ``left of'' and ``right of'' are inverses) 
\item We are told that $a$ is left of $b$ (premise 1)
\item So $a$ is left of $c$ (from 5, 4 as ``left of'' is transitive)\\
\qquad QED
\een
\end{column}
\end{columns}

}

\subsec{Example: Formal Proof}{

\setlength{\fitchprfwidth}{10em}

\fitchprf{\pline[1.]{LeftOf(a, b)}\\
\pline[2.]{RightOf(c, d)}\\
\pline[3.]{b = d}}{
\pline[4.]{RightOf(c, b)}{\eqe{2}{3}}\\
\pline[5.]{LeftOf(b, c)}{{\bf AnaCon:} 4}\\
\pline[6.]{LeftOf(a, c)}{{\bf AnaCon:} 1, 5}
}

}


\end{document}
