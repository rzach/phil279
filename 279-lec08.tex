\documentclass[bigger]{beamer}

\usetheme{Frankfurt}
\useoutertheme[subsection=false]{miniframes}
\setbeamercolor{structure}{parent=palette quaternary,fg=uofcred}
\setbeamercolor{section in toc}{fg=black,bg=white}
\setbeamercolor{alerted text}{fg=uofcred}
\setbeamercolor*{palette quaternary}{fg=white,bg=black}
\setbeamercolor*{palette secondary}{fg=black,bg=uofcyellow}
\setbeamercolor*{palette tertiary}{bg=uofcred,fg=white}
\setbeamercolor*{palette primary}{fg=white,bg=black}
\setbeamercolor*{sidebar}{fg=uofcred,bg=black!15!white}
\setbeamercolor*{palette sidebar primary}{fg=uofcred!10!black}
\setbeamercolor*{palette sidebar secondary}{fg=white}
\setbeamercolor*{palette sidebar tertiary}{fg=uofcred!50!black}
\setbeamercolor*{palette sidebar quaternary}{fg=black!10!white}

\setbeamercolor{titlelike}{parent=palette primary}
\setbeamercolor{frametitle}{fg=white,bg=black}
\setbeamercolor{frametitle right}{bg=gray!60!white}
\let\emph\alert
\beamertemplatenavigationsymbolsvertical
\def\bits{\begin{itemize}[<+->]}
\def\bens{\begin{enumerate}[<+->]}
\def\items#1{\item<#1>}
\def\uncovers#1{\uncover<#1>}
\def\pauses{\pause}

 % change to header-handout for handouts

% ====================
\title[Lecture 8]{Logic I F13 Lecture 8}
\date{October 3, 2013}
% ====================

% lecture header include

\usepackage{lplfitch,amsmath}
\usepackage{qtree,hyperref}
\usepackage{pgf,amssymb}

\author{Richard Zach}
\institute{Department of Philosophy\\
University of Calgary\\
  \href{http://ucalgary.ca/rzach/279}{ucalgary.ca/rzach/279}}

\definecolor{uofcred}{RGB}{227,39,38}
\definecolor{uofcyellow}{RGB}{255,210,0}

\DeclareSymbolFont{symbolsC}{U}{txsyc}{m}{n}
\DeclareMathSymbol{\strictif}{\mathrel}{symbolsC}{74}
\DeclareMathSymbol{\boxright}{\mathrel}{symbolsC}{128}

\let\IFF\Leftrightarrow
\let\iff\leftrightarrow
\let\impl\to

\def\T{{\color{green}\begin{colormixin}{25!black}\text{T}\end{colormixin}}}
\def\F{{\color{red}\begin{colormixin}{25!black}\text{F}\end{colormixin}}}

\long\def\subsec#1#2{\subsection{#1}\frame{\frametitle{#1} #2}}
\def\bit{\begin{itemize}[<1->]}
\def\eit{\end{itemize}}
\def\ben{\begin{enumerate}[<1->]}
\def\een{\end{enumerate}}

\makeatletter\let\@makefnmark\noindent\makeatother

%\setbeamercolor{footnote}{fg=black!70}

\def\foot#1{\footnotetext{\color{black!70}#1}}
\def\deemph#1{{\color{black!70}#1}}
\let\phi\varphi

\setbeamertemplate{theorems}[numbered]
%\useinnertheme{circles}
\setbeamertemplate{itemize subitems}[triangle]

\renewcommand{\beamertemplatetransparentcovereddynamic}{
  \beamersetuncovermixins
  {\opaqueness<1>{50}\opaqueness<2>{30}\opaqueness<3>{15}\opaqueness<4->{5}}%
  {\opaqueness<1>{50}\opaqueness<2>{30}\opaqueness<3>{15}\opaqueness<4->{5}}}
                                                                                
%\beamertemplatetransparentcovereddynamic

\defbeamertemplate*{footline}{my theme}
{%
  \leavevmode%
  \hbox{\begin{beamercolorbox}[wd=.5\paperwidth,ht=2.5ex,dp=1.125ex,leftskip=.3cm,rightskip=.3cm]{author in head/foot}%
    \insertframenumber/\inserttotalframenumber \hfil \usebeamerfont{author in head/foot}\insertshortauthor
  \end{beamercolorbox}%
  \begin{beamercolorbox}[wd=.5\paperwidth,ht=2.5ex,dp=1.125ex,leftskip=.3cm,rightskip=.3cm plus1fil]{title in head/foot}%
    \usebeamerfont{title in head/foot}Logic I F13---\insertshorttitle---\insertdate
  \end{beamercolorbox}}%
  \vskip0pt%
}

\begin{document}
\setlength{\fitchargwidth}{7em}
\setlength{\fitchprfwidth}{7em}

\frame{\frametitle{\insertshorttitle\ (\insertdate)}
\tableofcontents[hidesubsections]
}










\section{Review of Rules}

\subsec{$\land$Intro}{

If $A$ and $B$ are both true, then $A \land B$ is true
\bigskip
\setlength{\fitchprfwidth}{1.2in}
\fitchctx{
\nline[$n.$]{A}\\
\nline[$m.$]{B}\\
\fpline[\quad]{A \land B}[\landi{n}{m}]}

}

\subsec{$\land$Elim}{

If $A \land B$ is true, then $A$ ($B$) is true
\bigskip
\begin{tabular}{@{}ll@{}}
\setlength{\fitchprfwidth}{1in}
\fitchctx{
\nline[$n.$]{A \land B}\\
\fpline[\quad]{A}[\lande{n}]}
&\setlength{\fitchprfwidth}{1in}
\fitchctx{
\nline[$n.$]{A \land B}\\
\fpline[\quad]{B}[\lande{n}]}
\end{tabular}
}

\subsec{$\lor$Intro}{

If $A$ ($B$) is true, then $A \lor B$ is true
\bigskip
\begin{tabular}{@{}ll@{}}
\setlength{\fitchprfwidth}{1in}
\fitchctx{
\nline[$n.$]{A}\\
\fpline[\quad]{A \lor B}[\lori{n}]}
&\setlength{\fitchprfwidth}{1in}
\fitchctx{
\nline[$n.$]{B}\\
\fpline[\quad]{A \lor B}[\lori{n}]}
\end{tabular}
}

\subsec{$\lor$Elim}{

Formal version of proof by cases
\bigskip
\bigskip
\setlength{\fitchprfwidth}{1.2in}
\fitchctx{
\nline[$j.$]{A \lor B}\\
\subproof{\nline[$k.$]{A}}{\dots\\ \nline[$l.$]{C}}
\subproof{\nline[$m.$]{B}}{\dots\\ \nline[$n.$]{C}}
\fpline[\quad]{C}[\lore{$j$}{$k$--$l$}{$m$--$n$}]}


}

\subsec{Example}{

\fitchprf{\pline[1.]{A \lor (B \land C)}}{
\subproof{\pline[2.]{A}}{\pline[3.]{A \lor B}[\lori{2}]}
\subproof{\pline[4.]{B \land C}}{
\pline[5.]{B}[\lande{4}]\\
\pline[6.]{A \lor B}[\lori{5}]}
\pline[7.]{A \lor B}[\lore{1}{2--3}{4--6}]
}}


\section{Indirect Proof}


\subsec{Indirect Proof}{

\begin{quote}
To prove that $A$ is true (false), you may proceed by pretending that
$A$ is false (true), and showing that that's impossible.
\end{quote}

\bit
\item ``show that's impossible'': prove something known to be false,
  e.g., proving a claim that is in conflict with the premises, or
  outright contradictory
\item ``proof by contradiction'' = ``reductio (ad absurdum)'' =
  indirect proof \eit

}

\subsec{Galileo on Falling Bodies}{ 

From Galileo Galilei's, \textit{Discourses and Mathematical
  Demonstrations Relating to Two New Sciences}, 1638.
\medskip

\emph{Simplicio.}  There can be no doubt but
  that one and the same body moving in a single medium has a fixed
  velocity which is determined by nature and which cannot be increased
  except by the addition of momentum or diminished except by some
  resistance which retards it.
\medskip

\emph{Salviati.}  If then we take two bodies whose natural speeds are
different, it is clear that on uniting the two, the more rapid one
will be partly retarded by the slower, and the slower will be somewhat
hastened by the swifter. Do you not agree with me in this opinion?
\medskip 

\emph{Simp.}  You are unquestionably right.}
\subsec{Galileo on Falling Bodies}{ 
 
\emph{Salv.}  But if this is true, and if a large stone moves with a speed
of, say, eight while a smaller moves with a speed of four, then when
they are united, the system will move with a speed less than eight;
but the two stones when tied together make a stone larger than that
which before moved with a speed of eight.  Hence the heavier body
moves with less speed than the lighter; an effect which is contrary to
your supposition.  Thus you see how, from your assumption that the
heavier body moves more rapidly than the lighter one, I infer that the
heavier body moves more slowly.
}


\subsec{Disjunctive Syllogism}{

\setlength{\fitchprfwidth}{1.2in}

\begin{columns}
\begin{column}{3cm}
\fitchprf{
\nline{A \lor B}\\
\nline{\lnot A}}{\nline{B}}
\end{column}
\begin{column}{7cm}
\bens
\item Suppose $\sf B$
were not true.  
\item By premise 1, $\sf A \lor B$ is true. 
\item Case 1.  $\sf A$ is
true. By premise 2, $\sf \lnot A$ is true, so we have a contradiction.
\item Case 2. $\sf B$ is true. This contradicts the assumption that $\sf B$ is false, so again we have a contradiction. 
\item So by proof by
cases, we have proved a contradiction from the assumption that $\sf 
B$ is not true. 
\item So $\sf B$ must be true, by indirect proof. QED.
\een
\end{column}
\end{columns}
}


\subsec{Rules for $\lnot$}{

\setlength{\fitchprfwidth}{.9in}
\fitchctx{
\subproof{\nline[$k.$]{A}}{\dots\\ \nline[$l.$]{\bot}}
\fpline[\quad]{\lnot A}[\lnoti{$k$--$l$}]}
\fitchctx{
\subproof{\nline[$k.$]{\lnot A}}{\dots\\ \nline[$l.$]{\bot}}
\fpline[\quad]{A}[\lnoti{$k$--$l$}]}

\bigskip
\fitchctx{
\nline[$k.$]{\lnot\lnot A}\\
\fpline[\quad]{A}[\lnote{k}]}

}

\subsec{Rules for $\bot$}{

\fitchctx{
\nline[$k.$]{A}\\
\nline[$l.$]{\lnot A}\\
\fpline[\quad]{\bot}[\lfalsei{k}{l}]}
\bigskip
\setlength{\fitchprfwidth}{.9in}
\fitchctx{
\nline[$k.$]{\bot}\\
\fpline[\quad]{A}[\lfalsee{k}]}

}

\subsec{Disjunctive Syllogism}{

\setlength{\fitchprfwidth}{1.2in}

\fitchprf{
\nline{A \lor B}\\
\nline{\lnot A}}{\nline{B}}
}

\subsec{De Morgan's Law}{
\setlength{\fitchprfwidth}{1.2in}

\fitchprf{
\nline{A \lor B}}{\nline{\lnot(\lnot A \land \lnot B)}}
}

\section{Proofs from Contradictory Premises}

\subsec{Proofs from Contradictory Premises}{

\begin{quote}
Any argument with TT-contradictory premises is tautologically valid.
\end{quote}

}

\subsec{Proofs from Contradictory Premises}{

\begin{columns}
\begin{column}{3cm}
\fitchprf{
\pline{Small(b)}\\
\pline{\lnot Small(b)}}{
\pline{Tet(a)}}
\end{column}
\begin{column}{7cm}
\bens
\item Suppose, by way of contradiction, that $\sf \lnot Tet(a)$. 
\item We are told (premise 1) that $\sf
Small(b)$. 
\item We are also told (premise 2) that $\sf \lnot
Small(b)$. 
\item That's a contradiction. 
\item So $Tet(a)$ by indirect proof. QED.
\een
\end{column}
\end{columns}
}

\end{document}

\section{Proof Strategies}

\subsec{Strategy for Proving Conjunctions}{

\begin{tabular}{@{}lp{13em}}
\fitchprf{
   \pline[1.]{}
}{
 \ellipsesline \\
 \pline[2.]{} \\
 \ellipsesline \\
 \pline[3.]{} \\
 \pline[4.]{P \land Q}[\hphantom{\landi{2}{3}}]
} 
  &
If we want to prove $\sf P \land Q$, we have to use $\land$Intro.
\end{tabular}

}

\subsec{Strategy for Proving Conjunctions}{

\begin{tabular}{@{}lp{13em}}
  \fitchprf{\pline[1.]{}}{
    \ellipsesline \\
    \pline[2.]{P} \\
    \ellipsesline \\
    \pline[3.]{Q} \\
    \pline[4.]{P \land Q}[\landi{2}{3}]
  } 
  &
  For $\land$Intro, we need to prove $\sf P$ first, then prove $\sf Q$. For the last line,
  we'll cite the lines where we proved $\sf P$ and  $\sf Q$, and use $\land$Intro.
\end{tabular}
}



\subsec{Strategy for Using Conjunctions}{

\begin{tabular}{@{}lp{13em}}
\fitchprf{
   \pline[1.]{P \land Q}
}{
 \pline{}[\hphantom{\lande{1}}] \\
 \pline{} \\
 \ellipsesline \\
 \pline{R}
} 
  &
If we want to use a sentence of the form $\sf P \land Q$ which is one
of our premises, or a sentence we've already proved, use $\land$Elim.
\end{tabular}
}

\subsec{Strategy for Using Conjunctions}{

\begin{tabular}{@{}lp{13em}}
\fitchprf{
   \pline[1.]{P \land Q}
}{
 \pline[2.]{P}[\lande{1}]\\
 \pline[3.]{Q}[\lande{1}]\\
 \ellipsesline \\
 \pline{R}
} 
& Use $\land$Elim to ``split up'' $\sf P \land Q$, proving both $\sf
P$ and $\sf Q$.  We'll probably need them further down in the proof.
\end{tabular}
}


\subsec{Strategy for Proving Disjunctions}{

\begin{tabular}{@{}lp{13em}}
\fitchprf{
   \pline[1.]{}
}{
 \ellipsesline \\
 \pline[2.]{} \\
 \pline[3.]{P \lor Q}[\hphantom{\lori{2}}]
} 
  &
We want to prove $\sf P \lor Q$. We could do this using  $\lor$Intro.
\end{tabular}
}


\subsec{Strategy for Proving Disjunctions}{

\begin{tabular}{@{}lp{13em}}
\fitchprf{\pline[1.]{}}{
\ellipsesline \\
\pline[2.]{P} \\
\pline[3.]{P \lor Q}[\lori{2}]
} 
&
For $\lor$Intro, we need to prove one of the disjuncts.
We'll try $\sf P$ first. For the last line,
we'll cite the line where we proved $\sf P$, and use $\lor$Intro.
\end{tabular}
}

\subsec{Strategy for Proving Disjunctions}{

\begin{tabular}{@{}lp{13em}}
\fitchprf{\pline[1.]{}}{
\ellipsesline \\
\pline[2.]{Q} \\
\pline[3.]{P \lor Q}[\lori{2}]
} 
&
If we can't prove $\sf P$, we may have to backtrack and try to prove $\sf Q$
instead.  Again, we'll cite the line where we proved $\sf Q$, and use
$\lor$Intro.
\end{tabular}
}


\subsec{Strategy for Using Disjunctions}{

\setlength{\fitchprfwidth}{.9in}

\begin{tabular}{@{}l@{}p{13em}}
\fitchprf{
   \pline[1.]{P \lor Q}
}{
 \pline{}\\
\pline{} \\
\pline{}\\
\pline{} \\
\pline{} \\
\pline{} \\
\ellipsesline \\[.7em]
  \pline[6.]{R}[\hphantom{\lore{1}{2--3}{4--5}}]
} 
  &
Suppose we want to prove R, and we have $\sf P \lor Q$ as a premise,
as the assumption of a subproof, or as something proved before.
We'll try to use it by applying $\lor$Elim.
\end{tabular}

}

\subsec{Strategy for Using Disjunctions}{

\setlength{\fitchprfwidth}{.9in}

\begin{tabular}{@{}l@{}p{13em}}
\fitchprf{
   \pline[1.]{P \lor Q} }{ \subproof{\pline[2.]{P}}{ \ellipsesline \\
\pline[3.]{R}} \subproof{\pline[4.]{Q}}{ \ellipsesline \\
\pline[5.]{R}} \pline[6.]{R}[\lore{1}{2--3}{4--5}] } 
& To prove $\sf R$ using $\lor$Elim from $\sf P \lor Q$, we'll have to
look for two subproofs: one in which we prove $\sf R$ from $\sf P$,
another where we prove $\sf R$ from $\sf Q$. Once we have both, we
can justify $\sf R$ by using $\lor$Elim, citing the line with $\sf P
\lor Q$ and the two subproofs.
\end{tabular}
}

\subsec{Strategy for Proving Negations}{

\begin{tabular}{@{}lp{13em}}
\fitchprf{
   \pline[1.]{}
}{
 \pline{}\\
\pline{} \\
\pline{}\\
\ellipsesline \\[.7em]
  \pline[4.]{\neg P}[\hphantom{\lnoti{2}}]
} 
  &
If we want to prove $\neg\sf P$, we'll have to use  $\neg$Intro.
\end{tabular}

}

\subsec{Strategy for Proving Negations}{

\begin{tabular}{@{}lp{13em}}
\fitchprf{\pline[1.]{}}{
\subproof{\pline[2.]{P}}{
\ellipsesline \\
\pline[3.]{\bot}}
\pline[4.]{\neg P}[\lnoti{2--3}]
} 
& For $\neg$Intro, we have to start a subproof with assumption $\sf
P$; the last line of the subproof has to be $\bot$.  We'll cite the
subproof, and use $\lnot$Intro.
\end{tabular}
}


\subsec{Strategy for Indirect Proof}{

\begin{tabular}{@{}lp{13em}}
\fitchprf{
   \pline[1.]{}
}{
 \pline{}\\
\pline{} \\
\pline{}\\
\ellipsesline \\[.7em]
  \pline[4.]{P}[\hphantom{\lnoti{2}}]
} 
& Sometimes, we want to prove a sentence $\sf P$, but our other
strategies didn't get us far. This is often the case, e.g., when we
want to prove disjunctions but don't have any premises (for example:
$\sf A \lor \neg A$).  Then we can also use $\neg$Intro.
\end{tabular}

}

\subsec{Strategy for Indirect Proof}{

\begin{tabular}{@{}lp{13em}}
\fitchprf{\pline[1.]{}}{
\subproof{\pline[2.]{\neg P}}{
\ellipsesline \\
\pline[3.]{\bot}}
\pline[4.]{P}[\lnoti{2--3}]} 
&
Here, we have to start a subproof with assumption $\sf \neg P$;
the last line of the subproof has to be $\bot$.
We'll cite the subproof, and use $\lnot$Intro.  In the subproof, we now
have an additional assumption (on line 2) to work with.
\end{tabular}

}

\subsec{How to Find a Contradiction}{

\begin{tabular}{@{}lp{13em}}
\fitchprf{\pline[1.]{}}{
\subproof{\pline[2.]{\neg P}}{
\pline{}\\
\ellipsesline \\
\pline[4.]{\bot}}
\pline[5.]{P}[\lnoti{2--3}]
} 
&
Suppose we used the indirect proof strategy, or we're in some other situation
where we're looking for a proof of $\bot$.  What's a good candidate?
\end{tabular}

}

\subsec{How to Find a Contradiction}{

\begin{tabular}{@{}lp{13em}}
\fitchprf{\pline[1.]{}}{
\subproof{\pline[2.]{\neg P}}{
\ellipsesline \\
\pline[3.]{P}\\
\pline[4.]{\bot}[\lfalsei{2}{3}]}
\pline[5.]{P}[\lnoti{2--4}]
} 
& One good candidate is $\sf P$ itself!  This may look weird but
remember that inside the subproof, we have an additional assumption
$\sf \lnot P$ to work with.
\end{tabular}

}

\subsec{How to Use a Negated Sentence}{

\begin{tabular}{@{}lp{13em}}
\fitchprf{
   \pline[1.]{\neg P}
}{
 \pline{}\\
\pline{} \\
\pline{}\\
\ellipsesline \\[.7em]
  \pline[4.]{R}[\hphantom{\lnoti{2}}]
} 
  &
Suppose $\sf\neg P$ is one of your premises, or something else you've
already proved, and the sentence you're shooting for doesn't contain
$\sf \neg P$. How are you going to make use of  $\sf\neg P$?
\end{tabular}

}

\subsec{How to Use a Negated Sentence}{

\begin{tabular}{@{}lp{13em}}
\fitchprf{\pline[1.]{\neg P}}{
\subproof{\pline[2.]{\neg R}}{
\ellipsesline \\
\pline[3.]{P}\\
\pline[4.]{\bot}[\lfalsei{1}{3}]}
\pline[5.]{R}[\lnoti{2--4}]
} 
&
The only way to make use of $\sf\neg P$ in such a situation is as part
of a contradiction. For instance, apply the indirect proof strategy,
and then look for $\sf P$ inside the subproof.
\end{tabular}

}

\subsec{Summary}{

\bits
\item Trying to prove $\sf P \land Q$? Look for a proof of $\sf P$,
  then for a proof of $\sf Q$ ($\land$Intro).
\item Split up conjunctions ($\land$Elim) whenever you can.
\item Next, use the disjunctions you have ($\lor$Elim).
\item Trying to prove $\sf\neg P$? Look for a subproof of $\bot$ from
  $\sf P$ ($\neg$Intro).
\item Trying to prove $\sf P \lor Q$?  Look for a proof of $\sf P$, if
  that doesn't work, look for a proof of $\sf Q$ ($\lor$Intro). If
  that doesn't work either, use indirect proof.  (Note: $\lor$Intro
  rarely works unless you're in a subproof.)
\item Apply the strategies at each step in the construction of your
  proof. You may not be able to prove $\sf P \lor Q$ using $\lor$Intro
  at one point, but then you might find yourself in a subproof where
  it is possible.
\eit
}

\subsec{Summary}{
\bits
\item It's good to know where you're going: So use the strategies
  based on $\land$Intro, $\lor$Elim, $\neg$Intro before those based on
  $\land$Elim and $\lor$Intro.
\item Picking a disjunct to prove for $\lor$Intro or a candidate for a
  contradiction doesn't always work. If it doesn't, you have to
  backtrack and try something else.
\item If you're looking for a proof of $\sf P$ and you're not getting
  anywhere, make sure you're on the right track: is $\sf P$ even a
  tautological consequence of the premises and assumptions? If not,
  you'll have to backtrack.
\end{itemize}

}

\section{Examples}

\setlength{\fitchprfwidth}{1.5in}

\subsec{An Example}{

\begin{tabular}{@{}l@{}p{13em}}
\fitchprf{
   \pline[1.]{(A \land B) \lor (A \land C)}
}{
 \pline{}\\
\pline{} \\
\pline{}\\
\ellipsesline \\[.7em]
  \pline{A \land (B \lor C)}[\hphantom{\lore{1}{2--5}{6--9}}]
} 
  &
First, we use the disjunction on line 1, and set up the subproofs we
need for $\lor$Elim.
\end{tabular}

}
\subsec{An Example}{

\begin{tabular}{@{}l@{}p{13em}}
\fitchprf{
   \pline[1.]{(A \land B) \lor (A \land C)}
}{
\subproof{\pline[2.]{A \land B}}{
\ellipsesline \\
  \pline[6.]{A \land (B \lor C)}}\\[.5ex]
\subproof{\pline[7.]{A \land C}}{
\ellipsesline \\
  \pline[11.]{A \land (B \lor C)}}
  \pline{A \land (B \lor C)}[\lore{1}{2--6}{7--11}]
} 
  &
In the first subproof, we now set up the sub goals for proving line 6
using $\land$Intro.
\end{tabular}

}
\subsec{An Example}{

\begin{tabular}{@{}l@{}p{13em}}
\fitchprf{
   \pline[1.]{(A \land B) \lor (A \land C)}
}{
\subproof{\pline[2.]{A \land B}}{
\ellipsesline \\
  \pline[4.]{A}\\
  \pline[5.]{B \lor C}\\
  \pline[6.]{A \land (B \lor C)}[\landi{4}{5}]}\\[.5ex]
\subproof{\pline[7.]{A \land C}}{
  \ellipsesline \\
  \pline[11.]{A \land (B \lor C)}}
\pline{A \land (B \lor C)}[\lore{1}{2--6}{7--11}]
} 
  &
We immediately see that we get line 4 from 2 by $\land$Elim; let's
apply the strategy for proving disjunctions to line 5: look for a
proof of $\sf B$.
\end{tabular}


}
\subsec{An Example}{

\begin{tabular}{@{}l@{}p{13em}}
\fitchprf{
   \pline[1.]{(A \land B) \lor (A \land C)}
}{
\subproof{\pline[2.]{A \land B}}{
  \pline[3.]{B}\\
  \pline[4.]{A}[\lande{2}]\\
  \pline[5.]{B \lor C}[\lori{3}]\\
  \pline[6.]{A \land (B \lor C)}[\landi{4}{5}]}\\[.5ex]
\subproof{\pline[7.]{A \land C}}{
\ellipsesline \\
  \pline[11.]{A \land (B \lor C)}}
  \pline[12.]{A \land (B \lor C)}[\lore{1}{2--6}{7--11}]
} 
  &
Like line 4 before, we get line 3 from 2 by $\land$Elim. That's it for
the first subproof. The second subproof is exactly the same.
\end{tabular}

}
\subsec{An Example}{

\begin{tabular}{@{}l@{}p{13em}}
\fitchprf{
   \pline[1.]{(A \land B) \lor (A \land C)}
}{
\subproof{\pline[2.]{A \land B}}{
  \pline[3.]{B}[\lande{2}]\\
  \pline[4.]{A}[\lande{2}]\\
  \pline[5.]{B \lor C}[\lori{3}]\\
  \pline[6.]{A \land (B \lor C)}[\landi{4}{5}]}\\[.5ex]
\subproof{\pline[7.]{A \land C}}{
  \pline[8.]{C}[\lande{7}]\\
  \pline[9.]{A}[\lande{7}]\\
  \pline[10.]{B \lor C}[\lori{8}]\\
  \pline[11.]{A \land (B \lor C)}[\landi{9}{10}]}
  \pline[12.]{A \land (B \lor C)}[\lore{1}{2--6}{7--11}]
} 
  &
\end{tabular}

}
\subsec{Example Indirect Proof}{

\begin{tabular}{@{}lp{13em}}
\fitchprf{\pline[1.]{}}{
\subproof{\pline[2.]{\neg (P \lor ~P)}}{
\ellipsesline\\
\pline[8.]{\bot}[\lfalsei{2}{7}]}
\pline[9.]{P \lor \neg P}[\lnoti{2--8}]
} 
&
Suppose we want to prove $\sf P \lor \neg P$.  A little reflection
shows that this can only be done by indirect proof.  So we set up a subproof
for use with $\neg$Intro.
\end{tabular}

}
\subsec{Example Indirect Proof}{

\begin{tabular}{@{}lp{13em}}
\fitchprf{\pline[1.]{}}{
\subproof{\pline[2.]{\neg (P \lor ~P)}}{
\ellipsesline\\
\pline[7.]{P \lor \neg P}[?]\\
\pline[8.]{\bot}[\lfalsei{2}{7}]}
\pline[9.]{P \lor \neg P}[\lnoti{2--8}]
} 
&
Our first candidate for a contradiction is the assumption of the
subproof.  How should we get $\sf P \lor neg P$ inside that
subproof?
\end{tabular}

}
\subsec{Example Indirect Proof}{

\begin{tabular}{@{}lp{13em}}
\fitchprf{\pline[1.]{}}{
\subproof{\pline[2.]{\neg (P \lor ~P)}}{
\subproof{\pline[3.]{P}}{
\ellipsesline\\
\pline[5.]{\bot}[?]
}
\pline[6.]{\neg P}[\lnoti{3--5}]\\
\pline[7.]{P \lor \neg P}[\lori{6}]\\
\pline[8.]{\bot}[\lfalsei{2}{7}]}
\pline[9.]{P \lor \neg P}[\lnoti{2--8}]
} 
&
Inside the subproof, we \emph{can} get $\sf P \lor \neg P$ by
$\lor$Intro. So let's try to prove $\sf \neg P$; it's a negated
sentence, so we'll use $\neg$Intro.
\end{tabular}

}
\subsec{Example Indirect Proof}{

\begin{tabular}{@{}lp{13em}}
\fitchprf{\pline[1.]{}}{
\subproof{\pline[2.]{\neg (P \lor ~P)}}{
\subproof{\pline[3.]{P}}{
\pline[4.]{P \lor \neg P}[\lori{3}]\\
\pline[5.]{\bot}[\lfalsei{2}{4}]
}
\pline[6.]{\neg P}[\lnoti{3--5}]\\
\pline[7.]{P \lor \neg P}[\lori{6}]\\
\pline[8.]{\bot}[\lfalsei{2}{7}]}
\pline[9.]{P \lor \neg P}[\lnoti{2--8}]
} 
&
Here, the assumption of the subproof, $\sf P$, is not
really a good candidate for a negation. But we can use the negated
sentence on line 2 again, and it's easy to get in this situation.
\end{tabular}

}
\subsec{Another Example}{

\begin{tabular}{@{}lp{13em}}
\fitchprf{\pline[1.]{}}{
\ellipsesline\\
\pline[13.]{\neg \neg (P \lor Q) \lor \neg P}[\hphantom{\lnoti{2--12}}]
} 
  &
Here, we're proving that $\sf\neg \neg (P \lor Q) \lor \neg P$ is a
tautology---that's why there are no premises.  Since neither P nor
$\sf\neg\neg( P \lor Q)$ is a tautology, we won't be able to
prove either of these, so the strategy for proving the conclusion
using $\lor$Intro won't work. We have to use indirect proof.
\end{tabular}
}
\subsec{Another Example}{

\begin{tabular}{@{}lp{13em}}
\fitchprf{\pline[1.]{}}{
\subproof{\pline[2.]{\neg(\neg \neg (P \lor Q) \lor \neg P)}}{
\ellipsesline\\
\pline[12.]{\bot}[?]
}
\pline[13.]{\neg \neg (P \lor  Q) \lor \neg P}[\lnoti{2--12}]
} 
  &
To use $\neg$Intro as the last step we need a subproof that proves
a contradiction from the negation of the goal sentence.  What's a 
good candidate for sentences from which we get $\bot$ using $\bot$Intro?
\end{tabular}
}
\subsec{Another Example}{

\begin{tabular}{@{}lp{13em}}
\fitchprf{\pline[1.]{}}{
\subproof{\pline[2.]{\neg(\neg\neg (P \lor Q) \lor \neg P)}}{
\ellipsesline\\
\pline[11.]{\neg\neg( P \lor  Q) \lor \neg P}[?] \\
\pline[12.]{\bot}[\lfalsei{2}{11}]
}
\pline[13.]{\neg \neg (P \lor Q) \lor \neg P}[\lnoti{2--12}]
} 
  &
The only plausible candidate is the sentence $\sf \neg\neg (P \lor Q)
\lor \neg P$ itself, since no other negated sentence is available.  So
let's try to prove that inside the subproof.  We'll get $\bot$ by
$\bot$Intro from lines 2 and 11. So how do we get line 11?
\end{tabular}
}
\subsec{Another Example}{

\begin{tabular}{@{}lp{13em}}
\fitchprf{\pline[1.]{}}{
\subproof{\pline[2.]{\neg(\neg\neg( P \lor Q) \lor \neg P)}}{
\ellipsesline\\
\pline[10.]{\neg\neg (P \lor  Q)}[?] \\
\pline[11.]{\neg \neg (P \lor Q) \lor \neg P}[\lori{10}]\\
\pline[12.]{\bot}[\lfalsei{2}{11}]
}
\pline[13.]{\neg \neg (P \lor Q) \lor \neg P}[\lnoti{2--12}]
} 
  &
Line 11 is a disjunction, so the strategy for $\lor$Intro applies.  We
have two choices: try to prove $\sf\neg\neg (P \lor Q)$, or prove $\sf \neg P$.
Either would work, but lets try to prove $\sf \neg\neg (P \lor Q)$.
\end{tabular}
}
\subsec{Another Example}{

\begin{tabular}{@{}lp{13em}}
\fitchprf{\pline[1.]{}}{
\subproof{\pline[2.]{\neg(\neg\neg( P \lor Q) \lor \neg P)}}{
\subproof{\pline[3.]{\neg(P \lor Q)}}{
\ellipsesline\\
\pline[9.]{\bot}[?]}
\pline[10.]{\neg\neg (P \lor  Q)}[\lnoti{3--9}]\\
\pline[11.]{\neg \neg (P \lor Q) \lor \neg P}[\lori{10}]\\
\pline[12.]{\bot}[\lfalsei{2}{11}]
}
\pline[13.]{\neg \neg (P \lor Q) \lor \neg P}[\lnoti{2--12}]
} 
  &
Since $\sf\neg\neg(P \lor Q)$ starts with a $\neg$, we'll apply the
strategy for proving negations: start a subproof with $\sf\neg(P \lor
Q)$ as the assumption, and try to prove $\bot$.  We justify line 10
using $\neg$Intro.
\end{tabular}
}
\subsec{Another Example}{

\begin{tabular}{@{}lp{13em}}
\fitchprf{\pline[1.]{}}{
\subproof{\pline[2.]{\neg(\neg\neg( P \lor Q) \lor \neg P)}}{
\subproof{\pline[3.]{\neg(P \lor Q)}}{
\ellipsesline\\
\pline[8.]{P \lor Q}[?]\\
\pline[9.]{\bot}[\lfalsei{3}{8}]}
\pline[10.]{\neg\neg (P \lor  Q)}[\lnoti{3--9}]\\
\pline[11.]{\neg \neg (P \lor Q) \lor \neg P}[\lori{10}]\\
\pline[12.]{\bot}[\lfalsei{2}{11}]
}
\pline[13.]{\neg \neg (P \lor Q) \lor \neg P}[\lnoti{2--12}]
} 
  &
Now we need a candidate for a contradiction.  There are two negated
lines in the avaliable assumptions. Let's pick line 3: try to prove
$\sf P \lor Q$.
\end{tabular}
}
\subsec{Another Example}{

\begin{tabular}{@{}lp{13em}}
\fitchprf{\pline[1.]{}}{
\subproof{\pline[2.]{\neg(\neg\neg( P \lor Q) \lor \neg P)}}{
\subproof{\pline[3.]{\neg(P \lor Q)}}{
\ellipsesline\\
\pline[7.]{P}[?]\\
\pline[8.]{P \lor Q}[\lori{7}]\\
\pline[9.]{\bot}[\lfalsei{3}{8}]}
\pline[10.]{\neg\neg (P \lor  Q)}[\lnoti{3--9}]\\
\pline[11.]{\neg \neg (P \lor Q) \lor \neg P}[\lori{10}] \\
\pline[12.]{\bot}[\lfalsei{2}{11}]
}
\pline[13.]{\neg \neg (P \lor Q) \lor \neg P}[\lnoti{2--12}]
} 
  &
To get $\sf P \lor Q$, we'll use $\lor$Intro.  For this we have to
prove one of the disjuncts, say, $\sf P$.
\end{tabular}
}
\subsec{Another Example}{


\begin{tabular}{@{}lp{13em}}
\fitchprf{\pline[1.]{}}{
\subproof{\pline[2.]{\neg(\neg\neg( P \lor Q) \lor \neg P)}}{
\subproof{\pline[3.]{\neg(P \lor Q)}}{
\subproof{\pline[4.]{\neg P}}{
\ellipsesline\\
\pline[6.]{\bot}[?]
}
\pline[7.]{P}[\lnoti{4--6}]\\
\pline[8.]{P \lor Q}[\lori{7}]\\
\pline[9.]{\bot}[\lfalsei{3}{8}]
}
\pline[10.]{\neg\neg (P \lor  Q)}[\lnoti{3--9}]\\
\pline[11.]{\neg \neg (P \lor Q) \lor \neg P}[\lori{10}]\\
\pline[12.]{\bot}[\lfalsei{2}{11}]
}
\pline[13.]{\neg \neg (P \lor Q) \lor \neg P}[\lnoti{2--12}]
} 
& $\sf P$ is an atomic sentence, so no introduction rule can give us
$\sf P$.  We'll have to do indirect proof: start a subproof using $\sf
\neg P$ a the assumption, and prove another contradiction.
\end{tabular}
}
\subsec{Another Example}{

\begin{tabular}{@{}lp{13em}}
\fitchprf{\pline[1.]{}}{
\subproof{\pline[2.]{\neg(\neg\neg( P \lor Q) \lor \neg P)}}{
\subproof{\pline[3.]{\neg(P \lor Q)}}{
\subproof{\pline[4.]{\neg P}}{
\pline[5.]{\neg \neg (P \lor Q) \lor \neg P}[\lori{4}] \\
\pline[6.]{\bot}[\lfalsei{2}{5}]
}
\pline[7.]{P}[\lnoti{4--6}] \\
\pline[8.]{P \lor Q}[\lori{7}] \\
\pline[9.]{\bot}[\lfalsei{3}{8}]}
\pline[10.]{\neg\neg (P \lor  Q)}[\lnoti{3--9}]\\
\pline[11.]{\neg \neg (P \lor Q) \lor \neg P}[\lori{10}]\\
\pline[12.]{\bot}[\lfalsei{2}{11}]
}
\pline[13.]{\neg \neg (P \lor Q) \lor \neg P}[\lnoti{2--12}]
} 
  &
How are we getting the contradiction now?  Note how since we have
$\sf\neg P$ as an assumption, we can immediately get $\sf\neg\neg(P
\lor Q) \lor \neg P$ by $\lor$Intro.  Everything's now in place.
\end{tabular}
}

\end{document}


\end{document}






\end{document}
