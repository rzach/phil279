\documentclass[bigger]{beamer}

\usetheme{Frankfurt}
\useoutertheme[subsection=false]{miniframes}
\setbeamercolor{structure}{parent=palette quaternary,fg=uofcred}
\setbeamercolor{section in toc}{fg=black,bg=white}
\setbeamercolor{alerted text}{fg=uofcred}
\setbeamercolor*{palette quaternary}{fg=white,bg=black}
\setbeamercolor*{palette secondary}{fg=black,bg=uofcyellow}
\setbeamercolor*{palette tertiary}{bg=uofcred,fg=white}
\setbeamercolor*{palette primary}{fg=white,bg=black}
\setbeamercolor*{sidebar}{fg=uofcred,bg=black!15!white}
\setbeamercolor*{palette sidebar primary}{fg=uofcred!10!black}
\setbeamercolor*{palette sidebar secondary}{fg=white}
\setbeamercolor*{palette sidebar tertiary}{fg=uofcred!50!black}
\setbeamercolor*{palette sidebar quaternary}{fg=black!10!white}

\setbeamercolor{titlelike}{parent=palette primary}
\setbeamercolor{frametitle}{fg=white,bg=black}
\setbeamercolor{frametitle right}{bg=gray!60!white}
\let\emph\alert
\beamertemplatenavigationsymbolsvertical
\def\bits{\begin{itemize}[<+->]}
\def\bens{\begin{enumerate}[<+->]}
\def\items#1{\item<#1>}
\def\uncovers#1{\uncover<#1>}
\def\pauses{\pause}

 % change to header-handout for handouts

% ====================
\title[Lecture 14]{Logic I F13 Lecture 14}
\date{October 29, 2013}
% ====================

% lecture header include

\usepackage{lplfitch,amsmath}
\usepackage{qtree,hyperref}
\usepackage{pgf,amssymb}

\author{Richard Zach}
\institute{Department of Philosophy\\
University of Calgary\\
  \href{http://ucalgary.ca/rzach/279}{ucalgary.ca/rzach/279}}

\definecolor{uofcred}{RGB}{227,39,38}
\definecolor{uofcyellow}{RGB}{255,210,0}

\DeclareSymbolFont{symbolsC}{U}{txsyc}{m}{n}
\DeclareMathSymbol{\strictif}{\mathrel}{symbolsC}{74}
\DeclareMathSymbol{\boxright}{\mathrel}{symbolsC}{128}

\let\IFF\Leftrightarrow
\let\iff\leftrightarrow
\let\impl\to

\def\T{{\color{green}\begin{colormixin}{25!black}\text{T}\end{colormixin}}}
\def\F{{\color{red}\begin{colormixin}{25!black}\text{F}\end{colormixin}}}

\long\def\subsec#1#2{\subsection{#1}\frame{\frametitle{#1} #2}}
\def\bit{\begin{itemize}[<1->]}
\def\eit{\end{itemize}}
\def\ben{\begin{enumerate}[<1->]}
\def\een{\end{enumerate}}

\makeatletter\let\@makefnmark\noindent\makeatother

%\setbeamercolor{footnote}{fg=black!70}

\def\foot#1{\footnotetext{\color{black!70}#1}}
\def\deemph#1{{\color{black!70}#1}}
\let\phi\varphi

\setbeamertemplate{theorems}[numbered]
%\useinnertheme{circles}
\setbeamertemplate{itemize subitems}[triangle]

\renewcommand{\beamertemplatetransparentcovereddynamic}{
  \beamersetuncovermixins
  {\opaqueness<1>{50}\opaqueness<2>{30}\opaqueness<3>{15}\opaqueness<4->{5}}%
  {\opaqueness<1>{50}\opaqueness<2>{30}\opaqueness<3>{15}\opaqueness<4->{5}}}
                                                                                
%\beamertemplatetransparentcovereddynamic

\defbeamertemplate*{footline}{my theme}
{%
  \leavevmode%
  \hbox{\begin{beamercolorbox}[wd=.5\paperwidth,ht=2.5ex,dp=1.125ex,leftskip=.3cm,rightskip=.3cm]{author in head/foot}%
    \insertframenumber/\inserttotalframenumber \hfil \usebeamerfont{author in head/foot}\insertshortauthor
  \end{beamercolorbox}%
  \begin{beamercolorbox}[wd=.5\paperwidth,ht=2.5ex,dp=1.125ex,leftskip=.3cm,rightskip=.3cm plus1fil]{title in head/foot}%
    \usebeamerfont{title in head/foot}Logic I F13---\insertshorttitle---\insertdate
  \end{beamercolorbox}}%
  \vskip0pt%
}

\begin{document}
\setlength{\fitchargwidth}{7em}
\setlength{\fitchprfwidth}{7em}

\frame{\frametitle{\insertshorttitle\ (\insertdate)}
\tableofcontents[hidesubsections]
}










\setlength{\fitchprfwidth}{5em}

\section{Review}

\subsec{First-order Logical Notions}{

\bit
\item An argument is \emph{first-order valid}, and its conclusion is a \emph{first-order consequence} of its premises,
if there is no \emph{first-order structure} in which the premises are true and the conclusion is false
\item A sentence is a \emph{first-order validity} if it is true in every first-order stucture
\item Two sentences are \emph{first-order equivalent} if there is no first-order structure in which they have different truth values.
\item A (set of) sentence(s) is \emph{first-order satisfiable} if it is (all) true in at least on first-order structure
\item NB: There is \emph{no mechanical test} for any of these (The Church-Turing Undecidability Theorem), by contrast to the truth-table case!
\eit

}


\subsec{Skorupski Revisited}{

\fitchprf{\pline[1.]{\text{If $\sf x$ is morally wrong, then}}\\
\pline{\qquad \text{$\sf A$ is blameworthy for freely doing $\sf x$}}\\
  \pline[2.]{\text{If $\sf x$ is
  rationally optimal, then }}\\
\pline{\qquad\text{A is not blameworthy for freely doing x.}}}
{\pline[3.]{\text{If $\sf x$ is morally wrong, then $\sf x$ is not rationally
      optimal.}}}

\bigskip

\fitchprf{\pline[1.]{\forall x(Wrong(x) \lif BlameFreelyDoing(a, x))}\\
  \pline[2.]{\forall x(ROpt(x) \lif \lnot BlameFreelyDoing(a, x))}}
{\pline[3.]{\forall x(Wrong(x) \lif \lnot ROpt(x))}}
}

\subsec{Skorupski Revisited}{

\fitchprf{\pline[1.]{\text{If $\sf x$ is morally wrong and $\sf A$ freely does $\sf x$,
    then}}\\
\pline{\qquad\text{$\sf A$ is blameworthy for doing $\sf x$.}}\\ 
  \pline[2.]{\text{If $\sf x$ is
  rationally optimal, then}}\\
\pline{\qquad\text{there is no reason to blame $\sf A$ for freely doing
  $\sf x$.}}}
{\pline[3.]{\text{If $\sf x$ is morally wrong, then $\sf x$ is not rationally
      optimal.}}}

\bigskip

\fitchprf{\pline[1.]{\forall x((Wrong(x) \land DoesF(a, x)) \lif Blame(a, x))}\\
  \pline[2.]{\forall x((ROpt(x) \land DoesF(a, x)) \lif \lnot Blame(a, x))}}
{\pline[3.]{\forall x(Wrong(x) \lif \lnot ROpt(x))}}

}

\subsec{Skorupski Revisited}{

\fitchprf{\pline[1.]{\forall x((Wrong(x) \land DoesF(a, x)) \lif Blame(a, x))}\\
  \pline[2.]{\forall x((ROpt(x) \land DoesF(a, x)) \lif \lnot Blame(a, x))}}
{\pline[3.]{\forall x(Wrong(x) \lif \lnot ROpt(x))}}

\bigskip

\fitchprf{\pline[1.]{\forall x((Cube(x) \land LeftOf(a, x)) \lif SameRow(a, x))}\\
  \pline[2.]{\forall x((Small(x) \land LeftOf(a, x)) \lif \lnot SameRow(a, x))}}
{\pline[3.]{\forall x(Cube(x) \lif \lnot Small(x))}}
}

\section{Relations Between Logical Notions}

\subsec{Relations Between Different Interpretations}{

\bit
\item Every TW-world is also a possible world
\item Every possible world yields a FO structure
\item Every FO structure determines a truth value for all
  sub\textit{sentences}, i.e., a truth-value assignment for its
  truth-functional form, and the sentence has the same truth value in
  the FO structure as its truth-functional form does in the
  corresponding truth-value assignment \eit }

\subsec{Tautology $\Rightarrow$ FO Validity}{

\bit
\item If there were a FO structure which makes a sentence S false, the corresponding truth-value assignment would make its truth-functional form false
\item If the truth-functional form of S is true in \emph{all} truth-value assignments, S must be true in all FO structures
\item If S is a tautology, it is also a FO validity
\item But not vice versa, e.g., $\sf a = a$, $\sf \forall x(Cube(x) \lor \lnot Cube(x))$
\eit

}

\subsec{FO Validity $\Rightarrow$ Logical Truth}{

\bit
\item If there were a possible world which makes S false, this would also
yield a FO structure which makes S false
\item If S is true in all FO structures, it is true in all possible worlds
\item If S is a FO validity, it is also a logical truth
\item But not vice versa, e.g., $\sf SameSize(a, a)$, $\sf \forall
  x(Larger(x, a) \to Smaller(a, x))$ \eit

}

\subsec{Relations Between Logical Notions}{

\bit
\item tautology $\Rightarrow$ FO validity $\Rightarrow$ logical truth $\Rightarrow$ TW necessity
\item tautologically valid $\Rightarrow$ FO valid $\Rightarrow$\\ logically valid $\Rightarrow$ TW valid
\item TT contradictory $\Rightarrow$ FO unsatisfiable $\Rightarrow$\\ logically contradictory $\Rightarrow$ TW contradictory
\item TW-possible $\Rightarrow$ logically possible $\Rightarrow$\\ FO satisfiable $\Rightarrow$ TT possible
\eit
}




\section{FO Equivalence}

\subsec{FO Equivalence of WFFs}{

\bit
\item Two sentences $\sf A$ and $\sf B$ are \emph{FO equivalent} iff in
  every FO structure they have the same truth value.
\item Two wffs $A(x)$ and $B(x)$ are \emph{FO equivalent}
iff in every FO structure, $A(x)$ is satisfied by the same objects
as $B(x)$, i.e.,
\item Two wffs $A(x)$ and $B(x)$ are \emph{FO equivalent}
iff in every FO structure, $\sf A(x)$ and $\sf B(x)$ have the same extension
\item More generally: $\sf A(x_1, \dots, x_n)$ is \emph{FO equivalent}
  to $\sf B(x_1, \dots, x_n)$ iff they have the same extension in every
  FO structure.  \eit }

\subsec{Important FO Equivalences}{

\let\IFF\Leftrightarrow

\[
\begin{array}{@{}rcl@{}}
\lnot\forall x\,A(x) & \IFF & \exists x\, \lnot A(x)\\
\lnot\exists x\,A(x) & \IFF & \forall x\, \lnot A(x)\\
\forall x(A(x) \land B(x))& \IFF & \forall x\, A(x) \land \forall x\, B(x) \\
\exists x(A(x) \lor B(x))& \IFF & \exists x\, A(x) \lor \exists x\, B(x) \\
\forall x(A(x) \lor B)& \IFF & \forall x\, A(x) \lor B \\
\multicolumn{3}{c}{\text{(provided $x$ not free in B)}} \\
\exists x(A(x) \land B)& \IFF & \exists x\, A(x) \land B \\
\multicolumn{3}{c}{\text{(provided $x$ not free in B)}} \\
\forall x\,A & \IFF & A\\
\multicolumn{3}{c}{\text{(provided $x$ not free in A)}} \\
\exists x\,A & \IFF & \exists x\, A \\
\multicolumn{3}{c}{\text{(provided $x$ not free in A)}} \\
\forall x\,A(x) & \IFF & \forall y\, A(y)\\
\exists x\,A(x) & \IFF & \exists y\, A(y)\\
\end{array}\]
}

\subsec{Proof of a FO Equivalence}{

\[
\forall x(A(x) \land B(x)) \IFF \forall x\, A(x) \land \forall x\, B(x)
\]

\bit
\item First suppose that $\forall x(A(x) \land B(x))$ is true in a FO structure.
\item Then every object in the domain satisfies $A(x) \land B(x)$.
\item So every object satisfies $A(x)$, i.e., $\forall x\, A(x)$ is true.
\item Ditto for $B(x)$
\item Now suppose an structure makes $\forall x\, A(x) \land \forall x\, B(x)$ true
\item Then it makes $\forall x\, A(x)$ true, i.e., every object satisfies $A(x)$
\item Ditto for $B(x)$, so every object satisfies $B(x)$.
\item So every object satisfies $A(x) \land B(x)$, i.e., $\forall x(A(x) \land B(x))$ is true
\eit
}

\subsec{Proof of a FO Non-Equivalence}{

\[
\exists x(A(x) \land B(x)) \IFF  \exists x\, A(x) \land \exists x\, B(x)
\]

\bit
\item The $\Rightarrow$ part holds!
\item Here's a FO interpretation which makes the RHS true but the LHS false:
\bit
\item Domain: $\{1, 2\}$
\item $A(x)$ \dots $\{1\}$
\item $B(x)$ \dots $\{2\}$
\eit
\eit
}


\section{Multiple Uses of a Single Quantifier}

\subsec{Wffs Expressing Relations}{

\bit
\item A wff A(x) with one free variable expresses a \emph{property}
\item A wff B(x, y) with two free variables expresses a \emph{relation}
\item $\sf \forall x\forall y\, B(x, y)$ is a sentence; it's true iff\\
any pair of objects $\alpha$, $\beta$ stand in the relation expressed by $B(x, y)$
\item $\sf \exists x\exists y\, B(x, y)$ is a sentence; it's true iff\\
at least one pair of objects $\alpha$, $\beta$ stand in the relation expressed by $B(x, y)$
\eit
}

\subsec{Multiple Uses of Single Quantifier: Examples}{

\bit
\item SameSize(x, y) \dots x and y are the same size
\item $\sf \forall x\forall y\, SameSize(x, y)$ \dots every pair $\langle \alpha,\beta\rangle$ stands in the relation ``is the same size as''
\item in other words: any pair of objects are the same size
\item NB: ``every pair'' includes pairs $\langle\alpha, \alpha\rangle$, i.e.,
\item $\sf \forall x\forall y\, A(x, y)$ only true if all pairs $\langle \alpha, \alpha\rangle$ satisfy $\sf A(x, y)$
\eit

}

\subsec{Multiple Uses of Single Quantifier: Examples}{

\bit
\item Larger(x, y) \dots x is larger than y
\item $\sf \exists x\exists y\, Larger(x, y)$ \dots at least one pair $\langle \alpha,\beta\rangle$ stands in the relation ``is larger than''
\item in other words: at least one objects is larger than at least one object
\item NB: includes pairs $\langle\alpha, \alpha\rangle$, i.e.,
\item $\sf \exists x\exists y\, A(x, y)$ is already true even if a single pair $\langle \alpha, \alpha\rangle$ satisfies $\sf A(x, y)$
\eit 

}

\subsec{Different Variables $\neq$ Different Objects}{

\bits
\item Does $\sf\forall x\forall y\, Adjoins(x, y)$ mean ``any \emph{two} objects adjoin each other''?
\item It does not!
\item \dots true iff any pair $\langle\alpha, \beta\rangle$ satisfies $\sf Adjoins(x, y)$
\item \emph{including pairs where $\alpha = \beta$}
\item Hence, $\sf\forall x\forall y\, Adjoins(x, y)$ is \emph{always false} since no block adjoins itself!
\item Correct: $\sf \forall x\forall y(x \neq y \to Adjoins(x, y))$
\eit

}

\subsec{Different Variables $\neq$ Different Objects}{

\bits
\item Does $\sf\exists x\exists y\, SameRow(x, y)$ mean ``there are \emph{two} objects in the same row''?
\item It does not!
\item \dots true iff at least one pair $\langle\alpha, \beta\rangle$ satisfies $SameRow(x, y)$
\item \emph{including pairs where $\alpha = \beta$}
\item Hence, $\sf\exists x\exists y\, SameRow(x, y)$ is \emph{always true} since any block is in the same row as itself!
\item Correct: $\sf \exists x\exists y(x \neq y \land SameRow(x, y))$
\eit
}

\subsec{Restricted Quantification: All}{

\bit 
\item All cubes are left of all tetrahedra.
\item Any pair $\langle \alpha, \beta\rangle$ where $\alpha$ is a cube and $\beta$ is a tetrahedron is so that $\alpha$ is left of $\beta$.\pause
\[\sf
\forall x\forall y((Cube(x) \land Tet(y)) \to LeftOf(x, y))
\]
\eit
}

\subsec{Restricted Quantification: Some}{

\bit
\item A cube is left of a tetrahedron.
\item At least one pair $\langle \alpha, \beta\rangle$ where $\alpha$ is a cube and $\beta$ is a tetrahedron is so that x is left of y.\pause
\[\sf
\exists x\exists y(Cube(x) \land Tet(y) \land LeftOf(x, y))
\]
\eit
}

\subsec{Restricted Quantification: No}{

\bit 
\item No two cubes are the same size.
\item Any pair $\langle \alpha, \beta\rangle$ where $\alpha$ and $\beta$ are different cubes is so that $\alpha$ is not the same size as $\beta$.\pause
\[\sf
\forall x\forall y((Cube(x) \land Cube(y) \land x \neq y) \to \lnot SameSize(x, y))
\]
\item There is no pair $\langle \alpha, \beta\rangle$ where $\alpha$ and $\beta$ are different cubes of the same size.\pause
\[\sf
\lnot \exists x\exists y(Cube(x) \land Cube(y) \land x \neq y \land SameSize(x, y)
\]
\eit
}

\end{document} 




