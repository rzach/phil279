\documentclass[bigger]{beamer}

\usetheme{Frankfurt}
\useoutertheme[subsection=false]{miniframes}
\setbeamercolor{structure}{parent=palette quaternary,fg=uofcred}
\setbeamercolor{section in toc}{fg=black,bg=white}
\setbeamercolor{alerted text}{fg=uofcred}
\setbeamercolor*{palette quaternary}{fg=white,bg=black}
\setbeamercolor*{palette secondary}{fg=black,bg=uofcyellow}
\setbeamercolor*{palette tertiary}{bg=uofcred,fg=white}
\setbeamercolor*{palette primary}{fg=white,bg=black}
\setbeamercolor*{sidebar}{fg=uofcred,bg=black!15!white}
\setbeamercolor*{palette sidebar primary}{fg=uofcred!10!black}
\setbeamercolor*{palette sidebar secondary}{fg=white}
\setbeamercolor*{palette sidebar tertiary}{fg=uofcred!50!black}
\setbeamercolor*{palette sidebar quaternary}{fg=black!10!white}

\setbeamercolor{titlelike}{parent=palette primary}
\setbeamercolor{frametitle}{fg=white,bg=black}
\setbeamercolor{frametitle right}{bg=gray!60!white}
\let\emph\alert
\beamertemplatenavigationsymbolsvertical
\def\bits{\begin{itemize}[<+->]}
\def\bens{\begin{enumerate}[<+->]}
\def\items#1{\item<#1>}
\def\uncovers#1{\uncover<#1>}
\def\pauses{\pause}

 % change to header-handout for handouts

% ====================
\title[Lecture 7]{Logic I F13 Lecture 7}
\date{October 1, 2013}
% ====================

% lecture header include

\usepackage{lplfitch,amsmath}
\usepackage{qtree,hyperref}
\usepackage{pgf,amssymb}

\author{Richard Zach}
\institute{Department of Philosophy\\
University of Calgary\\
  \href{http://ucalgary.ca/rzach/279}{ucalgary.ca/rzach/279}}

\definecolor{uofcred}{RGB}{227,39,38}
\definecolor{uofcyellow}{RGB}{255,210,0}

\DeclareSymbolFont{symbolsC}{U}{txsyc}{m}{n}
\DeclareMathSymbol{\strictif}{\mathrel}{symbolsC}{74}
\DeclareMathSymbol{\boxright}{\mathrel}{symbolsC}{128}

\let\IFF\Leftrightarrow
\let\iff\leftrightarrow
\let\impl\to

\def\T{{\color{green}\begin{colormixin}{25!black}\text{T}\end{colormixin}}}
\def\F{{\color{red}\begin{colormixin}{25!black}\text{F}\end{colormixin}}}

\long\def\subsec#1#2{\subsection{#1}\frame{\frametitle{#1} #2}}
\def\bit{\begin{itemize}[<1->]}
\def\eit{\end{itemize}}
\def\ben{\begin{enumerate}[<1->]}
\def\een{\end{enumerate}}

\makeatletter\let\@makefnmark\noindent\makeatother

%\setbeamercolor{footnote}{fg=black!70}

\def\foot#1{\footnotetext{\color{black!70}#1}}
\def\deemph#1{{\color{black!70}#1}}
\let\phi\varphi

\setbeamertemplate{theorems}[numbered]
%\useinnertheme{circles}
\setbeamertemplate{itemize subitems}[triangle]

\renewcommand{\beamertemplatetransparentcovereddynamic}{
  \beamersetuncovermixins
  {\opaqueness<1>{50}\opaqueness<2>{30}\opaqueness<3>{15}\opaqueness<4->{5}}%
  {\opaqueness<1>{50}\opaqueness<2>{30}\opaqueness<3>{15}\opaqueness<4->{5}}}
                                                                                
%\beamertemplatetransparentcovereddynamic

\defbeamertemplate*{footline}{my theme}
{%
  \leavevmode%
  \hbox{\begin{beamercolorbox}[wd=.5\paperwidth,ht=2.5ex,dp=1.125ex,leftskip=.3cm,rightskip=.3cm]{author in head/foot}%
    \insertframenumber/\inserttotalframenumber \hfil \usebeamerfont{author in head/foot}\insertshortauthor
  \end{beamercolorbox}%
  \begin{beamercolorbox}[wd=.5\paperwidth,ht=2.5ex,dp=1.125ex,leftskip=.3cm,rightskip=.3cm plus1fil]{title in head/foot}%
    \usebeamerfont{title in head/foot}Logic I F13---\insertshorttitle---\insertdate
  \end{beamercolorbox}}%
  \vskip0pt%
}

\begin{document}
\setlength{\fitchargwidth}{7em}
\setlength{\fitchprfwidth}{7em}

\frame{\frametitle{\insertshorttitle\ (\insertdate)}
\tableofcontents[hidesubsections]
}










\section{The Limits of Tarski's World}

\subsec{Limitations of Tarski's World}{

\bit
\item Only $8 \times 8$ squares
\item Only three shapes, three sizes
\item Nothing can adjoin a large block
\item No more than 12 blocks on the grid
\eit

}

\subsec{Sentences Which are Always True}{

\[
\sf Cube(a) \lor \lnot Cube(a)\]\[
\sf Smaller(a, b) \lor Larger(a, b) \lor SameSize(a, b)
\]
\[
\color{red}\sf Cube(a) \lor Tet(a) \lor Dodec(a)
\]

}

\subsec{Tautologies, Logic Truths, TW Necessities}{

\bit
\item A sentence is a \emph{tautology} if it is true in every \emph{truth-value assignment}
\item A sentence is a \emph{logical truth} if it is true in every \emph{logically possible world}
\item A sentence is a \emph{TW necessity} if it is true in every \emph{TW world}
\eit

}

\subsec{TW Validity, Possible Worlds, TV Assignments}{

\bit
\item Every TW-world is also a possible world
\item Every possible world generates a truth-value assignment which makes the same sentences true/false.
\item Not TW-necessity $\Rightarrow$ Not logical necessity $\Rightarrow$ not tautology
\item tautology $\Rightarrow$ logical necessity $\Rightarrow$ TW-necessity
\eit
}

\subsec{Overview}{

\small

\begin{tabular}{l|lll}
Concept & applies to & definition & based on \\\hline\hline
\begin{tabular}{@{}l@{}}
tautology $\Downarrow$ \\
logical truth $\Downarrow$ \\
TW-necessity
\end{tabular} & sentences & never false &
\begin{tabular}{@{}l@{}}
TV-assignment \\
possible world\\
TW-world
\end{tabular}\\
\hline
\begin{tabular}{@{}l@{}}
tautologically valid $\Downarrow$ \\
logically valid
\end{tabular} & arguments & 
\begin{tabular}{@{}l@{}}
never true \\ premises and \\ false conclusion
\end{tabular} & 
\begin{tabular}{@{}l@{}}
TV-assignment \\
possible world
\end{tabular}\\
\hline
\begin{tabular}{@{}l@{}}
tautologically \\ \quad equivalent $\Downarrow$ \\
logically \\ \quad equivalent
\end{tabular} & 2 sentences & 
\begin{tabular}{@{}l@{}}
never different \\ truth values
\end{tabular} & 
\begin{tabular}{@{}l@{}}
TV-assignment \\[2ex]
possible world
\end{tabular}
\end{tabular}

}


\section{Informal Proofs involving the Boolean Connectives}

\subsec{Proof}{

\bit
\item Demonstration of conclusion from premises
\item Proceeds step-by-step
\item Each step justified
\item Rigorous (without gaps)
\item Informal vs. formal
\eit
}


\subsec{Justification in Informal Proofs}{

\bit
\item Basic properties of predicate symbols
\item Appeal to truth tables
\item Use special patterns of inference that are justified by truth tables
\item Type of consequence:
\bit
\item If proof needs first: logical consequence
\item If proof only needs last two: tautological consequence
\eit
\eit

}

\subsec{Appealing to Truth Tables}{

\setlength{\fitchprfwidth}{1.2in}

\begin{columns}
\begin{column}{3cm}
\fitchprf{
\nline{A \lor B}\\
\nline{\lnot A}}{\nline{B}}
\end{column}
\begin{column}{7cm}
\bens
\item We are told that $\sf \lnot A$ is true (premise 2).  \item $\sf \lnot
A$ is true iff $\sf A$ is false (by the truth table for $\sf \lnot $),
so $\sf A$ is false.  \item By premise 1, $\sf A \lor B$ is true.  \item The only
line in the truth table for $\sf \lor$ where $\sf A$ is false and $\sf
A \lor B$ is true is where $\sf B$ is true, so $\sf B$ must be
true. 
\een
\end{column}
\end{columns}
}

\subsec{Basic Properties of $\land$, $\lor$, $\lor$}{

\bit
\item If $A \land B$ is true, then both $A$ and $B$ are true
\item If $A$ and $B$ are both true, then $A \land B$ is true
\item If $A$ is true, then $A \lor B$ is true
\item If $B$ is true, then $A \lor B$ is true
\item If $A$ is false, then $A \land B$ is false
\item If $B$ is false, then $A \land B$ is false
\item If $A \lor B$ is false, then both $A$ and $B$ are false
\item $A$ is true iff $\lnot A$ is false
\item $A$ is false iff $\lnot A$ is true
\item If $A \lor B$ is true and $A$ is false, then $B$ is true (Disjunctive Syllogism)
\eit

}

\subsec{Basic Patterns of Inference}{

\begin{itemize}
\item $\sf \land $--Elimination: If you know that $\sf A \land B$ is
  true, you may conclude that $\sf A$ is true. You may of course also
  conclude that $\sf B$ is true.
\item $\sf \land $--Introduction: If you know that $\sf A$ is true and
  that $\sf B$ is true, then you may conclude that $\sf A \land B$ is
  true.
\item $\sf \lor$--Introduction: If you know that $\sf A$ is true, then
  you may conclude that $\sf A \lor B$ is true (for whatever $\sf B$
  you like).
\item $\sf \lnot $--Elimination: If you know that $\sf \lnot \lnot A$
  is true (or that $\sf \lnot A$ is false), you may conclude that $\sf
  A$ is true.
\end{itemize}
}


\subsec{Example}{

\setlength{\fitchprfwidth}{1.2in}

\begin{columns}
\begin{column}{3cm}
\fitchprf{
\nline{Large(a) \land Cube(a)}\\
\nline{Small(b) \land Cube(b)}}{\nline{Cube(a) \land Cube(b)}}
\end{column}
\begin{column}{7cm}
\bens
\item We are told that a is a large cube (premise 1).  \item By $\land$Elim, a is a cube. \item We are also told that b is a small cube (premise 2). \item So, b is a cube (by $\land$Elim).  \item By $\land$Intro, a and b are cubes.
\een
\end{column}
\end{columns}
}

\section{Formal Rules}

\subsec{$\land$Intro}{

If $A$ and $B$ are both true, then $A \land B$ is true
\bigskip
\setlength{\fitchprfwidth}{1.2in}
\fitchctx{
\nline[$n.$]{A}\\
\nline[$m.$]{B}\\
\fpline[\quad]{A \land B}[\landi{n}{m}]}

}

\subsec{$\land$Elim}{

If $A \land B$ is true, then $A$ ($B$) is true
\bigskip
\begin{tabular}{@{}ll@{}}
\setlength{\fitchprfwidth}{1in}
\fitchctx{
\nline[$n.$]{A \land B}\\
\fpline[\quad]{A}[\lande{n}]}
&\setlength{\fitchprfwidth}{1in}
\fitchctx{
\nline[$n.$]{A \land B}\\
\fpline[\quad]{B}[\lande{n}]}
\end{tabular}
}

\subsec{$\lor$Intro}{

If $A$ ($B$) is true, then $A \lor B$ is true
\bigskip
\begin{tabular}{@{}ll@{}}
\setlength{\fitchprfwidth}{1in}
\fitchctx{
\nline[$n.$]{A}\\
\fpline[\quad]{A \lor B}[\lori{n}]}
&\setlength{\fitchprfwidth}{1in}
\fitchctx{
\nline[$n.$]{B}\\
\fpline[\quad]{A \lor B}[\lori{n}]}
\end{tabular}
}

\subsec{Example}{
\setlength{\fitchprfwidth}{2in}
\fitchprf{\pline[1.]{Large(a) \land Cube(a)}\\
\pline[2.]{Small(b) \land Cube(b)}}{
\pline[3.]{Cube(a)}[\lande{1}]\\
\pline[4.]{Cube(b)}[\lande{2}]\\
\pline[5.]{Cube(a) \land Cube(b)}[\landi{3}{4}]}

}

\section{Proof by Cases}

\subsec{Proof by Cases}{
\begin{quote}
  Proof by cases: If you know that $\sf A \lor B$ is true, and you can
  prove that $\sf C$ is true if you assume that $\sf A$ is true, and
  you can also prove that $\sf C$ is true assuming that $\sf B$ is
  true, then you can conclude $\sf C$.
\end{quote}
}

\subsec{Example}{
\setlength{\fitchprfwidth}{1.2in}

\begin{columns}
\begin{column}{3.2cm}
\fitchprf{\pline{Cube(a) \lor {}}\\
\pline{\quad (Small(a) \land  Tet(a))}\\
\pline{\lnot Cube(a) \lor Large(a)}}{
\pline{Small(a) \lor Large(a)}}
\end{column}
\begin{column}{7cm}
\bens
\item We are told that either $\sf Cube(a)$ or $\sf Small(a) \land
Tet(a)$.  
\item Case 1. Suppose $\sf Cube(a)$ is true.  
\bit \item By the second
premise, either $\sf \lnot Cube(a)$ or $\sf Large(a)$.  
\item By Disjunctive
Syllogism, $\sf Large(a)$.  
\item By $\sf \lor$--Introduction, $\sf Small(a)
\lor Large(a)$\eit  
\een
\end{column}
\end{columns}
}

\subsec{Example}{
\setlength{\fitchprfwidth}{1.2in}

\begin{columns}
\begin{column}{3.2cm}
\fitchprf{\pline{Cube(a) \lor {}}\\
\pline{\quad (Small(a) \land  Tet(a))}\\
\pline{\lnot Cube(a) \lor Large(a)}}{
\pline{Small(a) \lor Large(a)}}
\end{column}
\begin{column}{7cm}
\bens
\item<1-> We are told that either $\sf Cube(a)$ or $\sf Small(a) \land
Tet(a)$.  
\item<1-> Case 1. Suppose $\sf Cube(a)$ is true.  
\bit \item<1-> \dots $\sf Small(a)
\lor Large(a)$\eit  
\item Case 2. Suppose $\sf Small(a) \land Tet(a)$ is
true. 
\bit 
\item Then by $\sf \land $--Elimination, $\sf Small(a)$ is true. 
\item By
$\sf \lor$--Introduction, $\sf Small(a) \lor Large(a)$
\eit
\item By proof by
cases, it follows that $\sf Small(a) \lor Large(a)$.
\een
\end{column}
\end{columns}
}

\subsec{$\lor$Elim}{

Formal version of proof by cases
\bigskip
\bigskip
\setlength{\fitchprfwidth}{1.2in}
\fitchctx{
\nline[$j.$]{A \lor B}\\
\subproof{\nline[$k.$]{A}}{\dots\\ \nline[$l.$]{C}}
\subproof{\nline[$m.$]{B}}{\dots\\ \nline[$n.$]{C}}
\fpline[\quad]{C}[\lore{$j$}{$k$--$l$}{$m$--$n$}]}


}

\subsec{Example}{

\fitchprf{\pline[1.]{A \lor (B \land C)}}{
\subproof{\pline[2.]{A}}{\pline[3.]{A \lor B}[\lori{2}]}
\subproof{\pline[4.]{B \land C}}{
\pline[5.]{B}[\lande{4}]\\
\pline[6.]{A \lor B}[\lori{5}]}
\pline[7.]{A \lor B}[\lore{1}{2--3}{4--6}]
}}


\end{document}


