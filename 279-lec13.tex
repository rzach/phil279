\documentclass[bigger]{beamer}

\usetheme{Frankfurt}
\useoutertheme[subsection=false]{miniframes}
\setbeamercolor{structure}{parent=palette quaternary,fg=uofcred}
\setbeamercolor{section in toc}{fg=black,bg=white}
\setbeamercolor{alerted text}{fg=uofcred}
\setbeamercolor*{palette quaternary}{fg=white,bg=black}
\setbeamercolor*{palette secondary}{fg=black,bg=uofcyellow}
\setbeamercolor*{palette tertiary}{bg=uofcred,fg=white}
\setbeamercolor*{palette primary}{fg=white,bg=black}
\setbeamercolor*{sidebar}{fg=uofcred,bg=black!15!white}
\setbeamercolor*{palette sidebar primary}{fg=uofcred!10!black}
\setbeamercolor*{palette sidebar secondary}{fg=white}
\setbeamercolor*{palette sidebar tertiary}{fg=uofcred!50!black}
\setbeamercolor*{palette sidebar quaternary}{fg=black!10!white}

\setbeamercolor{titlelike}{parent=palette primary}
\setbeamercolor{frametitle}{fg=white,bg=black}
\setbeamercolor{frametitle right}{bg=gray!60!white}
\let\emph\alert
\beamertemplatenavigationsymbolsvertical
\def\bits{\begin{itemize}[<+->]}
\def\bens{\begin{enumerate}[<+->]}
\def\items#1{\item<#1>}
\def\uncovers#1{\uncover<#1>}
\def\pauses{\pause}

 % change to header-handout for handouts

% ====================
\title[Lecture 13]{Logic I F13 Lecture 13}
\date{October 24, 2013}
% ====================

% lecture header include

\usepackage{lplfitch,amsmath}
\usepackage{qtree,hyperref}
\usepackage{pgf,amssymb}

\author{Richard Zach}
\institute{Department of Philosophy\\
University of Calgary\\
  \href{http://ucalgary.ca/rzach/279}{ucalgary.ca/rzach/279}}

\definecolor{uofcred}{RGB}{227,39,38}
\definecolor{uofcyellow}{RGB}{255,210,0}

\DeclareSymbolFont{symbolsC}{U}{txsyc}{m}{n}
\DeclareMathSymbol{\strictif}{\mathrel}{symbolsC}{74}
\DeclareMathSymbol{\boxright}{\mathrel}{symbolsC}{128}

\let\IFF\Leftrightarrow
\let\iff\leftrightarrow
\let\impl\to

\def\T{{\color{green}\begin{colormixin}{25!black}\text{T}\end{colormixin}}}
\def\F{{\color{red}\begin{colormixin}{25!black}\text{F}\end{colormixin}}}

\long\def\subsec#1#2{\subsection{#1}\frame{\frametitle{#1} #2}}
\def\bit{\begin{itemize}[<1->]}
\def\eit{\end{itemize}}
\def\ben{\begin{enumerate}[<1->]}
\def\een{\end{enumerate}}

\makeatletter\let\@makefnmark\noindent\makeatother

%\setbeamercolor{footnote}{fg=black!70}

\def\foot#1{\footnotetext{\color{black!70}#1}}
\def\deemph#1{{\color{black!70}#1}}
\let\phi\varphi

\setbeamertemplate{theorems}[numbered]
%\useinnertheme{circles}
\setbeamertemplate{itemize subitems}[triangle]

\renewcommand{\beamertemplatetransparentcovereddynamic}{
  \beamersetuncovermixins
  {\opaqueness<1>{50}\opaqueness<2>{30}\opaqueness<3>{15}\opaqueness<4->{5}}%
  {\opaqueness<1>{50}\opaqueness<2>{30}\opaqueness<3>{15}\opaqueness<4->{5}}}
                                                                                
%\beamertemplatetransparentcovereddynamic

\defbeamertemplate*{footline}{my theme}
{%
  \leavevmode%
  \hbox{\begin{beamercolorbox}[wd=.5\paperwidth,ht=2.5ex,dp=1.125ex,leftskip=.3cm,rightskip=.3cm]{author in head/foot}%
    \insertframenumber/\inserttotalframenumber \hfil \usebeamerfont{author in head/foot}\insertshortauthor
  \end{beamercolorbox}%
  \begin{beamercolorbox}[wd=.5\paperwidth,ht=2.5ex,dp=1.125ex,leftskip=.3cm,rightskip=.3cm plus1fil]{title in head/foot}%
    \usebeamerfont{title in head/foot}Logic I F13---\insertshorttitle---\insertdate
  \end{beamercolorbox}}%
  \vskip0pt%
}

\begin{document}
\setlength{\fitchargwidth}{7em}
\setlength{\fitchprfwidth}{7em}

\frame{\frametitle{\insertshorttitle\ (\insertdate)}
\tableofcontents[hidesubsections]
}










\setlength{\fitchprfwidth}{5em}
\setlength{\fitchargwidth}{25em}

\section{Review}

\subsec{Properties and Quantifiers}{

\bit
\item Wff $A(x)$ expresses a \emph{property}
\item The sentence
\[
\forall x\, A(x)
\]
is true iff 
\emph{every object} in the domain has the property expressed by A(x),
\item The sentence
\[\sf
\exists x\, A(x)
\]
is true iff
\emph{at least one object} in the domain has the property expressed by A(x),
\eit

}

\subsec{The Aristotelian Forms}{

\bit
\item ``Some A is B''
\[
\sf \exists x(A(x) \land B(x)
\]
\item ``Every A is B'' (``All A are B'')
\[\sf \forall x(A(x) \to B(x))\]
\item ``No A is B''
\begin{align*}
\sf \forall x(A(x) & \to \sf\lnot B(x))\\
\sf \lnot\exists x(A(x) & \sf\land B(x))
\end{align*}
\item ``Some A is not B'' (``Not all A are B'')
\begin{align*}
\sf \exists x(A(x) & \land \sf\lnot B(x)) \\
\sf \lnot\forall x(A(x) & \to \sf B(x))
\end{align*}

\eit

}

\subsec{Only, Except}{

\bit
\item ``Only A are B''
\begin{align*}
\sf \forall x(B(x) & \sf\to A(x))\\
\sf \lnot\exists x(\lnot A(x) & \sf\land B(x))
\end{align*}
\item ``Only a is large''
\[
\sf \forall x(Large(x) \iff x = a)
\]
\eit
}

\subsec{Existential Import and Implicature}{

\bit
\item ``All A are B'' does not imply ``There are A''
\item $\sf \forall x(A(x) \impl B(x))$ is true if nothing satisfies $A(x)$\\
(vacuous quantification)
\item ``Some A are B'' does not imply ``Not all A are B''
\item \dots but they do implicate it
\eit

}

\subsec{Quantifiers and Function Symbols}{

\bits
\item ``The leftmost block in the same row as any cube is small''
\bits
\item ``For every cube, the leftmost block in the same row as it is small''
\item
\[
\sf \forall x(Cube(x) \to Small(lm(x)))
\]
\eit
\item ``Some multiples of 3 are even, and some aren't''
\bits
\item Note: every multiple of 3 is the multiple of 3 \emph{by some number}
\item ``There is a number such that the multiple of 3 by it is even''
\item 
$
\sf \exists x\,Even(x \times (1 + (1+1)))
$
\item
$
\sf \exists x\, Even(x \times (1 + (1+1))) \land \exists x\, \lnot Even(x \times (1 + (1+1)))
$
\eit
\eit
}





\section{Arguments involving Quantifiers}

\subsec{Example}{

\ben
\item If an action x is morally wrong then 
A is blameworthy for freely doing x. 
\item If x is rationally optimal (there is no action which A has
  reason to think there is more reason for A to do), then A is not
  blameworthy for freely doing x.
\item Therefore, if x is morally wrong, then 
x is not rationally optimal. 
(Principle of moral categoricity.)
\een

\begin{raggedleft}
\small(John Skorupski, \textit{Ethical Explorations}, 2000 (\href{http://books.google.ca/books?id=bxIzZYqRZdwC&lpg=PP1&pg=PA170\#v=onepage&q&f=false}{link} \href{http://peasoup.typepad.com/peasoup/2008/07/logic-help.html}{link}))
 \end{raggedleft}

}

\subsec{Example}{

\fitcharg{
\nline{\forall x(Wrong(x) \to BlameFreelyDoing(A, x))}\\
\nline{\forall x(ROpt(x) \to \lnot BlameFreelyDoing(A, x))}}{
\nline{\forall x(Wrong(x) \to \lnot ROpt(x))}\\
}

\bigskip

\fitcharg{\tline{All A are B}\\
\tline{No C are B \qquad ($\Leftrightarrow$ No B are C)}}{
\tline{No A are C}}
}



\subsec{FOL and the Truth Table Method}{

\bit
\item Some arguments valid only in virtue of the meanings of quantifiers
\item Some arguments valid independently of the meanings of quantifiers, e.g.,
\fitcharg{\tline{If there are cubes, then there are also tetrahedra}\\
\tline{There are no tetrahedra}}{\tline{There are no cubes}}
\item Challenge: how does the truth-table method apply?  When should we say an argument is \emph{tautologically} valid even though it contains quantifiers?

\eit

}

\subsec{The Truth-Functional Form Algorithm}{

\bit
\item Underline every quantifier and its entire scope (unless it is in the scope of another quantifier)
\item Label every underlined bit with a schematic sentence letter (A, B, C, \dots), use the same letter if the underlined parts are \emph{identical}
\item Replace underlined parts by sentence letters
\item Result is the \emph{truth-functional form} of the argument
\item An argument (with quantifiers) is \emph{tautologically valid} if its truthfunctional form is tautologically valid
\eit

}

\subsec{The Truth-Functional Form Algorithm}{

\fitcharg{
\nline{\underline{\exists x\, Cube(x)}_A \to \underline{\exists x\, Tet(x)}_B}\\
\nline{\lnot\underline{\exists x\, Tet(x)}_B}}{
\nline{\lnot\underline{\exists x\, Cube(x)}_A}}

\bigskip

\fitcharg{
\nline{A \to B}\\
\nline{\lnot B}}{\nline{\lnot A}}

}

\subsec{The Truth-Functional Form Algorithm}{

\fitcharg{
\nline{\underline{\exists x\, Cube(x)}_A \to \underline{\exists x\, Tet(x)}_B}\\
\nline{\underline{\forall x\, \lnot Tet(x)}_C}}{
\nline{\underline{\forall x\, \lnot Cube(x)}_D}}

\bigskip

\fitcharg{
\nline{A \to B}\\
\nline{C}}{\nline{D}}

}

\subsec{First-order Validity}{

\bit
\item Want to capture validity \emph{in virtue of the meanings of the
  connectives and the quantifiers} (but ignoring meanings of predicate
  symbols other than =)
\item So we want to ignore any restrictions the predicate symbols place on their extensions
\item Hence: allow \emph{any} extension in a potential counterexample
\item An argument is \emph{first-order valid} if there is no \emph{first-order structure} in which the premises are true and the conclusion false

\item Variant interpretation: \emph{first-order structure}
\eit

}

\subsec{The Replacement Method}{

\bit
\item Existing meanings of predicate symbols may confuse you, so:
\item Replace each predicate symbol (other than =) with a ``nonsense predicate'': A, B, C, \dots
\item Arity the same as original predicate
\eit
}

\subsec{First-order Structure}{

\bit
\item Specify domain (e.g., ``Canadian cities'', ``the natural numbers'', graphically using a collection of dots, \dots)
\item Specify extension of A(x), B(x, y), C(x, y, z), \dots{} by
\bit
\item a diagram (as usual)
\item enumeration of elements, pairs of elements, triples of elements of the domain, e.g., ``Calgary, Edmonton, Lethbridge'', ``(0, 1), (0, 2), (0, 3), \dots''
\item determinate predicate (expression involving x, y, z), e.g., ``x is entirely located in Alberta'', ``x evenly divides y'', \dots
\eit
\item Specify referents of constant symbols, e.g., ``c: Calgary'', ``d: 5'', graphically by writing the constant next to the dot named by it
\eit
}

\section{FO Consequence and FO Structures}

\subsec{First-order Logical Notions}{

\bit
\item An argument is \emph{first-order valid}, and its conclusion is a \emph{first-order consequence} of its premises,
if there is no \emph{first-order structure} in which the premises are true and the conclusion is false
\item A sentence is a \emph{first-order validity} if it is true in every first-order stucture
\item Two sentences are \emph{first-order equivalent} if there is no first-order structure in which they have different truth values.
\item A (set of) sentence(s) is \emph{first-order satisfiable} if it is (all) true in at least one first-order structure
\item NB: There is \emph{no mechanical test} for any of these (The Church-Turing Undecidability Theorem), by contrast to the truth-table case!
\eit

}

\subsec{First-order Structure}{

\bit
\item $\sf \forall x\,(P(x) \impl R(x, b))$
\item We need:
\begin{itemize}
\item A domain: objects which can take the value of \textsf x. 
\item Extension of $\sf P(x)$ (or a determinate property of the objects in that domain as an
interpretation for $\sf P(x)$). 
\item Extension of $\sf R(x, y)$ (or a determinate 2-place relation as an interpretation of the
predicate $\sf R(x, y)$). 
\item An object to be named by \textsf a.
\end{itemize}
\eit
}


\subsec{FO Validity of Arguments}{

\fitcharg{
\nline{\forall x\, Cube(x) \lor \forall x\, Tet(x)}\\
\nline{\exists x\, \lnot Cube(x) \land \forall y(Tet(y) \to Small(y))}}{
\nline{\lnot \exists x\, Large(x)}}
}

\subsec{FO (In)Validity of Arguments}{

\fitcharg{
\nline{\forall x\, A(x) \lor \forall x\, B(x)}\\
\nline{\exists x\, \lnot A(x) \land \forall y(B(y) \to C(y))}}{
\nline{\lnot \exists x D(x)}}

\begin{itemize}
\item Domain: the planets
\item $\sf A(x)$ \dots \textsf x is inhabited
\item $\sf B(x)$ \dots \textsf x is smaller than the Sun
\item $\sf C(x)$ \dots \textsf x orbits the Sun
\item $\sf D(x)$ \dots \textsf x has a moon 
\end{itemize}
Another FO counterexample given by an Euler diagram: 
\[
\begin{array}{|lc|}
\hline
\sf B = C = D & \\
& \begin{array}{|lc|}
\hline
\sf A & \cdot 1\\
\hline
\end{array}\quad \\
\cdot 2 & \\ \hline
\end{array}
\]

}

\subsec{Making ``All As are Bs'' True}{

\begin{align*} 
\sf \forall x\, (A(x) & \impl \sf B(x)) \\
\sf \lnot\exists x(A(x) & \land \sf\lnot B(x))
\end{align*}

\[
\begin{array}{|lc|}
\hline 
\sf B & \cdots \\
& \begin{array}{|c|}
\hline 
\sf A \cdots \\
\hline
\end{array} \quad \\[1ex]
\hline
\end{array}
\qquad\qquad
\begin{array}{|lc|}
\hline 
\sf B = A & \\
& \cdots \\
\hline
\end{array}
\qquad\qquad
\begin{array}{c}
\begin{array}{|lc|}
\hline 
\sf B & \\
& \cdots \\
\hline 
\end{array} \\[2ex]
\sf A = \emptyset
\end{array}
\]


Makes ``Some As are not Bs'' \emph{false}
}

\subsec{Making ``All As are Bs'' False}{

\begin{align*}
\sf \lnot\forall x\, (A(x) & \impl \sf B(x)) \\
\sf \exists x(A(x) & \land \sf\lnot B(x))
\end{align*}

\[
\begin{array}{lcr|}
  \cline{1-2}
  \multicolumn{2}{|c|}{\sf A \quad \cdots}\\
  \cline{2-3}
  \multicolumn{1}{|c|}{} & \multicolumn{1}{|c|}{\cdots} & 
  \multicolumn{1}{|c|}{} \\  
  \cline{1-2} 
  & \multicolumn{2}{|c|}{\sf B \quad \cdots}\\
  \cline{2-3}
\end{array}
\qquad\qquad
\begin{array}{|lc|c|lc|}
\cline{1-2}\cline{4-5}
\sf A  & & & \sf B & \\
& \cdots & & & \cdots\\
\cline{1-2}\cline{4-5} 
\end{array}
\qquad\qquad
\begin{array}{c}
\begin{array}{|lc|}
\hline 
\sf A & \\
& \cdots \\
\hline 
\end{array} \\[1ex]
\sf B = \emptyset
\end{array}
\]

Makes ``Some As are not Bs'' \emph{true}

}

\subsec{Making ``Some As are Bs'' True}{

\begin{align*}
\sf \exists x (A(x) & \sf\land B(x)) \\
\sf \lnot\forall x(A(x) & \impl\sf \lnot B(x))
\end{align*}

\[
\begin{array}{lcr|}
  \cline{1-2}
  \multicolumn{2}{|c|}{\sf A \quad \cdots}\\
  \cline{2-3}
  \multicolumn{1}{|c|}{} & \multicolumn{1}{|c|}{\cdots} & 
  \multicolumn{1}{|c|}{} \\  
  \cline{1-2} 
  & \multicolumn{2}{|c|}{\sf B \quad \cdots}\\
  \cline{2-3}
\end{array}
\qquad\qquad
\begin{array}{|lc|}
\hline 
\sf A & \cdots \\
& \begin{array}{|c|}
\hline 
\sf B \cdots \\
\hline
\end{array} \quad \\[1ex]
\hline
\end{array}
\qquad\qquad
\begin{array}{|lc|}
\hline 
\sf B & \cdots \\
& \begin{array}{|c|}
\hline 
\sf A \cdots \\
\hline
\end{array} \quad \\[1ex]
\hline
\end{array}
\qquad\qquad
\begin{array}{|lc|}
\hline 
\sf B = A & \\
& \cdots \\
\hline
\end{array}
\]

Makes ``No As are Bs'' \emph{false}

}

\subsec{Making ``Some As are Bs'' False}{

\begin{align*}
\sf \lnot\exists x (A(x) & \sf\land B(x)) \\
\sf \forall x(A(x) & \impl\sf \lnot B(x))
\end{align*}

\[
\begin{array}{|lc|c|lc|}
\cline{1-2}\cline{4-5}
\sf A  & & & \sf B & \\
& \cdots & & & \cdots\\
\cline{1-2}\cline{4-5} 
\end{array}
\qquad\qquad
\begin{array}{c}
\begin{array}{|lc|}
\hline 
\sf A & \\
& \cdots \\
\hline
\end{array}\\[2ex]
\sf B = \emptyset
\end{array}
\qquad\qquad
\begin{array}{c}
\begin{array}{|lc|}
\hline 
\sf B & \\
& \cdots \\
\hline
\end{array} \\[2ex]
\sf A = \emptyset
\end{array}
\]

Makes ``No As are Bs'' \emph{true}
}


\subsec{Skorupski Revisited}{

\fitchprf{\pline[1.]{\text{If $\sf x$ is morally wrong, then}}\\
\pline{\qquad \text{$\sf A$ is blameworthy for freely doing $\sf x$}}\\
  \pline[2.]{\text{If $\sf x$ is
  rationally optimal, then }}\\
\pline{\qquad\text{A is not blameworthy for freely doing x.}}}
{\pline[3.]{\text{If $\sf x$ is morally wrong, then $\sf x$ is not rationally
      optimal.}}}

\bigskip

\fitchprf{\pline[1.]{\forall x(Wrong(x) \lif BlameFreelyDoing(a, x))}\\
  \pline[2.]{\forall x(ROpt(x) \lif \lnot BlameFreelyDoing(a, x))}}
{\pline[3.]{\forall x(Wrong(x) \lif \lnot ROpt(x))}}
}

\subsec{Skorupski Revisited}{

\href{http://books.google.ca/books?id=bxIzZYqRZdwC&lpg=PP1&pg=PA170\#v=onepage&q&f=false}{Actual argument} from Skurupski's book:

\fitchprf{\pline[1.]{\text{If $\sf x$ is morally wrong and $\sf A$ freely does $\sf x$,
    then}}\\
\pline{\qquad\text{$\sf A$ is blameworthy for doing $\sf x$.}}\\ 
  \pline[2.]{\text{If doing $\sf x$ is
  rationally optimal, then}}\\
\pline{\qquad\text{there is no reason to blame $\sf A$ for freely doing
  $\sf x$.}}}
{\pline[3.]{\text{If $\sf x$ is morally wrong, then $\sf x$ is not rationally
      optimal.}}}

\bigskip

\fitchprf{\pline[1.]{\forall x((Wrong(x) \land DoesF(a, x)) \lif Blame(a, x))}\\
  \pline[2.]{\forall x((ROpt(x) \land DoesF(a, x)) \lif \lnot Blame(a, x))}}
{\pline[3.]{\forall x(Wrong(x) \lif \lnot ROpt(x))}}

}

\subsec{Skorupski Revisited}{

\fitchprf{\pline[1.]{\forall x((Wrong(x) \land DoesF(a, x)) \lif Blame(a, x))}\\
  \pline[2.]{\forall x((ROpt(x) \land DoesF(a, x)) \lif \lnot Blame(a, x))}}
{\pline[3.]{\forall x(Wrong(x) \lif \lnot ROpt(x))}}

\bigskip

\fitchprf{\pline[1.]{\forall x((Cube(x) \land LeftOf(a, x)) \lif SameRow(a, x))}\\
  \pline[2.]{\forall x((Small(x) \land LeftOf(a, x)) \lif \lnot SameRow(a, x))}}
{\pline[3.]{\forall x(Cube(x) \lif \lnot Small(x))}}
}



\end{document}




