\documentclass[bigger]{beamer}

\usetheme{Frankfurt}
\useoutertheme[subsection=false]{miniframes}
\setbeamercolor{structure}{parent=palette quaternary,fg=uofcred}
\setbeamercolor{section in toc}{fg=black,bg=white}
\setbeamercolor{alerted text}{fg=uofcred}
\setbeamercolor*{palette quaternary}{fg=white,bg=black}
\setbeamercolor*{palette secondary}{fg=black,bg=uofcyellow}
\setbeamercolor*{palette tertiary}{bg=uofcred,fg=white}
\setbeamercolor*{palette primary}{fg=white,bg=black}
\setbeamercolor*{sidebar}{fg=uofcred,bg=black!15!white}
\setbeamercolor*{palette sidebar primary}{fg=uofcred!10!black}
\setbeamercolor*{palette sidebar secondary}{fg=white}
\setbeamercolor*{palette sidebar tertiary}{fg=uofcred!50!black}
\setbeamercolor*{palette sidebar quaternary}{fg=black!10!white}

\setbeamercolor{titlelike}{parent=palette primary}
\setbeamercolor{frametitle}{fg=white,bg=black}
\setbeamercolor{frametitle right}{bg=gray!60!white}
\let\emph\alert
\beamertemplatenavigationsymbolsvertical
\def\bits{\begin{itemize}[<+->]}
\def\bens{\begin{enumerate}[<+->]}
\def\items#1{\item<#1>}
\def\uncovers#1{\uncover<#1>}
\def\pauses{\pause}

 % change to header-handout for handouts

% ====================
\title[Lecture 17]{Logic I F13 Lecture 17}
\date{November 7, 2013}
% ====================

% lecture header include

\usepackage{lplfitch,amsmath}
\usepackage{qtree,hyperref}
\usepackage{pgf,amssymb}

\author{Richard Zach}
\institute{Department of Philosophy\\
University of Calgary\\
  \href{http://ucalgary.ca/rzach/279}{ucalgary.ca/rzach/279}}

\definecolor{uofcred}{RGB}{227,39,38}
\definecolor{uofcyellow}{RGB}{255,210,0}

\DeclareSymbolFont{symbolsC}{U}{txsyc}{m}{n}
\DeclareMathSymbol{\strictif}{\mathrel}{symbolsC}{74}
\DeclareMathSymbol{\boxright}{\mathrel}{symbolsC}{128}

\let\IFF\Leftrightarrow
\let\iff\leftrightarrow
\let\impl\to

\def\T{{\color{green}\begin{colormixin}{25!black}\text{T}\end{colormixin}}}
\def\F{{\color{red}\begin{colormixin}{25!black}\text{F}\end{colormixin}}}

\long\def\subsec#1#2{\subsection{#1}\frame{\frametitle{#1} #2}}
\def\bit{\begin{itemize}[<1->]}
\def\eit{\end{itemize}}
\def\ben{\begin{enumerate}[<1->]}
\def\een{\end{enumerate}}

\makeatletter\let\@makefnmark\noindent\makeatother

%\setbeamercolor{footnote}{fg=black!70}

\def\foot#1{\footnotetext{\color{black!70}#1}}
\def\deemph#1{{\color{black!70}#1}}
\let\phi\varphi

\setbeamertemplate{theorems}[numbered]
%\useinnertheme{circles}
\setbeamertemplate{itemize subitems}[triangle]

\renewcommand{\beamertemplatetransparentcovereddynamic}{
  \beamersetuncovermixins
  {\opaqueness<1>{50}\opaqueness<2>{30}\opaqueness<3>{15}\opaqueness<4->{5}}%
  {\opaqueness<1>{50}\opaqueness<2>{30}\opaqueness<3>{15}\opaqueness<4->{5}}}
                                                                                
%\beamertemplatetransparentcovereddynamic

\defbeamertemplate*{footline}{my theme}
{%
  \leavevmode%
  \hbox{\begin{beamercolorbox}[wd=.5\paperwidth,ht=2.5ex,dp=1.125ex,leftskip=.3cm,rightskip=.3cm]{author in head/foot}%
    \insertframenumber/\inserttotalframenumber \hfil \usebeamerfont{author in head/foot}\insertshortauthor
  \end{beamercolorbox}%
  \begin{beamercolorbox}[wd=.5\paperwidth,ht=2.5ex,dp=1.125ex,leftskip=.3cm,rightskip=.3cm plus1fil]{title in head/foot}%
    \usebeamerfont{title in head/foot}Logic I F13---\insertshorttitle---\insertdate
  \end{beamercolorbox}}%
  \vskip0pt%
}

\begin{document}
\setlength{\fitchargwidth}{7em}
\setlength{\fitchprfwidth}{7em}

\frame{\frametitle{\insertshorttitle\ (\insertdate)}
\tableofcontents[hidesubsections]
}










\setlength{\fitchprfwidth}{14em}

\section{Rules for $\forall$}

\subsec{Rules for Fitch}{

\bit
\item Need rules for $\forall$ and $\exists$ for formal proofs
\item Formal proofs now more important, because no alternative
  (truth-table method)
\item Intro and Elim rules should be
\bit
\item simple
\item elegant (not involve other connectives or quantifiers)
\item yield only valid arguments
\eit 
\eit

}

\subsec{Candidates for Rules}{

\bit
\item Only simple sentence close to $\sf \forall x\, A(x)$ is $\sf A(c)$
\item Gives simple, elegant $\forall$Elim rule:

\fitchctx{
\nline[$k.$]{\forall x\, A(x)}\\
\fpline{\quad A(c)}[\lalle{}]
}
\item Problem: corresponding ``intro rule'' isn't valid:
\fitchctx{
\nline[$k.$]{A(c)}\\
\fpline{\quad\forall x\,A(x)}[\emph{doesn't follow!}]
}
\eit

}

\subsec{Names for Arbitrary Objects}{

\bit
\item Diagnosis: the c in A(c) is a name for a \emph{specific object}
We need a name for an \emph{arbitrary, unspecified object}
\item Then, if A(c) is true for whatever c could name, then A(x) is satisfied by \emph{every} object 
\item In fact, this is what we do when we give proofs of general claims, e.g.,

\fitchprf{\tline{All cubes adjoin a}\\
\tline{Only small things adjoin a}}{
\tline{All cubes are small}}

\pauses
Proof: Let Carl be any cube.  Since all cubes adjoin a, Carl adjoins
a. Since only small things adjoin a, Carl is small. But ``Carl''
stands for \emph{any} cube. So all cubes are small.
  \eit

}

\subsec{Arbitrary Objects in Fitch}{

\bit 
\item Every special constant with stands for an arbitrary object is
  introduced only pro tem
\item In particular, they never appear in the premises or conclusion of an argument
\item In Fitch, this is enforced by requiring that special constants only appear in subproofs
\item To indicate that a constant names an arbitrary object, it is put in a box at the beginning of this subproof
\eit

}

\subsec{General Conditional Proof}{

\setlength{\fitchctxwidth}{10em}

\fitchctx{
\boxedsubproof[$m$.]{c}{A(c)}{\ellipsesline \\ \nline[$n.$]{B(c)}}
\fpline{\quad \forall x(A(x) \to B(x))}[\lalli{$m$--$n$}]}

\bit
\item c is special: c must not appear anywhere outside the subproof
\eit
}

\subsec{Universal Generalization}{

\fitchctx{
\boxedsubproof[$m$.]{c}{}{\ellipsesline \\ \nline[$n.$]{A(c)}}
\fpline{\quad \forall x\,A(x)}[\lalli{$m$--$n$}]}
\bit
\item c is special: c must not appear anywhere outside the subproof
\eit
}

\subsec{Example}{

\fitchprf{\tline{All cubes adjoin a}\\
\tline{Only small things adjoin a}}{
\tline{All cubes are small}}

\bigskip

\fitchprf{\pline{\forall x\, (Cube(x) \to Adjoins(x, a))}\\
\pline{\forall x(Adjoins(x, a) \to Small(x))}}{
\pline{\forall x(Cube(x) \to Small(x))}}

}

\subsec{Example}{

\fitchprf{\pline[1.]{\forall x\, (Cube(x) \to Adjoins(x, a))}\\
\pline[2.]{\forall x(Adjoins(x, a) \to Small(x))}}{
\boxedsubproof[3.]{c}{Cube(c)}{\pline[4.]{Cube(c) \to Adjoins(c, a)}[\lalle{1}] \\
\pline[5.]{Adjoins(c, a)}[\life{3}{4}]\\
\pline[6.]{Adjoins(c, a) \to Small(c)}[\lalle{2}] \\
\pline[7.]{Small(c)}[\life{5}{6}]}
\pline[8.]{\forall x(Cube(x) \to Small(x))}[\lalli{3--7}]}

}

\section{Rules for $\exists$}

\subsec{Intro Rule for $\exists$}{

\bit
\item If we know of a specific object that it satisfies A(x), we know that at least one object satisfies A(x)
\item So this rule is valid:

\fitchctx{\pline[$m.$]{A(c)}\\
\fpline{\quad \exists x\, A(x)}[\lexii{$m$}]}
\item Problem: corresponding ``elim rule'' isn't valid:
\fitchctx{
\nline{\exists x\, A(x)}\\
\fpline{\quad A(c)}[\emph{doesn't follow!}]
}


\eit

}

\subsec{Arbitrary Objects Again}{

\bit
\item If we know that $\sf\exists x\, A(x)$ is true, we know that \emph{some} objects satisfy A(x), but not which ones
\item To use this information, we have to introduce a temporary name that stands for any one of the objects that satisfy A(x)
\item This is what we'd do if we reason informally from existential information, e.g.,

\fitchprf{\tline{There are small cubes}\\
\tline{Anything small adjoins a}}{
\tline{Some cubes adjoin a}}

\pause
Proof: We know there are small cubes. Let Cate be an arbitrary one of them.  So
Cate is small. Since all small things adjoins a, Cate adjoins a. Since
Cate is a cube adjoining a, some cubes adjoin a.
\eit 

}

\subsec{Existential Instantiation}{

\bit
\item If \bit \item we know that some object satisfies A(x), \item we
  assume for the time being that c is one of them (i.e., assume A(c)),
  and \item we can prove that B follows from this
  assumption, 
\eit
then B follows already from $\sf\exists x\, A(x)$.
\item Rule for existential elimination:

\fitchctx{
\pline[$k$.]{\exists x\, A(x)}\\
\boxedsubproof[$m$.]{c}{A(c)}{\ellipsesline \\ \nline[$n.$]{B}}
\fpline{\quad B}[\lexie{k}{$m$--$n$}]}
\item c is special: c must not appear outside subproof
\eit
}

\subsec{Example}{

\fitchprf{\tline{There are small cubes}\\
\tline{Anything small adjoins a}}{
\tline{Some cubes adjoin a}}

\bigskip

\fitchprf{\pline{\exists x(Small(x) \land Cube(x))}\\
\pline{\forall x(Small(x) \to Adjoins(x, a))}}{
\pline{\exists x(Cube(x) \land Adjoins(x, a)}}
}

\subsec{Example}{

\fitchprf{\pline[1.]{\exists x(Small(x) \land Cube(x))}\\
\pline[2.]{\forall x(Small(x) \to Adjoins(x, a))}}{
\boxedsubproof[3.]{c}{Small(c) \land Cube(c)}{
\pline[4.]{Small(c)}[\lande{3}] \\
\pline[5.]{Small(c) \to Adjoins(c, a)}[\lalle{2}] \\
\pline[6.]{Adjoins(c, a)}[\life{4}{5}]\\
\pline[7.]{Cube(c)}[\lande{3}] \\
\pline[8.]{Cube(c) \land Adjoins(c, a)}[\landi{6}{7}]\\
\pline[9.]{\exists x(Cube(x) \land Adjoins(x, a)}[\lexii{8}]
}
\pline[10.]{\exists x(Cube(x) \land Adjoins(x, a)}[\lexie{1}{3--9}]
}

}

\section{Proofs with Alternating Quantifiers}

\subsec{Haters and Hatees}{

\fitchprf{\tline{Someone is hated by everyone}}{
\tline{Everyone hates someone}}

\bigskip

\fitchprf{\pline{\exists y\forall x\, Hates(x, y)}}{
\pline{\forall x\exists y\, Hates(x, y)}}
}

\subsec{Queen Anne Superior}{

\setlength{\fitchprfwidth}{20em}

\fitchprf{\tline{Some woman is superior to every man}}{
\tline{Every man is inferior to some woman}}

\bigskip

\fitchprf{\pline{\exists y(W(y) \land \forall x(M(x) \to S(y, x)))}}{
\pline{\forall x(M(x) \to \exists y(W(y) \land S(y, x)))}}
}

\subsec{Everybody Loves my Baby}{

\href{http://www.dailymotion.com/swf/video/x3qmw4_brigitte-bardot-everybody-loves-my_music?additionalInfos=0}{Brigitte Bardot singing ``Everybody Loves my Baby''}

\pause
\bigskip

\fitchprf{\tline{Everybody loves my baby}\\
\tline{My baby loves noone but me}}{
\tline{My baby is me}}

\bigskip

\fitchprf{\pline{\forall x\, Loves(x, b)}\\
\pline{\forall x(Loves(b, x) \liff b = i)}}{
\pline{b = i}}
}



\end{document} 





