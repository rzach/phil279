\documentclass[bigger]{beamer}

\usetheme{Frankfurt}
\useoutertheme[subsection=false]{miniframes}
\setbeamercolor{structure}{parent=palette quaternary,fg=uofcred}
\setbeamercolor{section in toc}{fg=black,bg=white}
\setbeamercolor{alerted text}{fg=uofcred}
\setbeamercolor*{palette quaternary}{fg=white,bg=black}
\setbeamercolor*{palette secondary}{fg=black,bg=uofcyellow}
\setbeamercolor*{palette tertiary}{bg=uofcred,fg=white}
\setbeamercolor*{palette primary}{fg=white,bg=black}
\setbeamercolor*{sidebar}{fg=uofcred,bg=black!15!white}
\setbeamercolor*{palette sidebar primary}{fg=uofcred!10!black}
\setbeamercolor*{palette sidebar secondary}{fg=white}
\setbeamercolor*{palette sidebar tertiary}{fg=uofcred!50!black}
\setbeamercolor*{palette sidebar quaternary}{fg=black!10!white}

\setbeamercolor{titlelike}{parent=palette primary}
\setbeamercolor{frametitle}{fg=white,bg=black}
\setbeamercolor{frametitle right}{bg=gray!60!white}
\let\emph\alert
\beamertemplatenavigationsymbolsvertical
\def\bits{\begin{itemize}[<+->]}
\def\bens{\begin{enumerate}[<+->]}
\def\items#1{\item<#1>}
\def\uncovers#1{\uncover<#1>}
\def\pauses{\pause}

 % change to header-handout for handouts

% ====================
\title[Lecture 4]{Logic I F13 Lecture 4}
\date{September 19, 2013}
% ====================

% lecture header include

\usepackage{lplfitch,amsmath}
\usepackage{qtree,hyperref}
\usepackage{pgf,amssymb}

\author{Richard Zach}
\institute{Department of Philosophy\\
University of Calgary\\
  \href{http://ucalgary.ca/rzach/279}{ucalgary.ca/rzach/279}}

\definecolor{uofcred}{RGB}{227,39,38}
\definecolor{uofcyellow}{RGB}{255,210,0}

\DeclareSymbolFont{symbolsC}{U}{txsyc}{m}{n}
\DeclareMathSymbol{\strictif}{\mathrel}{symbolsC}{74}
\DeclareMathSymbol{\boxright}{\mathrel}{symbolsC}{128}

\let\IFF\Leftrightarrow
\let\iff\leftrightarrow
\let\impl\to

\def\T{{\color{green}\begin{colormixin}{25!black}\text{T}\end{colormixin}}}
\def\F{{\color{red}\begin{colormixin}{25!black}\text{F}\end{colormixin}}}

\long\def\subsec#1#2{\subsection{#1}\frame{\frametitle{#1} #2}}
\def\bit{\begin{itemize}[<1->]}
\def\eit{\end{itemize}}
\def\ben{\begin{enumerate}[<1->]}
\def\een{\end{enumerate}}

\makeatletter\let\@makefnmark\noindent\makeatother

%\setbeamercolor{footnote}{fg=black!70}

\def\foot#1{\footnotetext{\color{black!70}#1}}
\def\deemph#1{{\color{black!70}#1}}
\let\phi\varphi

\setbeamertemplate{theorems}[numbered]
%\useinnertheme{circles}
\setbeamertemplate{itemize subitems}[triangle]

\renewcommand{\beamertemplatetransparentcovereddynamic}{
  \beamersetuncovermixins
  {\opaqueness<1>{50}\opaqueness<2>{30}\opaqueness<3>{15}\opaqueness<4->{5}}%
  {\opaqueness<1>{50}\opaqueness<2>{30}\opaqueness<3>{15}\opaqueness<4->{5}}}
                                                                                
%\beamertemplatetransparentcovereddynamic

\defbeamertemplate*{footline}{my theme}
{%
  \leavevmode%
  \hbox{\begin{beamercolorbox}[wd=.5\paperwidth,ht=2.5ex,dp=1.125ex,leftskip=.3cm,rightskip=.3cm]{author in head/foot}%
    \insertframenumber/\inserttotalframenumber \hfil \usebeamerfont{author in head/foot}\insertshortauthor
  \end{beamercolorbox}%
  \begin{beamercolorbox}[wd=.5\paperwidth,ht=2.5ex,dp=1.125ex,leftskip=.3cm,rightskip=.3cm plus1fil]{title in head/foot}%
    \usebeamerfont{title in head/foot}Logic I F13---\insertshorttitle---\insertdate
  \end{beamercolorbox}}%
  \vskip0pt%
}

\begin{document}
\setlength{\fitchargwidth}{7em}
\setlength{\fitchprfwidth}{7em}

\frame{\frametitle{\insertshorttitle\ (\insertdate)}
\tableofcontents[hidesubsections]
}










\section[Boolean Connectives]{The Boolean Connectives}

\subsec{The Boolean Connectives}{

\bit
\item Negation (``not''): $\lnot$
\item Conjunction (``and''): $\land$
\item Disjunction (inclusive ``or''): $\lor$
\eit

}

\subsec{Sentences of FOL}{

\bit
\item Every atomic sentence is a sentence (duh!)
\item If $P$ is a sentence, then $\lnot P$ is a sentence as well.
\bit
\item Note: no parentheses
\item Instead of $\lnot a = b$ we write $a \neq b$
\eit
\item If $P$ and $Q$ are sentences, then so are\\
$(P \land Q)$ and $(P \lor Q)$
\bit
\item Note parentheses
\item But: we will leave off the outermost parentheses
\eit
\eit
}

\subsec{Constructing Sentences}{

\bits
\item $\sf Large(a)$, $\sf Cube(a)$, $\sf a = b$
\item $\sf \lnot Large(a)$, $\sf \lnot Cube(a)$, $\sf a \neq b$
\item $\sf \lnot\lnot Large(a)$, $\sf \lnot\lnot\lnot Large(a)$, \dots
\item $\sf (Cube(a) \land \lnot Large(a))$
\item $\sf ((Cube(a) \lor Tet(a)) \land \lnot(Large(a) \lor Medium(a)))$\\
\dots
\eit
}

\subsec{Ill-formed Expressions}{

Examples of expressions that are \emph{not} sentences:
\bit
\item $\sf \lnot (a = b)$
\item $\sf Cube(a \land b)$
\item $\sf Cube(\lnot a)$
\item $\sf Large(Cube(a))$
\item $\sf Cube(a) \lor Small(a) \land LeftOf(a, c)$
\item $\sf (Large \land Cube)(a)$
\item 
\eit

}

\subsec{Semantics of the Boolean Connectives}{

\bit
\item Atomic sentences $P(a)$ are true or false according to whether the object named by $a$ has the determinate property assigned to $P$
\item Non-atomic sentences are true or false according to 
\bit
\item what the \emph{main connective} is (*)
\item the truth values of the \emph{immediate subsentences} ($P$, $Q$)
\item the \emph{truth table} for the main connective
\eit
\eit
\begin{align*}
\underbrace{(Cube(a) \lor Tet(a))}_P \underbrace{\land}_\text{*} \underbrace{\lnot(Large(a) \lor Medium(a))}_Q \\
\underbrace{\lnot}_*\underbrace{(Large(a) \lor Medium(a))}_P
\end{align*}

}

\subsec{Truth Tables for the Boolean Connectives}{

\[
\begin{array}{c|c}
P & \lnot P\\
\hline
\T & \F \\
\F & \T
\end{array}
\qquad
\begin{array}{cc|c}
P & Q & (P \land Q)\\
\hline
\T & \T & \T\\
\T & \F & \F\\
\F & \T & \F\\
\F & \F & \F
\end{array}
\qquad
\begin{array}{cc|c}
P & Q & (P \lor Q)\\
\hline
\T & \T & \T\\
\T & \F & \T\\
\F & \T & \T\\
\F & \F & \F
\end{array}
\]

}

\subsec{Truth Conditions of Sentences}{

\bit
\item $\lnot P$ is 
\bit 
\item true iff $P$ is false
\item false iff $P$ is true
\eit
\item $P \land Q$ is 
\bit
\item true iff $P$ and $Q$ are both true
\item false iff at least one of $P$, $Q$ is false
\eit
\item $P \lor Q$ is 
\bit
\item true iff at least one of $P$, $Q$ is true
\item false iff $P$ and $Q$ are both false
\eit
\eit
}

\section{English and FOL}

\subsec{Negation}{

\bit
\item To negate a sentence, you use ``isn't'' or ``doesn't'':
\bit
\item b is a cube --- b \emph{isn't} a cube
\item a adjoins b --- a \emph{doesn't} adjoin b
\eit
\item You can prefix a sentence by ``it isn't the case that'':
\bit
\item a adjoins b --- \emph{it isn't the case that} a adjoins b
\eit
\item ``neither \dots nor \dots''
\bit
\item \emph{Neither a nor b} is a cube.
\eit
\eit
}

\subsec{Coordination}{

\bit
\item Can combine names, verbs, adjectives, nouns, even whole clauses using ``conjunctions'' (and, or, yet, but, neither, \dots)
\bit
\item \emph{a and b} are cubes.
\item a \emph{equals and adjoins} b.
\item a is \emph{large or small}.
\item a is a \emph{tetrahedron or dodecahedron}.
\item a is a cube \emph{and} b is a tetrahedron.
\item a and b are left of c and d, but e is not left of c or not left of d.
\item a and b are large cubes both of which adjoin c.
\eit
\eit

}


\subsec{English into FOL}{

\bit
\item a adjoins b --- a \emph{doesn't} adjoin b\\
$\sf Adjoins(a, b)$ --- $\sf\lnot Adjoins(a, b)$
\item \emph{a and b} are cubes.\\
$\sf Cube(a) \land Cube(b)$
\item a \emph{is left of and adjoins} b.\\
$\sf LeftOf(a, b) \land Adjoins(a, b)$ 
\item a is \emph{large or small}.\\
$\sf Large(a) \lor Small(a)$
\item a is a \emph{tetrahedron or dodecahedron}.\\
$\sf Tet(a) \lor Dodec(a)$
\item a is a cube \emph{but} b is a tetrahedron.\\
$\sf Cube(a) \land Tet(b)$
\eit
}



\section[Expressive Power]{Expressive Power of the Boolean Connectives}

\subsec{Neither \dots nor \dots}{

\bits
\item Neither a nor b is a cube
\item $\sf \lnot Cube(a) \land \lnot Cube(b)$
\item $\sf \lnot (Cube(a) \lor Cube(b))$
\eit

}

\subsec{Not Both}{

\bits
\item a and b are not both large
\item $\sf \lnot(Large(a) \land Large(b))$
\item $\sf \lnot Large(a) \lor \lnot Large(b)$
\eit

}

\subsec{Exclusive ``Or''}{

\bits
\item a or b is a cube (but not both)
\item $\sf (Cube(a) \lor Cube(b)) \land \lnot(Cube(a) \land Cube(b))$
\item $\sf (Cube(a) \land \lnot Cube(b)) \lor (\lnot Cube(a) \land Cube(b))$
\eit
}

\subsec{At Least One Of}{

\bits
\item At least one of a, b, and c is small
\item $\sf Small(a) \lor Small(b) \lor Small(c)$
\eit 

}

\subsec{At Least One Of}{

\bits
\item At least two of a, b, and c are small
\item $\sf (Small(a) \land Small(b)) \lor (Small(a) \land Small(c)) \lor (Small(b) \land Small(c))$
\eit 

}

\subsec{At Least As Large As and $\le$}{

\bits
\item a is at least as large as b
\item $\sf Larger(a, b) \lor SameSize(a, b)$
\item $\sf \lnot Smaller(a, b)$, $\sf\lnot Larger(b, a)$
\item $2 \le 3$
\item $(1+1) < (1+(1+1)) \lor (1+1) = (1+(1+1))$
\item $\lnot (1 + (1+1)) < (1+1)$ 
\eit

}

\end{document} %got to here

\section{Step-by-Step Translations}

\subsec{Step-by-Step Method of Translation}{

\bit
\item a and b are left of c or right of d, \emph{but}\\ e is neither left of c nor right of d.
\bit
\item a and b are left of c or right of d
\item but
\item e is neither left of c nor right of d
\eit\eit
}

\subsec{Step-by-Step Method of Translation}{

\bits
\item a \emph{and} b are left of c or right of d
\bit
\item a is left of c \emph{or} right of d
\bit
\item a is left of c or a is right of d\\
$\sf LeftOf(a, c) \lor RightOf(a, d)$
\eit
\item b is left of c \emph{or} right of d
\bit
\item b is left of c or b is right of d\\
$\sf LeftOf(b, c) \lor RightOf(b, d)$
\eit
\eit
\eit
\begin{align*}
(LeftOf(a, c) \lor {} & RightOf(a, d)) \land {}\\
(LeftOf(b, c) \lor {} & RightOf(b, d))
\end{align*}

}

\subsec{Step-by-Step Method of Translation}{

\bits
\item e is neither left of c nor right of d
\[\sf \lnot(LeftOf(e, c) \lor RightOf(e, d))\]
\eit
}

\subsec{Step-by-Step Method of Translation}{

\bit
\item a and b are left of c or right of d, but\\ e is neither left of c nor right of d.
\bit
\item a and b are left of c or right of d
\begin{align*}
(LeftOf(a, c) \lor {} & RightOf(a, d)) \land {}\\
(LeftOf(b, c) \lor {} & RightOf(b, d))
\end{align*}
\item but
$\land$
\item e is neither left of c nor right of d
\[\sf \lnot(LeftOf(e, c) \lor RightOf(e, d))\]
\eit\eit
\begin{multline*}
((LeftOf(a, c) \lor RightOf(a, d)) \land {} \\
(LeftOf(b, c) \lor RightOf(b, d))) \land {}\\
\lnot(LeftOf(e, c) \lor RightOf(e, d))
\end{multline*}

}



\section{Ambiguity}

\subsec{Ambiguity in English}{

\bit
\item Lexical ambiguity: one word---many meanings \\
e.g., ``bank'', ``crane''
\item Syntactic ambiguity: one sentence---many readings\\
e.g., ``Flying planes can be dangerous''
\eit

}

\subsec{Connectives and Ambiguity}{

\bits
\item a adjoins b and c or d
\item a adjoins [[b and c] or d]\\
$\sf (Adjoins(a, b) \land Adjoins(a, c)) \lor Adjoins(a, d)$
\item a adjoins [b and [c or d]]\\
$\sf Adjoins(a, b) \land (Adjoins(a, c) \lor Adjoins(a, d))$
\eit

}

\subsec{The Man Who Was Hanged by a Comma}{

\bit
\item Sir Roger Casement (1864--1916)
\item British consul to Congo and Peru
\item Tried to recruit Irish revolutionaries in Germany during WWI
\item Tried for treason
\eit

}

\subsec{Treason Act of 1351}{

\small 
ITEM, Whereas divers Opinions have been before this Time in what Case
Treason shall be said, and in what not; the King, at the Request of
the Lords and of the Commons, hath made a Declaration in the Manner as
hereafter followeth, that is to say; When a Man doth compass or
imagine the Death of our Lord the King, or of our Lady his Queen or of
their eldest Son and Heir; or if a Man do violate the King’s
Companion, or the King’s eldest Daughter unmarried, or the Wife of the
King’s eldest Son and Heir; or \textbf{if a Man do levy War against our Lord
the King in his Realm, or be adherent to the King’s Enemies in his
Realm, giving to them Aid and Comfort in the Realm, or elsewhere}, and
thereof be probably attainted of open Deed by the People of their
Condition: \dots And it is to be
understood, that in the Cases above rehearsed, that ought to be judged
Treason which extends to our Lord the King, and his Royal Majesty:
\dots

}

\subsec{R v. Casement in the Blocks Language}{

\bits
\item a is a cube in front of b, or a tetrahedron in front of b[,] or in back of b.

\item Without comma:
\begin{align*}
(Cube(a) \land {} & FrontOf(a, b)) \lor {}\\
(Tet(a) \land {} & (FrontOf(a,
b) \lor BackOf(a, b)))
\end{align*}
\item With comma:
\begin{align*}
(Cube(a) \land {} & (FrontOf(a, b) \lor BackOf(a, b))) \lor {}\\
(Tet(a) \land {} & (FrontOf(a, b) \lor BackOf(a, b))
\end{align*}
\eit
}




\end{document}
