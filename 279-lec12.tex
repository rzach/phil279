\documentclass[bigger]{beamer}

\usetheme{Frankfurt}
\useoutertheme[subsection=false]{miniframes}
\setbeamercolor{structure}{parent=palette quaternary,fg=uofcred}
\setbeamercolor{section in toc}{fg=black,bg=white}
\setbeamercolor{alerted text}{fg=uofcred}
\setbeamercolor*{palette quaternary}{fg=white,bg=black}
\setbeamercolor*{palette secondary}{fg=black,bg=uofcyellow}
\setbeamercolor*{palette tertiary}{bg=uofcred,fg=white}
\setbeamercolor*{palette primary}{fg=white,bg=black}
\setbeamercolor*{sidebar}{fg=uofcred,bg=black!15!white}
\setbeamercolor*{palette sidebar primary}{fg=uofcred!10!black}
\setbeamercolor*{palette sidebar secondary}{fg=white}
\setbeamercolor*{palette sidebar tertiary}{fg=uofcred!50!black}
\setbeamercolor*{palette sidebar quaternary}{fg=black!10!white}

\setbeamercolor{titlelike}{parent=palette primary}
\setbeamercolor{frametitle}{fg=white,bg=black}
\setbeamercolor{frametitle right}{bg=gray!60!white}
\let\emph\alert
\beamertemplatenavigationsymbolsvertical
\def\bits{\begin{itemize}[<+->]}
\def\bens{\begin{enumerate}[<+->]}
\def\items#1{\item<#1>}
\def\uncovers#1{\uncover<#1>}
\def\pauses{\pause}

 % change to header-handout for handouts

% ====================
\title[Lecture 12]{Logic I F13 Lecture 12}
\date{October 22, 2013}
% ====================

% lecture header include

\usepackage{lplfitch,amsmath}
\usepackage{qtree,hyperref}
\usepackage{pgf,amssymb}

\author{Richard Zach}
\institute{Department of Philosophy\\
University of Calgary\\
  \href{http://ucalgary.ca/rzach/279}{ucalgary.ca/rzach/279}}

\definecolor{uofcred}{RGB}{227,39,38}
\definecolor{uofcyellow}{RGB}{255,210,0}

\DeclareSymbolFont{symbolsC}{U}{txsyc}{m}{n}
\DeclareMathSymbol{\strictif}{\mathrel}{symbolsC}{74}
\DeclareMathSymbol{\boxright}{\mathrel}{symbolsC}{128}

\let\IFF\Leftrightarrow
\let\iff\leftrightarrow
\let\impl\to

\def\T{{\color{green}\begin{colormixin}{25!black}\text{T}\end{colormixin}}}
\def\F{{\color{red}\begin{colormixin}{25!black}\text{F}\end{colormixin}}}

\long\def\subsec#1#2{\subsection{#1}\frame{\frametitle{#1} #2}}
\def\bit{\begin{itemize}[<1->]}
\def\eit{\end{itemize}}
\def\ben{\begin{enumerate}[<1->]}
\def\een{\end{enumerate}}

\makeatletter\let\@makefnmark\noindent\makeatother

%\setbeamercolor{footnote}{fg=black!70}

\def\foot#1{\footnotetext{\color{black!70}#1}}
\def\deemph#1{{\color{black!70}#1}}
\let\phi\varphi

\setbeamertemplate{theorems}[numbered]
%\useinnertheme{circles}
\setbeamertemplate{itemize subitems}[triangle]

\renewcommand{\beamertemplatetransparentcovereddynamic}{
  \beamersetuncovermixins
  {\opaqueness<1>{50}\opaqueness<2>{30}\opaqueness<3>{15}\opaqueness<4->{5}}%
  {\opaqueness<1>{50}\opaqueness<2>{30}\opaqueness<3>{15}\opaqueness<4->{5}}}
                                                                                
%\beamertemplatetransparentcovereddynamic

\defbeamertemplate*{footline}{my theme}
{%
  \leavevmode%
  \hbox{\begin{beamercolorbox}[wd=.5\paperwidth,ht=2.5ex,dp=1.125ex,leftskip=.3cm,rightskip=.3cm]{author in head/foot}%
    \insertframenumber/\inserttotalframenumber \hfil \usebeamerfont{author in head/foot}\insertshortauthor
  \end{beamercolorbox}%
  \begin{beamercolorbox}[wd=.5\paperwidth,ht=2.5ex,dp=1.125ex,leftskip=.3cm,rightskip=.3cm plus1fil]{title in head/foot}%
    \usebeamerfont{title in head/foot}Logic I F13---\insertshorttitle---\insertdate
  \end{beamercolorbox}}%
  \vskip0pt%
}

\begin{document}
\setlength{\fitchargwidth}{7em}
\setlength{\fitchprfwidth}{7em}

\frame{\frametitle{\insertshorttitle\ (\insertdate)}
\tableofcontents[hidesubsections]
}










\setlength{\fitchprfwidth}{5em}

\section{Review}

\subsec{Satisfaction}{

\bit 
\item Atomic wffs like $\sf Cube(x)$ express the properties we've
  assigned to the predicate symbol
\item More general: any wff $\sf P(x)$ expresses a property
\item Which one? The property of \emph{satisfying $\sf P(x)$}.
\item \emph{$\alpha$ satisfies $\sf P(x)$ in~$W$} iff $\sf P(n)$ is true in
  the world~$W'$ which is just like $W$ except $n$ names $\alpha$
\item Extension of $\sf P(x)$ in $W$: set of all objects that satisfy
  $\sf P(x)$ in $W$.
\item Examples:
\bit
\item $x$ is a small tetrahedron to the left of $b$:\[
\sf Small(x) \land Tet(x) \land LeftOf(x, b)
\]
\item $x$ adjoins  c or d, but not both
\[
\sf \lnot(Adjoins(x, c) \iff Adjoins(x, d))
\]
\eit
\eit

}

\subsec{Quantifiers and Satisfaction: $\forall$}{

\bit
\item The sentence
\[\sf
\forall x\, P(x)
\]
says (is true iff) 
\bit
\item every object in the domain (= world) satisfies P(x)
\item P(n) is true for \emph{whatever} object is named by~n
\item extension of P(x) in $W$ is the entire domain
\eit
\eit
}

\subsec{Quantifiers and Satisfaction: $\exists$}{

\bit
\item The sentence
\[\sf
\exists x\, P(x)
\]
says (is true iff) 
\bit
\item at least one object in the domain satisfies P(x)
\item P(n) is true for \emph{at least one} object named by~n
\item extension of P(x) in $W$ is not empty
\eit
\eit

}

\subsec{Expressing ``Everything'', ``Something''}{


\bit
\item ``Everything is a large cube''
\[ \sf \forall x(Large(x) \land Cube(x)) \]
\item ``Something is a large cube''
\[ \sf \exists x(Large(x) \land Cube(x)) \]
\eit
}

\subsec{Expressing ``Nothing'', ``Not Everything''}{

\bit\item ``Nothing is a large cube'' \pause
\begin{align*}
& \sf \forall x\,\lnot(Large(x) \land Cube(x)) \\
& \sf \lnot\exists x(Large(x) \land Cube(x))
\end{align*}
Also: ``Everything is: not a large cube''
\item ``Something isn't a large cube'' \pause
\begin{align*}
& \sf \exists x\,\lnot(Large(x) \land Cube(x)) \\
& \sf \lnot\forall x(Large(x) \land Cube(x))
\end{align*}
Also: ``Not everything is a large cube''
\eit
}

\section{Restricted Quantification}

\subsec{Determiner Phrases}{

\bit
\item Determiners combine with noun phrases to make determiner phrases (DP):
\bit
\item ``\emph{A} large terahedron''
\item ``\emph{Three} cubes which are to the left of b''
\item ``\emph{Every} even number''
\item ``\emph{Some} cube(s) between d and e''
\item ``\emph{Most} philosophy majors''
\item ``\emph{Both} small cubes''
\eit
\eit
}

\subsec{Determiner Phrases in Sentences}{

\bit
\item DPs make subjects of sentence, just like names/constants and coordination constructions of them do
\bit
\item ``\emph{Claire and Alex} study logic''---``\emph{Most philosophy majors} study logic''
\item ``\emph{2} is prime''---``\emph{Some even number} is prime''
\item ``\emph{a} is between b and c''---``\emph{Every large cube} is between b and c''
\eit
\item We know how to translate the former---how do we deal with the latter?
\eit
}

\subsec{Restricted Quantification}{

\bit
\item ``Det A is/are B''
\item A is a noun (phrase), B an adjective or article + noun (phrase)
\item Translate ``x is an A'' into a wff of FOL: A(x)
\item Translate ``x is B''  into a wff of FOL: B(x)
\item Combine A(x), B(x) in the right way
\eit

}

\subsec{Some A is B}{

\bit
\item ``Some cube is large''
\item ``x is a cube'': $\sf Cube(x)$
\item ``x is large'': $\sf Large(x)$
\item Combine:
\[\sf
\uncovers{2-4}{\exists x}\uncovers{2-4}{(}Cube(x) \uncovers{3-4}{{}\land{}} Large(x)\uncovers{2-4}{)}
\]
\items{4} Also:
\bit
\item ``Some cubes are large''
\item ``There are large cubes''
\item ``Something large is a cube''
\item ``A cube is large''
\eit
\eit

}


\subsec{Every A is B}{

\bits
\item ``Every cube is large''
\item Options:
\bens
\item[] $\sf \forall x(Cube(x) \phantom{{}\land{}} Large(x))$
\item $\sf \forall x(Cube(x) \land Large(x))$ \uncovers{7-}{\color{red}NO}
\item $\sf \forall x(Cube(x) \lor Large(x))$ \uncovers{7-}{\color{red}NO}
\item $\sf \forall x(Cube(x) \to Large(x))$ \uncovers{7-}{\color{green}YES}
\een
\item Also:
\bit
\item ``All cubes are large''
\item ``Any cube is large''
\item ``Cubes are large''
\item ``If something is a cube, it is large''
\eit
\eit

}

\subsec{The Indefinite Article}{

\bits
\item ``A small cube adjoins a''
\item[]
$\sf \exists x((Small(x) \land Cube(x)) \land Adjoins(x, a))$
\item ``a is left of a large tetrahedron''
\item[]$
\sf \exists x((Large(x) \land Tet(x)) \land LeftOf(a, x))
$
\item ``An even number is a multiple of 2''
\item[]
$\sf \forall x(Even(x) \to MultipleOf(x, (1+1)))$
\item ``If a cube adjoins a, it is large''
\item[]$\sf \forall x((Cube(x) \land Adjoins(x, a)) \to Large(x))$
\eit

}


\subsec{No A is B}{

\bits
\item ``No cube is large''
\item Options:
\bens
\item[] $\sf \phantom{lnot\forall} x(Cube(x) \phantom{{}\land{}} Large(x))$
\item $\sf \forall x(Cube(x) \to \lnot Large(x))$
\item $\sf \lnot\exists x(Cube(x) \land Large(x))$
\een
\item Also: 
\bit
\item ``No cubes are large''
\item ``There are no large cubes''
\item ``Nothing large is a cube''
\eit
\eit
}

\subsec{Only As are Bs}{

\bits
\item ``Only cubes are large''
\bits
\item ``All non-cubes are non-large''
\item $\sf \forall x(\lnot Cube(x) \to \lnot Large(x))$
\item $\sf \forall x(Large(x) \to Cube(x))$
\eit
\item ``Only a is large''
\bits
\item ``a is large, and nothing other than a is large''
\item $\sf Large(a) \land \forall x(x \neq a \to \lnot Large(x))$
\item $\sf \forall x(Large(x) \iff x = a)$
\eit
\eit
}

\subsec{Existential Import}{

\bit
\item Does ``all cubes are small'' have ``there are cubes'' as a consequence?
\item Not according to our translations!
\[
\sf\forall x(Cube(x) \to Small(x))
\]
is true if there are no cubes at all.
\item (1) ``Everyone who took the exam passed''\\
(2) ``Noone who took the exam failed''
\bit
\item (1) and (2) are equivalent
\item If noone took the exam, then (2) is true 
\eit
\eit
}

\subsec{Existential Import and Implicature}{

\bit
\item P \emph{implies} Q iff Q is a consequence of P 
\item If P implies Q, then the denial of Q contradicts P
\bit
\item ``Some A are B'' implies ``There are As''
\item ``Some cubes are small'' implies ``There are cubes''
\item ``Some cubes are small. There are no cubes'': contradictory
\eit
\item Does ``All cubes are small'' imply ``There are cubes?''
\eit
}

\subsec{Existential Import and Implicature}{

\bit
\item P \emph{implicates} Q if in asserting P, it is (strongly) suggested that Q is true 
\item Existential import is only \emph{implicated}, not \emph{implied}
\item Test: No contradiction if implicature is denied:
\bit

\item ``Noone who took the exam failed.\\
In fact, noone took the exam at all''

\item ``Some students passed the exam.\\
In fact, all students passed.''

\item ``All unicorns are white.\\
All zero of them.''
\eit\eit

}

\subsec{Quantifiers and Function Symbols}{

\bits
\item ``The leftmost block in the same row as any cube is small''
\bits
\item ``For every cube, the leftmost block in the same column as it is small''
\item
\[
\sf \forall x(Cube(x) \to Small(lm(x)))
\]
\eit
\item ``Some multiples of 3 are even, and some aren't''
\bits
\item Note: every multiple of 3 is the multiple of 3 \emph{by some number}
\item ``There is a number such that the multiple of 3 by it is even''
\item 
$
\sf \exists x\,Even(x \times (1 + (1+1)))
$
\item
$
\sf \exists x\, Even(x \times (1 + (1+1))) \land \exists x\, \lnot Even(x \times (1 + (1+1)))
$
\eit
\eit
}


\end{document}




