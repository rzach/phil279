\documentclass[bigger]{beamer}

\usetheme{Frankfurt}
\useoutertheme[subsection=false]{miniframes}
\setbeamercolor{structure}{parent=palette quaternary,fg=uofcred}
\setbeamercolor{section in toc}{fg=black,bg=white}
\setbeamercolor{alerted text}{fg=uofcred}
\setbeamercolor*{palette quaternary}{fg=white,bg=black}
\setbeamercolor*{palette secondary}{fg=black,bg=uofcyellow}
\setbeamercolor*{palette tertiary}{bg=uofcred,fg=white}
\setbeamercolor*{palette primary}{fg=white,bg=black}
\setbeamercolor*{sidebar}{fg=uofcred,bg=black!15!white}
\setbeamercolor*{palette sidebar primary}{fg=uofcred!10!black}
\setbeamercolor*{palette sidebar secondary}{fg=white}
\setbeamercolor*{palette sidebar tertiary}{fg=uofcred!50!black}
\setbeamercolor*{palette sidebar quaternary}{fg=black!10!white}

\setbeamercolor{titlelike}{parent=palette primary}
\setbeamercolor{frametitle}{fg=white,bg=black}
\setbeamercolor{frametitle right}{bg=gray!60!white}
\let\emph\alert
\beamertemplatenavigationsymbolsvertical
\def\bits{\begin{itemize}[<+->]}
\def\bens{\begin{enumerate}[<+->]}
\def\items#1{\item<#1>}
\def\uncovers#1{\uncover<#1>}
\def\pauses{\pause}

 % change to header-handout for handouts

% ====================
\title[Lecture 1]{Philosophy 279: Logic I}
\date{September 10, 2013}
% ====================

\logo{\includegraphics[height=2cm]{UC-vert-phil}}
% lecture header include

\usepackage{lplfitch,amsmath}
\usepackage{qtree,hyperref}
\usepackage{pgf,amssymb}

\author{Richard Zach}
\institute{Department of Philosophy\\
University of Calgary\\
  \href{http://ucalgary.ca/rzach/279}{ucalgary.ca/rzach/279}}

\definecolor{uofcred}{RGB}{227,39,38}
\definecolor{uofcyellow}{RGB}{255,210,0}

\DeclareSymbolFont{symbolsC}{U}{txsyc}{m}{n}
\DeclareMathSymbol{\strictif}{\mathrel}{symbolsC}{74}
\DeclareMathSymbol{\boxright}{\mathrel}{symbolsC}{128}

\let\IFF\Leftrightarrow
\let\iff\leftrightarrow
\let\impl\to

\def\T{{\color{green}\begin{colormixin}{25!black}\text{T}\end{colormixin}}}
\def\F{{\color{red}\begin{colormixin}{25!black}\text{F}\end{colormixin}}}

\long\def\subsec#1#2{\subsection{#1}\frame{\frametitle{#1} #2}}
\def\bit{\begin{itemize}[<1->]}
\def\eit{\end{itemize}}
\def\ben{\begin{enumerate}[<1->]}
\def\een{\end{enumerate}}

\makeatletter\let\@makefnmark\noindent\makeatother

%\setbeamercolor{footnote}{fg=black!70}

\def\foot#1{\footnotetext{\color{black!70}#1}}
\def\deemph#1{{\color{black!70}#1}}
\let\phi\varphi

\setbeamertemplate{theorems}[numbered]
%\useinnertheme{circles}
\setbeamertemplate{itemize subitems}[triangle]

\renewcommand{\beamertemplatetransparentcovereddynamic}{
  \beamersetuncovermixins
  {\opaqueness<1>{50}\opaqueness<2>{30}\opaqueness<3>{15}\opaqueness<4->{5}}%
  {\opaqueness<1>{50}\opaqueness<2>{30}\opaqueness<3>{15}\opaqueness<4->{5}}}
                                                                                
%\beamertemplatetransparentcovereddynamic

\defbeamertemplate*{footline}{my theme}
{%
  \leavevmode%
  \hbox{\begin{beamercolorbox}[wd=.5\paperwidth,ht=2.5ex,dp=1.125ex,leftskip=.3cm,rightskip=.3cm]{author in head/foot}%
    \insertframenumber/\inserttotalframenumber \hfil \usebeamerfont{author in head/foot}\insertshortauthor
  \end{beamercolorbox}%
  \begin{beamercolorbox}[wd=.5\paperwidth,ht=2.5ex,dp=1.125ex,leftskip=.3cm,rightskip=.3cm plus1fil]{title in head/foot}%
    \usebeamerfont{title in head/foot}Logic I F13---\insertshorttitle---\insertdate
  \end{beamercolorbox}}%
  \vskip0pt%
}

\begin{document}
\setlength{\fitchargwidth}{7em}
\setlength{\fitchprfwidth}{7em}

\frame{\frametitle{\insertshorttitle\ (\insertdate)}
\tableofcontents[hidesubsections]
}









\beamertemplatenavigationsymbolsempty
\frame{
\titlepage
}
\logo{}
\frame{\frametitle{\insertshorttitle\ (\insertdate)}
\tableofcontents[hidesubsections]
}


\section{Course Info}
\subsection{}

\subsec{Instructors}{

\bit
\item Richard Zach (Prof)
\item Jared Houston (TA)
\item Dan Kary (TA)
\item Samara Burns (PASS Leader)
\bit
\item Find us on 12th floor of Social Sciences
\eit\eit
}

\subsec{Textbook and Software}{

\bit
\item John Barker-Plummer, Jon Barwise and John Etchemendy,\\
 \textit{Language, Proof, and
  Logic}, CSLI Press (2nd ed)
\item OpenProof software (included)
\bit
\item Tarski's World
\item Boole
\item Fitch
\eit
\item GradeGrinder grading service
\bit
\item Must buy \emph{new} copy, but
\item Electronic-only (cheaper) option available
\item Textbook and software available on computers in AFCL (018 SS)
\eit
\eit
}

\subsec{Evaluation}{

\bit
\item 6 Homework Assignments (50\%)
\bit
\item Lowest mark dropped
\item Must turn in all 6 within a week of due date
\eit
\item 3 Tests (45\%)
\bit
\item Thursday, Oct 17 (15\%)
\item Thursday, Nov 14 (20\%)
\item Thursday, Dec 5 (10\%)
\item In class, 75 minutes, closed book
\eit
\item Participation (5\%)
\bit
\item 5 posts on BlackBoard discussion board or in lecture/tutorial/lab
\eit
\eit
}


\subsec{Course Components}{

\bit
\item Lectures
\item Tutorials
\item Lab Workshop (SS 018)
\item PASS Sessions
\bit
\item Tutorials start next week
\item You have choice between lab and tutorial starting third week
\item Attendance not mandatory but strongly suggested
\item PASS sessions start third week
\eit
\item BlackBoard website discussion forum
\bit
\item Be civil and behave like adults
\item Don't give away answers
\eit
\eit
}

\subsec{Read the Outline and FAQ!}{

\bit
\item Official outline covers all policy questions
\item Syllabus (in outline) has all the dates on it
\item Outline is binding agreement and you are responsible for knowing policies
\item FAQ on \href{http://ucalgary.ca/rzach/279}{Logic I website}
will answer most of your questions, including:
\bit
\item Why do you make me take this course?
\item Why is the grade scale set the way it is?
\item How hard is this course?
\item Why do CPSC and PHIL need at least a C--?
\eit
\eit

}

\section{What is Logic?}

\subsec{What is Logic?}{

\bit
\item \emph{Logic is the science of what follows from what.}
\item Valid and invalid inference
\bit
\item Socrates is human. All humans are mortal.\\ 
Therefore, Socrates is mortal.
\item Socrates is human. All Greeks are human.\\ 
Therefore, Socrates is Greek.
\eit
\item Logic investigates what makes the first inference \emph{valid} and the
  second \emph{invalid}.
  \eit }



\subsec{What is Formal Logic?}{

\bit
\item Studies logical properties of formal languages (not English)
\bit
\item Logical consequence (what follows from what?)
\item Logical consistency (when do sentences contradict one another?)
\eit
\item Expressive power (what can be expressed in a given formal
  language, and how?)
\item Formal models (mathematical structures described by formal language)
\item Inference and proof systems (how can it be proved that something
  follows from something else?)
\item (Metalogical properties of logical systems)
\eit 
}

\subsec{What is Logic Good For? (Philosophy)}{

\bit
\item (Informal) logic originates in philosophy (Aristotle)
\item Valid inference cornerstone of philosophical research
\item Logic itself a subdiscipline of philosophy
\item Formal tools of logic useful to make intuitive
philosophical notions precise
\bit
\item Possibility and necessity
\item Time
\item Moral obligation and permissibility
\item Belief and knowledge
\eit
\eit

}

\subsec{What is Logic Good For? (Mathematics)}{

\bit
\item Formal logic developed in the quest for foundations of mathematics (19th C.)
\item Logical systems provide precise foundational framework for mathematics
\bit
\item Axiomatic systems (e.g, geometry)
\item Algebraic structures (e.g., groups)
\item Set theory (e.g, Zermelo-Fraenkel with Choice)
\eit
\item Formal methods make mathematics more precise
\bit
\item Formal language can make mathematical claims more precise
\item Formal structures can point to alternatives, unveil gaps in proofs
\item Formal proof systems make proofs rigorous
\item Formal proofs make mechanical \emph{proof checking} and \emph{proof search} possible
\eit
\item Logical tools can be applied to mathematical problems
\eit

}


\subsec{What is Logic Good For? (Computer Science)}{

\bit
\item Combinational logic circuits
\item Database query languages
\item Logic programming
\item Knowledge representation
\item Automated reasoning
\item Formal specification and verification (of programs, of hardware designs)
\item Theoretical computer science (theory of computational
  complexity, semantics of programming languages)
 \eit

}

\section{The Formal Language(s) of FOL}

\subsec{Functions of Language}{

\bit
\item Pick out objects in the ``world'' 
\item Say things about these objects
\item Different categories of words play different roles
\bit
\item Names (``Alice''), pronouns (``she''), demonstratives
  (``this''), certain phrases (``the Queen of England in 2011''):\\ pick
  out things (``singular terms'')
\item Verbs, adjectives, common nouns (in combination with other
  words): \\ used to say things about the objects picked out by singular
  terms
 \eit 
\item \emph{Predicating:} ascribing a property or relation to object(s)
\eit

}

\subsec{Simple Examples of Predication}{

\bit
\item Name + ``is'' + adjective\\
``Mars is red''\\
``Jupiter is large''
\item Name + ``is'' + comparative + ``than'' + Name\\
``Mars is smaller than Jupiter'' 
\item Name + ``is a''  + common noun\\
``Venus is a planet''
\item Name + verb (+ Name)
``The Earth moves''\\
``Phobos orbits Mars''\\
\eit

}

\subsec{First-order Languages}{

\bit
\item \emph{Individual constants} pick out things
\[
\sf a, b, c, \dots, mars, jupiter, \dots
\]
\item \emph{Predicate symbols} express properties and relations
\[
\sf Red, Large, Larger, Planet, Orbits, \le, =, \dots
\]
\item To say something about something, combine predicate symbols
  andindividual constants
\[
\sf Red(mars), Planet(mars),
Orbits(phobos, mars),
a = b
\]
(atomic sentences)
\eit
}

\section{The Blocks Language}

\subsec{The Blocks Language}{

\begin{tabular}{@{}ll@{}}
Tet(a) & a is a tetrahedron \\
Cube(a) & a is a cube \\
Dodec(a) & a is a dodecahedron \\
Small(a) & a is small \\
Medium(a) & a is medium \\
Large(a) & a is large \\[2ex]
SameSize(a, b) & a is the same size as b \\
SameShape(a, b) & a is the same shape as b\\
Larger(a, b) & a is larger than b \\
Smaller(a, b) & a is smaller than b\\
\end{tabular}

}

\subsec{The Blocks Language}{

\begin{tabular}{@{}ll@{}}
SameCol(a, b) & a is in the same column as b \\
SameRow(a, b) & a is in the same row as b\\
Adjoins(a, b) & a and b are located on adjacent\\ 
& (but not diagonally) squares \\
LeftOf(a, b) & a is located nearer to the left edge\\ & of the grid than b \\
RightOf(a, b) & a is located nearer to the right edge\\ & of the grid than b\\
FrontOf(a, b) & a is located nearer to the front\\ & of the grid than b\\
BackOf(a, b) & a is located nearer to the back\\ & of the grid than b\\
Between(a, b, c) & a, b and c are in the same row, column,\\ 
& or diagonal, and a is between b and c
\end{tabular}

}


\end{document}
