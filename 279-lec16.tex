\documentclass[bigger]{beamer}

\usetheme{Frankfurt}
\useoutertheme[subsection=false]{miniframes}
\setbeamercolor{structure}{parent=palette quaternary,fg=uofcred}
\setbeamercolor{section in toc}{fg=black,bg=white}
\setbeamercolor{alerted text}{fg=uofcred}
\setbeamercolor*{palette quaternary}{fg=white,bg=black}
\setbeamercolor*{palette secondary}{fg=black,bg=uofcyellow}
\setbeamercolor*{palette tertiary}{bg=uofcred,fg=white}
\setbeamercolor*{palette primary}{fg=white,bg=black}
\setbeamercolor*{sidebar}{fg=uofcred,bg=black!15!white}
\setbeamercolor*{palette sidebar primary}{fg=uofcred!10!black}
\setbeamercolor*{palette sidebar secondary}{fg=white}
\setbeamercolor*{palette sidebar tertiary}{fg=uofcred!50!black}
\setbeamercolor*{palette sidebar quaternary}{fg=black!10!white}

\setbeamercolor{titlelike}{parent=palette primary}
\setbeamercolor{frametitle}{fg=white,bg=black}
\setbeamercolor{frametitle right}{bg=gray!60!white}
\let\emph\alert
\beamertemplatenavigationsymbolsvertical
\def\bits{\begin{itemize}[<+->]}
\def\bens{\begin{enumerate}[<+->]}
\def\items#1{\item<#1>}
\def\uncovers#1{\uncover<#1>}
\def\pauses{\pause}

 % change to header-handout for handouts

% ====================
\title[Lecture 16]{Logic I F13 Lecture 16}
\date{November 5, 2013}
% ====================

% lecture header include

\usepackage{lplfitch,amsmath}
\usepackage{qtree,hyperref}
\usepackage{pgf,amssymb}

\author{Richard Zach}
\institute{Department of Philosophy\\
University of Calgary\\
  \href{http://ucalgary.ca/rzach/279}{ucalgary.ca/rzach/279}}

\definecolor{uofcred}{RGB}{227,39,38}
\definecolor{uofcyellow}{RGB}{255,210,0}

\DeclareSymbolFont{symbolsC}{U}{txsyc}{m}{n}
\DeclareMathSymbol{\strictif}{\mathrel}{symbolsC}{74}
\DeclareMathSymbol{\boxright}{\mathrel}{symbolsC}{128}

\let\IFF\Leftrightarrow
\let\iff\leftrightarrow
\let\impl\to

\def\T{{\color{green}\begin{colormixin}{25!black}\text{T}\end{colormixin}}}
\def\F{{\color{red}\begin{colormixin}{25!black}\text{F}\end{colormixin}}}

\long\def\subsec#1#2{\subsection{#1}\frame{\frametitle{#1} #2}}
\def\bit{\begin{itemize}[<1->]}
\def\eit{\end{itemize}}
\def\ben{\begin{enumerate}[<1->]}
\def\een{\end{enumerate}}

\makeatletter\let\@makefnmark\noindent\makeatother

%\setbeamercolor{footnote}{fg=black!70}

\def\foot#1{\footnotetext{\color{black!70}#1}}
\def\deemph#1{{\color{black!70}#1}}
\let\phi\varphi

\setbeamertemplate{theorems}[numbered]
%\useinnertheme{circles}
\setbeamertemplate{itemize subitems}[triangle]

\renewcommand{\beamertemplatetransparentcovereddynamic}{
  \beamersetuncovermixins
  {\opaqueness<1>{50}\opaqueness<2>{30}\opaqueness<3>{15}\opaqueness<4->{5}}%
  {\opaqueness<1>{50}\opaqueness<2>{30}\opaqueness<3>{15}\opaqueness<4->{5}}}
                                                                                
%\beamertemplatetransparentcovereddynamic

\defbeamertemplate*{footline}{my theme}
{%
  \leavevmode%
  \hbox{\begin{beamercolorbox}[wd=.5\paperwidth,ht=2.5ex,dp=1.125ex,leftskip=.3cm,rightskip=.3cm]{author in head/foot}%
    \insertframenumber/\inserttotalframenumber \hfil \usebeamerfont{author in head/foot}\insertshortauthor
  \end{beamercolorbox}%
  \begin{beamercolorbox}[wd=.5\paperwidth,ht=2.5ex,dp=1.125ex,leftskip=.3cm,rightskip=.3cm plus1fil]{title in head/foot}%
    \usebeamerfont{title in head/foot}Logic I F13---\insertshorttitle---\insertdate
  \end{beamercolorbox}}%
  \vskip0pt%
}

\begin{document}
\setlength{\fitchargwidth}{7em}
\setlength{\fitchprfwidth}{7em}

\frame{\frametitle{\insertshorttitle\ (\insertdate)}
\tableofcontents[hidesubsections]
}










\setlength{\fitchprfwidth}{5em}

\section{Ambiguity}

\subsec{Types of Ambiguity}{

\bit
\item Lexical ambiguity: same word, different meaning
\item Structural ambiguity: different ways to read a phrase/sentence
\item \emph{Scope ambiguity}: which connective or quantifier is in the
  scope of which other connective or quantifier\\[2ex]

a is left of b and large or left of c
\begin{align*}
\sf LeftOf(a, b) & \sf{} \land (Large(a) \lor RightOf(a, c)) \\
\sf (LeftOf(a, b) & \sf{} \land Large(a)) \lor RightOf(a, c)
\end{align*}
\eit

}

\subsec{``Anything''}{

\bit
\item Anything {\color{blue}left of a cube} is large\pauses
\[\sf
\uncovers{3-}{\forall x(}{\color{blue}\exists y(Cube(y) \land LeftOf(x, y))} \uncovers{3-}{{}\to Large(x))}
\]\pauses
\item {\color{red}No cube is} left of anything large \pauses
\[\sf
{\color{red}\forall x(Cube(x) \to \lnot} \uncovers{5-}{\exists y(Large(y) \land LeftOf(x, y))}{\color{red})}
\]\pauses
\item {\color{red}No cube is} left of everything large \pauses
\[\sf
{\color{red}\forall x(Cube(x) \to \lnot} \forall y(Large(y) \to LeftOf(x, y)){\color{red})}
\]
\eit

}

\subsec{Negation and the Quantifiers}{

\bit
\item ``All cubes are not large''
\bit
\item Denial of ``all cubes are large''\\
(``Are all cubes large? No, all cubes are not large'')\pauses
\begin{align*}
\sf\lnot\forall x(Cube(x) & \sf{} \to Large(x)) \\
\sf\exists x(Cube(x) & \sf{} \land\lnot Large(x)) 
\end{align*}
\item All cubes are: not large, i.e.,\\
No cubes are large\pauses
\begin{align*}
\sf\forall x(Cube(x) & \sf{} \to \lnot Large(x)) \\
\sf\lnot\exists x(Cube(x) & \sf{}\land Large(x)) 
\end{align*}
\eit
\eit
}

\subsec{Multiple Quantifiers and Ambiguity}{

\bit
\item ``All cubes adjoin a tetrahedron''
\bit
\item ``A tetrahedron'' in the scope of ``all cubes'', i.e.,\\
``For every cube, there is a tetrahedron it adjoins''\pauses
\[
\sf\forall x(Cube(x) \to \exists y(Tet(y) \land Adjoins(x, y))
\]
\item ``All cubes'' in scope of ``a tetrahedron'', i.e.,\\
``There is a tetrahedron which every cube adjoins''\pauses
\[
\sf\exists y(Tet(y) \land \forall x(Cube(x) \to Adjoins(x, y))
\]
\eit\eit
}


\section{Anaphora and Uniqueness}

\subsec{Anaphora (Donkey Sentences)}{

``Every farmer who owns a donkey is happy''

\bits
\item Step by step translation: ``All As are Bs''
\item x is a farmer who owns a donkey \dots\[\sf
Farmer(x) \land \exists y(Donkey(y) \land Owns(x, y))
\] 
\item {\color{blue}Every} farmer who owns a donkey {\color{blue}is happy}
\[\sf
{\color{blue}\forall x(}(Farmer(x) \land \exists y(Donkey(y) \land Owns(x, y))) {\color{blue}\to Happy(x))}
\]
\eit

}

\subsec{Anaphora (Donkey Sentences)}{

``Every farmer who owns a donkey beats it''

\bits
\item Step by step translation: ``All As are Bs''
\item x is a farmer who owns a donkey \dots\[\sf
Farmer(x) \land \exists y(Donkey(y) \land Owns(x, y))
\] 
\item {\color{blue}Every} farmer who owns a donkey {\color{blue}beats it}
\[\sf
{\color{blue}\forall x(}(Farmer(x) \land \exists y(Donkey(y) \land Owns(x, y))) {\color{blue}\to Beats(x, {\color{red}y}))}
\]
\eit

}

\subsec{Anaphora (Donkey Sentences)}{

``Every farmer who owns a donkey beats it''\pauses

\bits
\item When is it false that every farmer who owns a donkey beats it? \pauses 
If there's a farmer who owns a donkey but doesn't beat it. Deny that! \pauses
\[\sf
\lnot\exists x(Farmer(x) \land \exists y(Donkey(y) \land Owns(x, y) \land \lnot Beats(x, y)))
\]
\item For every farmer and every donkey: if the farmer owns the
  donkey, the farmer beats the donkey
\[\sf
\forall x\forall y((Farmer(x) \land Donkey(y)) \to (Owns(x, y) \to Beats(x, y)))
\]
\item Every farmer beats every donkey they own
\[\sf
\forall x(Farmer(x) \to \forall y((Donkey(y) \land Owns(x, y)) \to Beats(x, y)))
\]
\eit

}

\subsec{Uniqueness}{

\bit
\item There is at least one cube.
\[
\sf\exists x\, Cube(x)\]
\item There is exactly one cube.
\bit
\item<2-> There's at least one cube, and
\item<3-> There are no others
\begin{align*}
\uncover<2->{\sf\exists x\, (Cube(x)} & \uncover<4->{\sf\land \lnot \exists y\, (y \neq x \land Cube(y)))}\\
\uncover<3->{\sf\exists x\,(Cube(x)} & \uncover<5->{\sf\land \forall y(Cube(y) \to x = y))}
\end{align*}
\eit
\eit
}

\section{Rules for $\forall$}

\subsec{Rules for Fitch}{

\bit
\item Need rules for $\forall$ and $\exists$ for formal proofs
\item Formal proofs now more important, because no alternative (truth-table method)
\item Intro and Elim rules should be
\bit
\item simple
\item elegant (not involve other connectives or quantifiers)
\item yield only valid arguments
\eit 
\eit

}

\subsec{Candidates for Rules}{
\setlength{\fitchprfwidth}{6em}

\bit
\item Only simple sentence close to $\sf \forall x\, A(x)$ is $\sf A(c)$
\item Gives simple, elegant $\forall$Elim rule:

\fitchctx{
\nline[$k.$]{\forall x\, A(x)}\\
\fpline{\quad A(c)}[\lalle{}]
}
\item Problem: corresponding ``intro rule'' isn't valid:
\fitchctx{
\nline[$k.$]{A(c)}\\
\fpline{\quad\forall x\,A(x)}[\emph{doesn't follow!}]
}
\eit

}

\subsec{Names for Arbitrary Objects}{

\bit
\item Diagnosis: the c in A(c) is a name for a \emph{specific object}
We need a name for an \emph{arbitrary, unspecified object}
\item If A(c) is true for whatever c could name, then A(x) is satisfied by \emph{every} object 
\item In fact, this is what we do when we give proofs of general claims, e.g.,

\setlength{\fitchargwidth}{15em}
\fitcharg{\tline{All cubes adjoin a}\\
\tline{Only small things adjoin a}}{
\tline{All cubes are small}}

\pauses
Proof: Let Carl be any cube.  Since all cubes adjoin a, Carl adjoins
a. Since only small things adjoin a, Carl is small. But ``Carl''
stands for \emph{any} cube. So all cubes are small.
  \eit

}

\subsec{Arbitrary Objects in Fitch}{

\bit 
\item Every special constant with stands for an arbitrary object is
  introduced only pro tem
\item In particular, they never appear in the premises or conclusion of an argument
\item In Fitch, this is enforced by requiring that special constants only appear in subproofs
\item To indicate that a constant names an arbitrary object, it is put in a box at the beginning of this subproof
\eit

}

\subsec{General Conditional Proof}{
\setlength{\fitchctxwidth}{11em}

\fitchctx{
\boxedsubproof[$m$.]{c}{A(c)}{\ellipsesline \\ \nline[$n.$]{B(c)}}
\fpline{\quad \forall x(A(x) \to B(x))}[\lalli{$m$--$n$}]}

\bit
\item c is special: c must not appear anywhere outside the subproof
\eit
}

\subsec{Universal Generalization}{
\setlength{\fitchprfwidth}{11em}

\fitchctx{
\boxedsubproof[$m$.]{c}{}{\ellipsesline \\ \nline[$n.$]{A(c)}}
\fpline{\quad \forall x\,A(x)}[\lalli{$m$--$n$}]}
\bit
\item c is special: c must not appear anywhere outside the subproof
\eit
}

\subsec{Example}{

\setlength{\fitchargwidth}{15em}
\fitcharg{\tline{All cubes adjoin a}\\
\tline{Only small things adjoin a}}{
\tline{All cubes are small}}

\bigskip

\fitchprf{\pline{\forall x\, (Cube(x) \to Adjoins(x, a))}\\
\pline{\forall x(Adjoins(x, a) \to Small(x))}}{
\pline{\forall x(Cube(x) \to Small(x))}}

}

\subsec{Example}{

\setlength{\fitchprfwidth}{15em}

\fitchprf{\pline[1.]{\forall x\, (Cube(x) \to Adjoins(x, a))}\\
\pline[2.]{\forall x(Adjoins(x, a) \to Small(x))}}{
\boxedsubproof[3.]{c}{Cube(c)}{\pline[4.]{Cube(c) \to Adjoins(c, a)}[\lalle{1}] \\
\pline[5.]{Adjoins(c, a)}[\life{3}{4}]\\
\pline[6.]{Adjoins(c, a) \to Small(c)}[\lalle{2}] \\
\pline[7.]{Small(c)}[\life{5}{6}]}
\pline[8.]{\forall x(Cube(x) \to Small(x))}[\lalli{3--7}]}

}

\section{Rules for $\exists$}

\subsec{Intro Rule for $\exists$}{
\setlength{\fitchprfwidth}{7em}
\bit
\item If we know of a specific object that it satisfies A(x), we know that at least one object satisfies A(x)
\item So this rule is valid:

\fitchctx{\pline[$m.$]{A(c)}\\
\fpline{\quad \exists x\, A(x)}[\lexii{$m$}]}
\item Problem: corresponding ``elim rule'' isn't valid:
\fitchctx{
\nline{\exists x\, A(x)}\\
\fpline{\quad A(c)}[\emph{doesn't follow!}]
}


\eit

}

\subsec{Arbitrary Objects Again}{

\bit
\item If we know that $\sf\exists x\, A(x)$ is true, we know that \emph{some} object(s) satisfy A(x), but not which ones
\item To use this information, we have to introduce a temporary name that stands for any one of the objects that satisfy A(x)
\item This is what we'd do if we reason informally from existential information, e.g.,

\setlength{\fitchprfwidth}{15em}
\fitchprf{\tline{There are small cubes}\\
\tline{Anything small adjoins a}}{
\tline{Some cubes adjoin a}}

\pauses
Proof: We know there are small cubes. Let Cate be an arbitrary one of them.  So
Cate is small. Since all small things adjoins a, Cate adjoins a. Since
Cate is a cube adjoining a, some cubes adjoin a.
\eit 

}

\subsec{Existential Instantiation}{
\setlength{\fitchprfwidth}{7em}

\bit
\item If we know that some object satisfies A(x), we assume for the time being that c is one of them (i.e., assume A(c)), and we can prove that some claim B fololows from this assumption, then B follows already from $\sf\exists x\, A(x)$.
\item Rule for existential elimination:
\bigskip

\fitchctx{
\pline[$k$.]{\exists x\, A(x)}\\
\boxedsubproof[$m$.]{c}{A(c)}{\ellipsesline \\ \nline[$n.$]{B}}
\fpline{\quad B}[\lexie{k}{$m$--$n$}]}
\item c is special: c must not appear anywhere outside the subproof
\eit
}

\subsec{Example}{

\setlength{\fitchprfwidth}{15em}
\fitchprf{\tline{There are small cubes}\\
\tline{Anything small adjoins a}}{
\tline{Some cubes adjoin a}}

\bigskip

\fitchprf{\pline{\exists x(Small(x) \land Cube(x))}\\
\pline{\forall x(Small(x) \to Adjoins(x, a))}}{
\pline{\exists x(Cube(x) \land Adjoins(x, a)}}
}

\subsec{Example}{

\setlength{\fitchprfwidth}{15em}
\fitchprf{\pline[1.]{\exists x(Small(x) \land Cube(x))}\\
\pline[2.]{\forall x(Small(x) \to Adjoins(x, a))}}{
\boxedsubproof[3.]{c}{Small(c) \land Cube(c)}{
\pline[4.]{Small(c)}[\lande{3}] \\
\pline[5.]{Small(c) \to Adjoins(c, a)}[\lalle{2}] \\
\pline[6.]{Adjoins(c, a)}[\life{4}{5}]\\
\pline[7.]{Cube(c)}[\lande{3}] \\
\pline[8.]{Cube(c) \land Adjoins(c, a)}[\landi{6}{7}]\\
\pline[9.]{\exists x(Cube(x) \land Adjoins(x, a)}[\lexii{8}]
}
\pline[10.]{\exists x(Cube(x) \land Adjoins(x, a)}[\lexie{1}{3--9}]
}

}
\end{document} 




